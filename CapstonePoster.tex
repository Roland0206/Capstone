%%%%%%%%%%%%%%%%%%%%%%%%%%%%%%%%%%%%%%%%%
% a0poster Landscape Poster
% LaTeX Template
% Version 1.0 (22/06/13)
%
% The a0poster class was created by:
% Gerlinde Kettl and Matthias Weiser (tex@kettl.de)
% 
% This template has been downloaded from:
% http://www.LaTeXTemplates.com
%
% License:
% CC BY-NC-SA 3.0 (http://creativecommons.org/licenses/by-nc-sa/3.0/)
%
%%%%%%%%%%%%%%%%%%%%%%%%%%%%%%%%%%%%%%%%%

%----------------------------------------------------------------------------------------
%	PACKAGES AND OTHER DOCUMENT CONFIGURATIONS
%----------------------------------------------------------------------------------------

\documentclass[a1,portrait]{a0poster}
\usepackage{multicol} % This is so we can have multiple columns of text side-by-side
\columnsep=70pt % This is the amount of white space between the columns in the poster
\columnseprule=3pt % This is the thickness of the black line between the columns in the poster

\usepackage[svgnames]{xcolor} % Specify colors by their 'svgnames', for a full list of all colors available see here: http://www.latextemplates.com/svgnames-colors

\usepackage{times} % Use the times font
%\usepackage{palatino} % Uncomment to use the Palatino font

\usepackage{graphicx} % Required for including images
\graphicspath{{figures/}} % Location of the graphics files
\usepackage{booktabs} % Top and bottom rules for table
\usepackage[font=small,labelfont=bf]{caption} % Required for specifying captions to tables and figures
\usepackage{amsfonts, amsmath, amsthm, amssymb} % For math fonts, symbols and environments
\usepackage{wrapfig} % Allows wrapping text around tables and figures
\usepackage{svg}
\usepackage{pgf}
\usepackage{pgfplots}
\usepackage{tikz}
\usepgfplotslibrary{groupplots}
\begin{document}

%----------------------------------------------------------------------------------------
%	POSTER HEADER 
%----------------------------------------------------------------------------------------

% The header is divided into three boxes:
% The first is 55% wide and houses the title, subtitle, names and university/organization
% The second is 25% wide and houses contact information
% The third is 19% wide and houses a logo for your university/organization or a photo of you
% The widths of these boxes can be easily edited to accommodate your content as you see fit

\begin{minipage}{0.75\linewidth}
\veryHuge \color{NavyBlue} \textbf{How to build muscle} \color{Black}\\ % Title
\Huge\textit{Image and data analysis of myofibrillogenesis}\\[1cm] % Subtitle
\LARGE{Nico Wrobel, Tim Burghardt \& Roland Brandau}\\ % Author(s)
\end{minipage}%
\begin{minipage}{0.25\linewidth}
\centering
\includesvg[width=0.9\linewidth]{figures/TU_Dresden_Logo_blau_HKS41.svg}
\end{minipage}

\vspace{1cm} % A bit of extra whitespace between the header and poster content

%----------------------------------------------------------------------------------------

\begin{multicols}{2} % This is how many columns your poster will be broken into, a poster with many figures may benefit from less columns whereas a text-heavy poster benefits from more


%----------------------------------------------------------------------------------------
%	INTRODUCTION
%----------------------------------------------------------------------------------------

\color{SaddleBrown} % SaddleBrown color for the introduction

\section*{Introduction}
Striated muscle cells contain bundles of so-called myofibrils. These are characterized by highly regular periodic units called sarcomeres, each bordered by two Z-discs. It is an open question as to how the first periodic structures are formed during the development of the muscle (see reference).\\
To analyze the self-assembly process and answer this question, Francine Kolley (and co, reference) developed an image analysis algorithm that detects emerging periodic patterns based on autocorrelation functions along myofibrils. Using a steerable filter to determine local nematic order, the algorithm was applied primarily to images of Drosophila (fruit fly) muscle. To make it more applicable to myofibrillogenesis in human muscle cells, we have modified the algorithm (list changes?). The analysis of human data is more complex compared to fly data, and presents challenges such as... (elaborate on challenges, 2-pics side by side?).\\
\begin{minipage}{0.45\textwidth}
    \centering
    %\begin{tikzpicture}
% First plot
% Calculate the aspect ratio of the image

\begin{scope}
    \begin{axis}[
    hide x axis,
    hide y axis,
    height = 0.25\textwidth,
    width = 0.25\textwidth,
    tick align=outside,
    tick pos=left,
    title={Sallimus (fly data)},
    x grid style={darkgray176},
    xmin=-0.5, xmax=1945.5,
    xtick style={color=black},
    y dir=reverse,
    y grid style={darkgray176},
    ymin=-0.5, ymax=1945.5,
    ytick style={color=black}
    ]
    \addplot graphics [includegraphics cmd=\pgfimage,xmin=-0.5, xmax=1945.5, ymin=1945.5, ymax=-0.5] {figures/raw_fly-000.png};
    \end{axis}
\end{scope}

% Calculate the aspect ratio of the image
\pgfmathsetmacro{\xmax}{6540/2}
\pgfmathsetmacro{\aspectratio}{\xmax / 1946 }

% Second plot
\begin{scope}[xshift=0.25\textwidth] % Shift to the right for the second plot
    \begin{axis}[
    hide x axis,
    hide y axis,
    height = 0.25\textwidth,
    width = 0.25\textwidth*\aspectratio, % Set the width based on the aspect ratio
    tick align=outside,
    tick pos=left,
    title={Titin (human data)},
    x grid style={darkgray176},
    xmin=-0.5, xmax=\xmax,%6539.5,
    xtick style={color=black},
    y dir=reverse,
    y grid style={darkgray176},
    ymin=-0.5, ymax=1945.5,
    ytick style={color=black}
    ]
    \addplot graphics [includegraphics cmd=\pgfimage,xmin=-0.5, xmax=6539.5, ymin=1945.5, ymax=-0.5] {figures/raw_human-000.png};
    \end{axis}
\end{scope}
\end{tikzpicture}
\end{minipage}

On this poster, we show the application of the algorithm to myofibrillogenesis in human muscle cells and demonstrate its effectiveness with images illustrating key steps.

%----------------------------------------------------------------------------------------
%	OBJECTIVES
%----------------------------------------------------------------------------------------

\color{DarkSlateGray} % DarkSlateGray color for the rest of the content

\section*{Image Preprocessing}
To get optimal results when applying the steerable filter to the image, several steps have to be taken:
\begin{enumerate}
    \item Create a binary mask
    \begin{enumerate}
        \item Create a binary mask based on a threshold value \(t_{\text{mask}}\) for the pixel intensity \(I\): \(I \to 
        \begin{cases}
            1 & I<t_{\text{mask}}\\
            0 & I \geq t_{\text{mask}}
        \end{cases}\).
        \item Apply a Gaussian filter with standard deviation \(\sigma_{\text{gb}}\) to reduce noise that might interfere, e.g. with ridge detection.
        \item Exclude areas that overlap with nuclei: Therefore, we use a pretrained model \textit{2D\_versatile\_fluo} from the \textit{stardist} (ref) model to detect nuclei in the corresponding image. Afterward, we subtract the these regions from the mask.
    \end{enumerate}
    \begin{minipage}{0.5\textwidth}
        % This file was created with tikzplotlib v0.10.1.
\begin{tikzpicture}

\definecolor{darkgray176}{RGB}{176,176,176}

\begin{axis}[
height = \textwidth*0.29357798165137616,
hide x axis,
hide y axis,
tick align=outside,
tick pos=left,
title={Raw Mask ($t_{\text{mask}}$ = 350), $\sigma_{\text{gb}}$ = 2)},
width=\textwidth,
x grid style={darkgray176},
xmin=-0.5, xmax=6539.5,
xtick style={color=black},
y dir=reverse,
y grid style={darkgray176},
ymin=0, ymax=1920,
ytick style={color=black}
]
\addplot graphics [includegraphics cmd=\pgfimage,xmin=-0.5, xmax=6539.5, ymin=1919.5, ymax=-0.5] {raw_mask-003.png};
\addplot [red]
table {%
1446 59.2
1445 59.2
1444 59.2
1443 59.2
1442 59.2
1441 59.2
1440 59.2
1439 59.2
1438 59.2
1437.8 59
1437 58.2
1436 58.2
1435 58.2
1434 58.2
1433 58.2
1432.8 58
1432 57.2
1431 57.2
1430 57.2
1429.8 57
1429 56.2
1428 56.2
1427 56.2
1426.8 56
1426 55.2
1425 55.2
1424.8 55
1424 54.2
1423.8 54
1423 53.2
1422.8 53
1422 52.2
1421.8 52
1421 51.2
1420.8 51
1420 50.2
1419.8 50
1419 49.2
1418.8 49
1418.8 48
1418 47.2
1417.8 47
1417.8 46
1417 45.2
1416.8 45
1416 44.2
1415.8 44
1415.8 43
1415 42.2
1414.8 42
1414.8 41
1414.8 40
1414 39.2
1413.8 39
1413.8 38
1413.8 37
1413 36.2
1412.8 36
1412.8 35
1412.8 34
1412.8 33
1412.8 32
1412.8 31
1412.8 30
1412.8 29
1413 28.8
1413.8 28
1413.8 27
1413.8 26
1414 25.8
1414.8 25
1414.8 24
1415 23.8
1415.8 23
1415.8 22
1416 21.8
1416.8 21
1417 20.8
1417.8 20
1418 19.8
1418.8 19
1419 18.8
1419.8 18
1420 17.8
1420.8 17
1421 16.8
1421.8 16
1422 15.8
1422.8 15
1423 14.8
1423.8 14
1424 13.8
1425 13.8
1425.8 13
1426 12.8
1426.8 12
1427 11.8
1428 11.8
1428.8 11
1429 10.8
1430 10.8
1431 10.8
1431.8 10
1432 9.8
1433 9.8
1434 9.8
1435 9.8
1436 9.8
1437 9.8
1438 9.8
1438.8 9
1439 8.8
1440 8.8
1441 8.8
1441.2 9
1442 9.8
1443 9.8
1444 9.8
1445 9.8
1446 9.8
1446.2 10
1447 10.8
1448 10.8
1449 10.8
1449.2 11
1450 11.8
1451 11.8
1451.2 12
1452 12.8
1453 12.8
1453.2 13
1454 13.8
1454.2 14
1455 14.8
1456 14.8
1456.2 15
1457 15.8
1457.2 16
1458 16.8
1458.2 17
1459 17.8
1459.2 18
1460 18.8
1460.2 19
1461 19.8
1461.2 20
1462 20.8
1462.2 21
1463 21.8
1463.2 22
1463.2 23
1464 23.8
1464.2 24
1464.2 25
1465 25.8
1465.2 26
1465.2 27
1465.2 28
1466 28.8
1466.2 29
1466.2 30
1466.2 31
1466.2 32
1466.2 33
1467 33.8
1467.2 34
1467 34.2
1466.2 35
1466.2 36
1466.2 37
1466.2 38
1466.2 39
1466.2 40
1466.2 41
1466 41.2
1465.2 42
1465.2 43
1465.2 44
1465.2 45
1465 45.2
1464.2 46
1464.2 47
1464 47.2
1463.2 48
1463.2 49
1463 49.2
1462.2 50
1462 50.2
1461.2 51
1461 51.2
1460.2 52
1460 52.2
1459.2 53
1459 53.2
1458.2 54
1458 54.2
1457 54.2
1456.2 55
1456 55.2
1455 55.2
1454.2 56
1454 56.2
1453 56.2
1452.2 57
1452 57.2
1451 57.2
1450.2 58
1450 58.2
1449 58.2
1448 58.2
1447 58.2
1446.2 59
1446 59.2
};
\addplot [red]
table {%
1794 142.2
1793 142.2
1792 142.2
1791 142.2
1790 142.2
1789 142.2
1788 142.2
1787.8 142
1787 141.2
1786 141.2
1785 141.2
1784 141.2
1783 141.2
1782 141.2
1781.8 141
1781 140.2
1780 140.2
1779 140.2
1778 140.2
1777 140.2
1776 140.2
1775.8 140
1775 139.2
1774 139.2
1773 139.2
1772 139.2
1771.8 139
1771 138.2
1770 138.2
1769.8 138
1769 137.2
1768 137.2
1767.8 137
1767 136.2
1766 136.2
1765 136.2
1764.8 136
1764 135.2
1763 135.2
1762 135.2
1761 135.2
1760 135.2
1759 135.2
1758 135.2
1757.8 135
1757 134.2
1756 134.2
1755 134.2
1754 134.2
1753 134.2
1752 134.2
1751 134.2
1750.8 134
1750 133.2
1749 133.2
1748 133.2
1747 133.2
1746.8 133
1746 132.2
1745 132.2
1744 132.2
1743.8 132
1743 131.2
1742 131.2
1741 131.2
1740.8 131
1740 130.2
1739 130.2
1738 130.2
1737.8 130
1737 129.2
1736 129.2
1735 129.2
1734 129.2
1733.8 129
1733 128.2
1732 128.2
1731 128.2
1730.8 128
1730.8 127
1730.8 126
1731 125.8
1731.8 125
1731.8 124
1731.8 123
1731.8 122
1731.8 121
1732 120.8
1732.8 120
1732.8 119
1733 118.8
1733.8 118
1734 117.8
1735 117.8
1736 117.8
1737 117.8
1737.8 117
1738 116.8
1739 116.8
1740 116.8
1741 116.8
1741.8 116
1742 115.8
1743 115.8
1744 115.8
1745 115.8
1745.8 115
1746 114.8
1747 114.8
1748 114.8
1749 114.8
1750 114.8
1751 114.8
1752 114.8
1753 114.8
1754 114.8
1755 114.8
1755.8 114
1756 113.8
1757 113.8
1758 113.8
1759 113.8
1760 113.8
1761 113.8
1762 113.8
1763 113.8
1764 113.8
1765 113.8
1766 113.8
1767 113.8
1768 113.8
1769 113.8
1770 113.8
1771 113.8
1772 113.8
1773 113.8
1774 113.8
1775 113.8
1776 113.8
1777 113.8
1778 113.8
1779 113.8
1780 113.8
1781 113.8
1782 113.8
1782.2 114
1783 114.8
1784 114.8
1785 114.8
1786 114.8
1787 114.8
1788 114.8
1788.2 115
1789 115.8
1790 115.8
1791 115.8
1792 115.8
1793 115.8
1793.2 116
1794 116.8
1795 116.8
1796 116.8
1797 116.8
1797.2 117
1798 117.8
1799 117.8
1800 117.8
1800.2 118
1801 118.8
1802 118.8
1803 118.8
1803.2 119
1804 119.8
1805 119.8
1805.2 120
1806 120.8
1807 120.8
1807.2 121
1808 121.8
1809 121.8
1809.2 122
1810 122.8
1810.2 123
1811 123.8
1811.2 124
1812 124.8
1812.2 125
1813 125.8
1813.2 126
1814 126.8
1814.2 127
1815 127.8
1815.2 128
1815.2 129
1815.2 130
1815.2 131
1815.2 132
1815.2 133
1816 133.8
1816.2 134
1816 134.2
1815.2 135
1815 135.2
1814.2 136
1814 136.2
1813.2 137
1813 137.2
1812.2 138
1812 138.2
1811.2 139
1811 139.2
1810 139.2
1809 139.2
1808.2 140
1808 140.2
1807 140.2
1806 140.2
1805 140.2
1804.2 141
1804 141.2
1803 141.2
1802 141.2
1801 141.2
1800 141.2
1799 141.2
1798 141.2
1797 141.2
1796 141.2
1795 141.2
1794.2 142
1794 142.2
};
\addplot [red]
table {%
4226 334.2
4225 334.2
4224 334.2
4223.8 334
4223 333.2
4222 333.2
4221.8 333
4221 332.2
4220.8 332
4220 331.2
4219.8 331
4219 330.2
4218.8 330
4218.8 329
4218 328.2
4217.8 328
4217.8 327
4217.8 326
4217.8 325
4217.8 324
4217.8 323
4217.8 322
4217.8 321
4218 320.8
4218.8 320
4218.8 319
4218.8 318
4219 317.8
4219.8 317
4219.8 316
4220 315.8
4220.8 315
4220.8 314
4221 313.8
4221.8 313
4221.8 312
4221.8 311
4222 310.8
4222.8 310
4223 309.8
4223.8 309
4223.8 308
4224 307.8
4224.8 307
4225 306.8
4225.8 306
4226 305.8
4226.8 305
4227 304.8
4228 304.8
4228.8 304
4229 303.8
4230 303.8
4230.8 303
4231 302.8
4232 302.8
4233 302.8
4234 302.8
4234.8 302
4235 301.8
4236 301.8
4237 301.8
4238 301.8
4238.2 302
4239 302.8
4240 302.8
4241 302.8
4241.2 303
4242 303.8
4243 303.8
4243.2 304
4244 304.8
4244.2 305
4245 305.8
4245.2 306
4246 306.8
4246.2 307
4246.2 308
4246.2 309
4247 309.8
4247.2 310
4247.2 311
4247.2 312
4247.2 313
4247.2 314
4247.2 315
4247.2 316
4247 316.2
4246.2 317
4246.2 318
4246 318.2
4245.2 319
4245.2 320
4245 320.2
4244.2 321
4244.2 322
4244 322.2
4243.2 323
4243 323.2
4242.2 324
4242 324.2
4241.2 325
4241 325.2
4240.2 326
4240 326.2
4239 326.2
4238.2 327
4238 327.2
4237.2 328
4237 328.2
4236 328.2
4235.2 329
4235 329.2
4234 329.2
4233.2 330
4233 330.2
4232.2 331
4232 331.2
4231.2 332
4231 332.2
4230 332.2
4229.2 333
4229 333.2
4228 333.2
4227 333.2
4226.2 334
4226 334.2
};
\addplot [red]
table {%
4253 377.2
4252 377.2
4251 377.2
4250 377.2
4249.8 377
4249 376.2
4248 376.2
4247 376.2
4246 376.2
4245 376.2
4244.8 376
4244 375.2
4243 375.2
4242.8 375
4242 374.2
4241 374.2
4240.8 374
4240 373.2
4239.8 373
4239 372.2
4238.8 372
4238.8 371
4238 370.2
4237.8 370
4237.8 369
4237 368.2
4236.8 368
4236.8 367
4236.8 366
4236.8 365
4236 364.2
4235.8 364
4235.8 363
4235 362.2
4234.8 362
4234.8 361
4234 360.2
4233.8 360
4233.8 359
4233 358.2
4232.8 358
4232 357.2
4231.8 357
4231.8 356
4231.8 355
4231 354.2
4230.8 354
4230.8 353
4230.8 352
4230 351.2
4229.8 351
4229.8 350
4229.8 349
4229.8 348
4229.8 347
4229.8 346
4229.8 345
4229.8 344
4230 343.8
4230.8 343
4231 342.8
4231.8 342
4232 341.8
4232.8 341
4233 340.8
4234 340.8
4234.8 340
4235 339.8
4236 339.8
4237 339.8
4237.8 339
4238 338.8
4239 338.8
4240 338.8
4241 338.8
4241.8 338
4242 337.8
4243 337.8
4244 337.8
4245 337.8
4245.8 337
4246 336.8
4247 336.8
4248 336.8
4248.8 336
4249 335.8
4250 335.8
4251 335.8
4252 335.8
4252.8 335
4253 334.8
4254 334.8
4255 334.8
4256 334.8
4257 334.8
4257.2 335
4258 335.8
4259 335.8
4260 335.8
4260.2 336
4261 336.8
4261.2 337
4262 337.8
4262.2 338
4263 338.8
4263.2 339
4264 339.8
4264.2 340
4265 340.8
4265.2 341
4265.2 342
4266 342.8
4266.2 343
4267 343.8
4267.2 344
4267.2 345
4268 345.8
4268.2 346
4269 346.8
4269.2 347
4269.2 348
4270 348.8
4270.2 349
4271 349.8
4271.2 350
4271.2 351
4271.2 352
4271.2 353
4272 353.8
4272.2 354
4272.2 355
4272.2 356
4272.2 357
4272.2 358
4272 358.2
4271.2 359
4271.2 360
4271.2 361
4271 361.2
4270.2 362
4270.2 363
4270.2 364
4270 364.2
4269.2 365
4269.2 366
4269 366.2
4268.2 367
4268 367.2
4267.2 368
4267.2 369
4267 369.2
4266.2 370
4266 370.2
4265.2 371
4265 371.2
4264.2 372
4264 372.2
4263.2 373
4263 373.2
4262 373.2
4261.2 374
4261 374.2
4260 374.2
4259.2 375
4259 375.2
4258 375.2
4257.2 376
4257 376.2
4256 376.2
4255 376.2
4254 376.2
4253.2 377
4253 377.2
};
\addplot [red]
table {%
6123 405.2
6122 405.2
6121 405.2
6120 405.2
6119 405.2
6118 405.2
6117 405.2
6116 405.2
6115 405.2
6114 405.2
6113.8 405
6113 404.2
6112 404.2
6111 404.2
6110.8 404
6110 403.2
6109 403.2
6108 403.2
6107.8 403
6107 402.2
6106 402.2
6105.8 402
6105 401.2
6104 401.2
6103.8 401
6103 400.2
6102 400.2
6101.8 400
6101 399.2
6100 399.2
6099.8 399
6099 398.2
6098 398.2
6097.8 398
6097 397.2
6096 397.2
6095.8 397
6095 396.2
6094 396.2
6093.8 396
6093 395.2
6092 395.2
6091.8 395
6091 394.2
6090.8 394
6090 393.2
6089.8 393
6089.8 392
6089 391.2
6088.8 391
6088.8 390
6088.8 389
6088.8 388
6088.8 387
6088.8 386
6088.8 385
6088.8 384
6088.8 383
6089 382.8
6089.8 382
6089.8 381
6090 380.8
6090.8 380
6090.8 379
6091 378.8
6091.8 378
6091.8 377
6092 376.8
6092.8 376
6092.8 375
6093 374.8
6093.8 374
6093.8 373
6093.8 372
6094 371.8
6094.8 371
6094.8 370
6095 369.8
6095.8 369
6095.8 368
6096 367.8
6096.8 367
6096.8 366
6097 365.8
6097.8 365
6097.8 364
6098 363.8
6098.8 363
6098.8 362
6099 361.8
6099.8 361
6100 360.8
6100.8 360
6100.8 359
6101 358.8
6101.8 358
6102 357.8
6102.8 357
6103 356.8
6103.8 356
6104 355.8
6104.8 355
6105 354.8
6105.8 354
6106 353.8
6107 353.8
6107.8 353
6108 352.8
6109 352.8
6109.8 352
6110 351.8
6111 351.8
6112 351.8
6113 351.8
6113.8 351
6114 350.8
6115 350.8
6116 350.8
6117 350.8
6118 350.8
6119 350.8
6120 350.8
6120.2 351
6121 351.8
6122 351.8
6122.2 352
6123 352.8
6124 352.8
6124.2 353
6125 353.8
6126 353.8
6126.2 354
6127 354.8
6127.2 355
6128 355.8
6128.2 356
6129 356.8
6129.2 357
6130 357.8
6130.2 358
6131 358.8
6131.2 359
6132 359.8
6132.2 360
6133 360.8
6133.2 361
6133.2 362
6134 362.8
6134.2 363
6135 363.8
6135.2 364
6135.2 365
6136 365.8
6137 365.8
6137.8 365
6138 364.8
6139 364.8
6140 364.8
6141 364.8
6141.8 364
6142 363.8
6143 363.8
6144 363.8
6145 363.8
6146 363.8
6146.8 363
6147 362.8
6148 362.8
6149 362.8
6150 362.8
6151 362.8
6152 362.8
6153 362.8
6154 362.8
6154.8 362
6155 361.8
6156 361.8
6157 361.8
6158 361.8
6159 361.8
6160 361.8
6161 361.8
6162 361.8
6162.8 361
6163 360.8
6164 360.8
6165 360.8
6166 360.8
6167 360.8
6168 360.8
6169 360.8
6169.2 361
6170 361.8
6171 361.8
6172 361.8
6172.2 362
6173 362.8
6174 362.8
6174.2 363
6175 363.8
6175.2 364
6176 364.8
6176.2 365
6177 365.8
6177.2 366
6177.2 367
6178 367.8
6178.2 368
6178.2 369
6178.2 370
6179 370.8
6179.2 371
6179.2 372
6179.2 373
6179.2 374
6179.2 375
6179.2 376
6179.2 377
6179.2 378
6179.2 379
6179.2 380
6179.2 381
6179 381.2
6178.2 382
6178.2 383
6178.2 384
6178.2 385
6178.2 386
6178 386.2
6177.2 387
6177.2 388
6177.2 389
6177 389.2
6176.2 390
6176.2 391
6176 391.2
6175.2 392
6175 392.2
6174.2 393
6174 393.2
6173.2 394
6173.2 395
6173 395.2
6172.2 396
6172 396.2
6171.2 397
6171 397.2
6170 397.2
6169.2 398
6169 398.2
6168.2 399
6168 399.2
6167.2 400
6167 400.2
6166 400.2
6165.2 401
6165 401.2
6164 401.2
6163 401.2
6162.2 402
6162 402.2
6161 402.2
6160 402.2
6159 402.2
6158 402.2
6157 402.2
6156 402.2
6155 402.2
6154 402.2
6153 402.2
6152 402.2
6151 402.2
6150 402.2
6149 402.2
6148 402.2
6147 402.2
6146 402.2
6145 402.2
6144.8 402
6144 401.2
6143 401.2
6142 401.2
6141.8 401
6141 400.2
6140 400.2
6139.8 400
6139 399.2
6138 399.2
6137.8 399
6137 398.2
6136 398.2
6135.8 398
6135 397.2
6134.2 398
6134 398.2
6133.2 399
6133 399.2
6132.2 400
6132 400.2
6131.2 401
6131 401.2
6130.2 402
6130 402.2
6129 402.2
6128.2 403
6128 403.2
6127 403.2
6126.2 404
6126 404.2
6125 404.2
6124 404.2
6123.2 405
6123 405.2
};
\addplot [red]
table {%
6204 399.2
6203 399.2
6202 399.2
6201 399.2
6200.8 399
6200 398.2
6199 398.2
6198.8 398
6198 397.2
6197 397.2
6196.8 397
6196 396.2
6195.8 396
6195 395.2
6194.8 395
6194.8 394
6194 393.2
6193.8 393
6193.8 392
6193.8 391
6193 390.2
6192.8 390
6192.8 389
6192.8 388
6192.8 387
6192 386.2
6191.8 386
6191.8 385
6191.8 384
6191 383.2
6190.8 383
6190.8 382
6190.8 381
6190 380.2
6189.8 380
6189.8 379
6189 378.2
6188.8 378
6188.8 377
6188.8 376
6188.8 375
6188 374.2
6187.8 374
6187.8 373
6187.8 372
6187.8 371
6187.8 370
6188 369.8
6188.8 369
6188.8 368
6188.8 367
6189 366.8
6189.8 366
6190 365.8
6190.8 365
6191 364.8
6191.8 364
6192 363.8
6192.8 363
6193 362.8
6194 362.8
6195 362.8
6196 362.8
6197 362.8
6197.2 363
6198 363.8
6199 363.8
6199.2 364
6200 364.8
6201 364.8
6201.2 365
6202 365.8
6202.2 366
6203 366.8
6203.2 367
6204 367.8
6204.2 368
6205 368.8
6205.2 369
6206 369.8
6206.2 370
6207 370.8
6207.2 371
6207.2 372
6208 372.8
6208.2 373
6209 373.8
6209.2 374
6209.2 375
6209.2 376
6210 376.8
6210.2 377
6210.2 378
6210.2 379
6211 379.8
6211.2 380
6211.2 381
6211.2 382
6211.2 383
6211.2 384
6211.2 385
6211.2 386
6211.2 387
6212 387.8
6212.2 388
6212.2 389
6212 389.2
6211.2 390
6211.2 391
6211.2 392
6211.2 393
6211 393.2
6210.2 394
6210.2 395
6210 395.2
6209.2 396
6209 396.2
6208.2 397
6208 397.2
6207.2 398
6207 398.2
6206 398.2
6205 398.2
6204.2 399
6204 399.2
};
\addplot [red]
table {%
381 425.2
380 425.2
379 425.2
378 425.2
377 425.2
376 425.2
375 425.2
374 425.2
373 425.2
372 425.2
371 425.2
370 425.2
369 425.2
368.8 425
368 424.2
367 424.2
366 424.2
365 424.2
364 424.2
363.8 424
363 423.2
362 423.2
361.8 423
361 422.2
360 422.2
359.8 422
359 421.2
358 421.2
357.8 421
357 420.2
356 420.2
355.8 420
355 419.2
354.8 419
354 418.2
353.8 418
353 417.2
352.8 417
352 416.2
351 416.2
350.8 416
350 415.2
349.8 415
349.8 414
349.8 413
349 412.2
348.8 412
348.8 411
348.8 410
348 409.2
347.8 409
347.8 408
347.8 407
347.8 406
347.8 405
347.8 404
348 403.8
348.8 403
348.8 402
348.8 401
349 400.8
349.8 400
349.8 399
349.8 398
350 397.8
350.8 397
350.8 396
351 395.8
351.8 395
352 394.8
352.8 394
352.8 393
353 392.8
353.8 392
353.8 391
354 390.8
354.8 390
355 389.8
355.8 389
356 388.8
356.8 388
357 387.8
357.8 387
358 386.8
358.8 386
358.8 385
359 384.8
360 384.8
360.8 384
361 383.8
361.8 383
362 382.8
362.8 382
363 381.8
363.8 381
364 380.8
365 380.8
365.8 380
366 379.8
367 379.8
368 379.8
368.8 379
369 378.8
370 378.8
371 378.8
372 378.8
373 378.8
374 378.8
375 378.8
375.8 378
376 377.8
377 377.8
378 377.8
379 377.8
380 377.8
381 377.8
381.2 378
382 378.8
383 378.8
384 378.8
385 378.8
386 378.8
387 378.8
388 378.8
389 378.8
390 378.8
390.2 379
391 379.8
392 379.8
393 379.8
394 379.8
395 379.8
396 379.8
397 379.8
398 379.8
399 379.8
400 379.8
401 379.8
401.2 380
402 380.8
403 380.8
404 380.8
404.2 381
405 381.8
406 381.8
407 381.8
407.2 382
408 382.8
409 382.8
410 382.8
411 382.8
411.2 383
412 383.8
413 383.8
414 383.8
414.2 384
415 384.8
416 384.8
417 384.8
417.2 385
418 385.8
419 385.8
420 385.8
420.2 386
421 386.8
422 386.8
422.2 387
423 387.8
423.2 388
424 388.8
425 388.8
425.2 389
426 389.8
427 389.8
427.2 390
428 390.8
428.2 391
428 391.2
427.2 392
427.2 393
427.2 394
427.2 395
427.2 396
427.2 397
427.2 398
427.2 399
427 399.2
426.2 400
426.2 401
426 401.2
425.2 402
425.2 403
425 403.2
424.2 404
424.2 405
424.2 406
424 406.2
423.2 407
423.2 408
423 408.2
422.2 409
422.2 410
422 410.2
421.2 411
421 411.2
420.2 412
420 412.2
419 412.2
418.2 413
418 413.2
417.2 414
417 414.2
416 414.2
415.2 415
415 415.2
414 415.2
413 415.2
412.2 416
412 416.2
411 416.2
410 416.2
409.2 417
409 417.2
408 417.2
407 417.2
406 417.2
405 417.2
404 417.2
403 417.2
402.2 418
402 418.2
401 418.2
400 418.2
399 418.2
398.2 419
398 419.2
397 419.2
396 419.2
395 419.2
394.2 420
394 420.2
393 420.2
392.2 421
392 421.2
391 421.2
390 421.2
389.2 422
389 422.2
388 422.2
387.2 423
387 423.2
386 423.2
385 423.2
384.2 424
384 424.2
383 424.2
382 424.2
381.2 425
381 425.2
};
\addplot [red]
table {%
360 465.2
359 465.2
358 465.2
357 465.2
356 465.2
355 465.2
354.8 465
354 464.2
353 464.2
352 464.2
351 464.2
350 464.2
349 464.2
348 464.2
347 464.2
346 464.2
345 464.2
344 464.2
343 464.2
342 464.2
341 464.2
340.8 464
340 463.2
339 463.2
338 463.2
337 463.2
336 463.2
335 463.2
334 463.2
333 463.2
332.8 463
332 462.2
331 462.2
330 462.2
329 462.2
328.8 462
328 461.2
327 461.2
326 461.2
325 461.2
324.8 461
324 460.2
323 460.2
322 460.2
321 460.2
320 460.2
319.8 460
319 459.2
318 459.2
317 459.2
316 459.2
315 459.2
314 459.2
313 459.2
312 459.2
311 459.2
310 459.2
309 459.2
308.8 459
308 458.2
307 458.2
306 458.2
305 458.2
304 458.2
303 458.2
302.8 458
302 457.2
301 457.2
300 457.2
299 457.2
298 457.2
297.8 457
297 456.2
296 456.2
295 456.2
294 456.2
293 456.2
292.8 456
292 455.2
291 455.2
290 455.2
289 455.2
288 455.2
287.8 455
287 454.2
286 454.2
285 454.2
284 454.2
283 454.2
282.8 454
282 453.2
281 453.2
280 453.2
279 453.2
278 453.2
277.8 453
277 452.2
276 452.2
275 452.2
274 452.2
273 452.2
272.8 452
272 451.2
271 451.2
270 451.2
269 451.2
268 451.2
267 451.2
266.8 451
266 450.2
265 450.2
264 450.2
263 450.2
262 450.2
261.8 450
261 449.2
260 449.2
259 449.2
258.8 449
258 448.2
257 448.2
256.8 448
256 447.2
255 447.2
254.8 447
254 446.2
253 446.2
252.8 446
252 445.2
251 445.2
250 445.2
249 445.2
248 445.2
247 445.2
246 445.2
245 445.2
244 445.2
243 445.2
242 445.2
241 445.2
240.8 445
240 444.2
239 444.2
238 444.2
237 444.2
236 444.2
235 444.2
234.8 444
234 443.2
233 443.2
232 443.2
231 443.2
230 443.2
229 443.2
228 443.2
227.8 443
227 442.2
226 442.2
225 442.2
224 442.2
223 442.2
222 442.2
221.8 442
221 441.2
220 441.2
219 441.2
218 441.2
217 441.2
216 441.2
215.8 441
215 440.2
214 440.2
213 440.2
212 440.2
211.8 440
211 439.2
210 439.2
209 439.2
208 439.2
207.8 439
207 438.2
206 438.2
205 438.2
204.8 438
204 437.2
203 437.2
202 437.2
201.8 437
201 436.2
200 436.2
199 436.2
198.8 436
198 435.2
197 435.2
196 435.2
195 435.2
194.8 435
194 434.2
193 434.2
192 434.2
191 434.2
190 434.2
189 434.2
188.8 434
188 433.2
187 433.2
186 433.2
185 433.2
184 433.2
183.8 433
183 432.2
182 432.2
181 432.2
180 432.2
179.8 432
179 431.2
178 431.2
177 431.2
176 431.2
175.8 431
175 430.2
174 430.2
173 430.2
172.8 430
172 429.2
171 429.2
170 429.2
169.8 429
169 428.2
168 428.2
167 428.2
166.8 428
166 427.2
165 427.2
164 427.2
163.8 427
163 426.2
162 426.2
161 426.2
160 426.2
159.8 426
159 425.2
158 425.2
157 425.2
156.8 425
156 424.2
155 424.2
154 424.2
153 424.2
152.8 424
152 423.2
151 423.2
150 423.2
149 423.2
148.8 423
148 422.2
147 422.2
146 422.2
145 422.2
144.8 422
144 421.2
143 421.2
142.8 421
142 420.2
141 420.2
140.8 420
140 419.2
139 419.2
138 419.2
137.8 419
137 418.2
136 418.2
135.8 418
135 417.2
134 417.2
133.8 417
133 416.2
132 416.2
131 416.2
130.8 416
130 415.2
129 415.2
128.8 415
128 414.2
127 414.2
126.8 414
126 413.2
125 413.2
124 413.2
123.8 413
123 412.2
122 412.2
121.8 412
121.8 411
121 410.2
120.8 410
120.8 409
120.8 408
120 407.2
119.8 407
119.8 406
119.8 405
119 404.2
118.8 404
118.8 403
118.8 402
118 401.2
117.8 401
117.8 400
117.8 399
117 398.2
116.8 398
116.8 397
117 396.8
117.8 396
118 395.8
119 395.8
119.8 395
120 394.8
120.8 394
121 393.8
122 393.8
122.8 393
123 392.8
124 392.8
124.8 392
125 391.8
126 391.8
126.8 391
127 390.8
127.8 390
128 389.8
129 389.8
129.8 389
130 388.8
131 388.8
131.8 388
132 387.8
133 387.8
133.8 387
134 386.8
135 386.8
135.2 387
136 387.8
137 387.8
138 387.8
139 387.8
140 387.8
141 387.8
142 387.8
143 387.8
144 387.8
145 387.8
146 387.8
147 387.8
148 387.8
149 387.8
150 387.8
151 387.8
152 387.8
153 387.8
154 387.8
155 387.8
156 387.8
157 387.8
158 387.8
158.2 388
159 388.8
160 388.8
161 388.8
162 388.8
163 388.8
163.2 389
164 389.8
165 389.8
166 389.8
167 389.8
168 389.8
169 389.8
169.2 390
170 390.8
171 390.8
172 390.8
173 390.8
174 390.8
175 390.8
176 390.8
177 390.8
178 390.8
179 390.8
180 390.8
181 390.8
182 390.8
183 390.8
184 390.8
185 390.8
186 390.8
186.2 391
187 391.8
188 391.8
189 391.8
190 391.8
191 391.8
191.2 392
192 392.8
193 392.8
194 392.8
195 392.8
196 392.8
196.2 393
197 393.8
198 393.8
199 393.8
200 393.8
201 393.8
202 393.8
203 393.8
204 393.8
205 393.8
205.2 394
206 394.8
207 394.8
208 394.8
209 394.8
209.2 395
210 395.8
211 395.8
212 395.8
213 395.8
213.2 396
214 396.8
215 396.8
216 396.8
217 396.8
217.2 397
218 397.8
219 397.8
220 397.8
221 397.8
221.2 398
222 398.8
223 398.8
224 398.8
225 398.8
225.2 399
226 399.8
227 399.8
228 399.8
229 399.8
230 399.8
231 399.8
231.2 400
232 400.8
233 400.8
234 400.8
235 400.8
236 400.8
237 400.8
237.2 401
238 401.8
239 401.8
240 401.8
241 401.8
242 401.8
243 401.8
244 401.8
244.2 402
245 402.8
246 402.8
247 402.8
247.2 403
248 403.8
249 403.8
250 403.8
251 403.8
251.2 404
252 404.8
253 404.8
254 404.8
255 404.8
255.8 404
256 403.8
257 403.8
258 403.8
259 403.8
260 403.8
260.8 403
261 402.8
262 402.8
263 402.8
264 402.8
265 402.8
266 402.8
267 402.8
268 402.8
269 402.8
269.2 403
270 403.8
271 403.8
272 403.8
273 403.8
274 403.8
275 403.8
275.2 404
276 404.8
277 404.8
278 404.8
279 404.8
280 404.8
281 404.8
281.2 405
282 405.8
283 405.8
284 405.8
285 405.8
286 405.8
286.2 406
287 406.8
288 406.8
289 406.8
289.2 407
290 407.8
291 407.8
292 407.8
292.2 408
293 408.8
294 408.8
295 408.8
295.2 409
296 409.8
297 409.8
298 409.8
298.2 410
299 410.8
300 410.8
301 410.8
301.2 411
302 411.8
303 411.8
304 411.8
304.2 412
305 412.8
306 412.8
307 412.8
308 412.8
308.2 413
309 413.8
310 413.8
311 413.8
312 413.8
312.2 414
313 414.8
314 414.8
315 414.8
316 414.8
317 414.8
318 414.8
318.2 415
319 415.8
320 415.8
320.2 416
321 416.8
322 416.8
322.2 417
323 417.8
324 417.8
325 417.8
326 417.8
327 417.8
328 417.8
328.2 418
329 418.8
330 418.8
331 418.8
331.2 419
332 419.8
333 419.8
333.2 420
334 420.8
335 420.8
336 420.8
336.2 421
337 421.8
338 421.8
338.2 422
339 422.8
340 422.8
340.2 423
341 423.8
342 423.8
342.2 424
343 424.8
344 424.8
344.2 425
345 425.8
346 425.8
346.2 426
347 426.8
348 426.8
349 426.8
349.2 427
350 427.8
351 427.8
352 427.8
352.2 428
353 428.8
354 428.8
354.2 429
355 429.8
356 429.8
357 429.8
357.2 430
358 430.8
359 430.8
359.2 431
360 431.8
361 431.8
362 431.8
362.2 432
363 432.8
364 432.8
364.2 433
365 433.8
366 433.8
367 433.8
367.2 434
368 434.8
369 434.8
369.2 435
370 435.8
371 435.8
372 435.8
372.2 436
373 436.8
374 436.8
374.2 437
375 437.8
376 437.8
376.2 438
377 438.8
377.2 439
378 439.8
378.2 440
378.2 441
379 441.8
379.2 442
380 442.8
380.2 443
380.2 444
381 444.8
381.2 445
381.2 446
382 446.8
382.2 447
383 447.8
383.2 448
383.2 449
384 449.8
384.2 450
384.2 451
384.2 452
384.2 453
384 453.2
383.2 454
383 454.2
382.2 455
382 455.2
381.2 456
381.2 457
381 457.2
380.2 458
380 458.2
379.2 459
379 459.2
378.2 460
378.2 461
378 461.2
377.2 462
377 462.2
376 462.2
375 462.2
374 462.2
373.2 463
373 463.2
372 463.2
371 463.2
370 463.2
369 463.2
368 463.2
367.2 464
367 464.2
366 464.2
365 464.2
364 464.2
363 464.2
362 464.2
361 464.2
360.2 465
360 465.2
};
\addplot [red]
table {%
548 458.2
547 458.2
546 458.2
545 458.2
544 458.2
543 458.2
542 458.2
541 458.2
540 458.2
539 458.2
538 458.2
537 458.2
536 458.2
535 458.2
534 458.2
533 458.2
532 458.2
531 458.2
530 458.2
529 458.2
528 458.2
527 458.2
526.8 458
526 457.2
525 457.2
524 457.2
523 457.2
522 457.2
521.8 457
521 456.2
520 456.2
519 456.2
518 456.2
517.8 456
517 455.2
516 455.2
515 455.2
514 455.2
513 455.2
512 455.2
511 455.2
510.8 455
510 454.2
509 454.2
508 454.2
507 454.2
506.8 454
506 453.2
505 453.2
504 453.2
503 453.2
502 453.2
501.8 453
501 452.2
500 452.2
499 452.2
498 452.2
497 452.2
496 452.2
495 452.2
494 452.2
493 452.2
492 452.2
491 452.2
490.8 452
490 451.2
489 451.2
488 451.2
487 451.2
486.8 451
486 450.2
485 450.2
484 450.2
483 450.2
482.8 450
482 449.2
481 449.2
480 449.2
479.8 449
479 448.2
478 448.2
477.8 448
477 447.2
476 447.2
475.8 447
475 446.2
474.8 446
474 445.2
473 445.2
472.8 445
472 444.2
471.8 444
471 443.2
470 443.2
469.8 443
469 442.2
468 442.2
467.8 442
467 441.2
466 441.2
465.8 441
465 440.2
464 440.2
463.8 440
463 439.2
462 439.2
461.8 439
461 438.2
460 438.2
459.8 438
459 437.2
458 437.2
457.8 437
457 436.2
456 436.2
455.8 436
455 435.2
454 435.2
453.8 435
453 434.2
452 434.2
451.8 434
451 433.2
450 433.2
449.8 433
449 432.2
448 432.2
447.8 432
447 431.2
446.8 431
446 430.2
445.8 430
445.8 429
445 428.2
444.8 428
444 427.2
443.8 427
443 426.2
442.8 426
442 425.2
441.8 425
441.8 424
441 423.2
440.8 423
440 422.2
439.8 422
439 421.2
438.8 421
438.8 420
438 419.2
437.8 419
437 418.2
436.8 418
436 417.2
435.8 417
435 416.2
434.8 416
434.8 415
434 414.2
433.8 414
434 413.8
434.8 413
434.8 412
435 411.8
435.8 411
435.8 410
435.8 409
436 408.8
436.8 408
436.8 407
437 406.8
437.8 406
437.8 405
437.8 404
438 403.8
438.8 403
438.8 402
439 401.8
439.8 401
439.8 400
440 399.8
440.8 399
441 398.8
441.8 398
442 397.8
443 397.8
444 397.8
445 397.8
445.8 397
446 396.8
447 396.8
448 396.8
449 396.8
449.8 396
450 395.8
451 395.8
452 395.8
453 395.8
454 395.8
454.8 395
455 394.8
456 394.8
457 394.8
458 394.8
458.8 394
459 393.8
460 393.8
461 393.8
462 393.8
462.8 393
463 392.8
464 392.8
465 392.8
466 392.8
466.2 393
467 393.8
468 393.8
469 393.8
470 393.8
471 393.8
472 393.8
473 393.8
474 393.8
475 393.8
476 393.8
477 393.8
478 393.8
479 393.8
480 393.8
481 393.8
482 393.8
483 393.8
484 393.8
485 393.8
486 393.8
487 393.8
487.2 394
488 394.8
489 394.8
490 394.8
491 394.8
492 394.8
493 394.8
494 394.8
495 394.8
496 394.8
497 394.8
498 394.8
498.8 394
499 393.8
500 393.8
501 393.8
502 393.8
503 393.8
504 393.8
505 393.8
506 393.8
507 393.8
508 393.8
509 393.8
510 393.8
511 393.8
511.2 394
512 394.8
513 394.8
514 394.8
515 394.8
516 394.8
517 394.8
517.2 395
518 395.8
519 395.8
520 395.8
521 395.8
522 395.8
522.2 396
523 396.8
524 396.8
525 396.8
526 396.8
526.2 397
527 397.8
528 397.8
529 397.8
530 397.8
531 397.8
531.2 398
532 398.8
533 398.8
534 398.8
535 398.8
536 398.8
537 398.8
537.2 399
538 399.8
539 399.8
540 399.8
541 399.8
541.2 400
542 400.8
543 400.8
543.2 401
544 401.8
545 401.8
545.2 402
546 402.8
547 402.8
547.2 403
548 403.8
549 403.8
549.2 404
550 404.8
551 404.8
552 404.8
552.2 405
553 405.8
554 405.8
555 405.8
556 405.8
556.2 406
557 406.8
558 406.8
559 406.8
559.2 407
560 407.8
561 407.8
562 407.8
562.2 408
563 408.8
564 408.8
564.2 409
565 409.8
566 409.8
567 409.8
567.2 410
568 410.8
569 410.8
570 410.8
570.2 411
571 411.8
572 411.8
572.2 412
573 412.8
573.2 413
574 413.8
574.2 414
574.2 415
575 415.8
575.2 416
576 416.8
576.2 417
577 417.8
577.2 418
578 418.8
578.2 419
579 419.8
579.2 420
580 420.8
580.2 421
580 421.2
579.2 422
579.2 423
579.2 424
579.2 425
579.2 426
579.2 427
579.2 428
579.2 429
579 429.2
578.2 430
578.2 431
578.2 432
578.2 433
578 433.2
577.2 434
577.2 435
577 435.2
576.2 436
576.2 437
576 437.2
575.2 438
575.2 439
575.2 440
575 440.2
574.2 441
574.2 442
574 442.2
573.2 443
573 443.2
572.2 444
572 444.2
571.2 445
571 445.2
570.2 446
570.2 447
570 447.2
569.2 448
569 448.2
568.2 449
568 449.2
567.2 450
567 450.2
566.2 451
566 451.2
565 451.2
564.2 452
564 452.2
563 452.2
562.2 453
562 453.2
561 453.2
560.2 454
560 454.2
559.2 455
559 455.2
558 455.2
557 455.2
556.2 456
556 456.2
555 456.2
554 456.2
553 456.2
552.2 457
552 457.2
551 457.2
550 457.2
549 457.2
548.2 458
548 458.2
};
\addplot [red]
table {%
605 455.2
604 455.2
603 455.2
602 455.2
601 455.2
600 455.2
599 455.2
598 455.2
597 455.2
596 455.2
595.8 455
595 454.2
594 454.2
593 454.2
592.8 454
592 453.2
591.8 453
591 452.2
590.8 452
590 451.2
589.8 451
589 450.2
588.8 450
588 449.2
587.8 449
587 448.2
586.8 448
586 447.2
585.8 447
585 446.2
584.8 446
584 445.2
583.8 445
583.8 444
583 443.2
582.8 443
582 442.2
581.8 442
581.8 441
581.8 440
581.8 439
581.8 438
581.8 437
582 436.8
582.8 436
582.8 435
582.8 434
583 433.8
583.8 433
584 432.8
584.8 432
585 431.8
585.8 431
586 430.8
586.8 430
587 429.8
588 429.8
588.8 429
589 428.8
590 428.8
591 428.8
592 428.8
593 428.8
594 428.8
595 428.8
596 428.8
597 428.8
598 428.8
599 428.8
600 428.8
601 428.8
602 428.8
603 428.8
604 428.8
604.2 429
605 429.8
606 429.8
607 429.8
608 429.8
609 429.8
610 429.8
610.2 430
611 430.8
612 430.8
613 430.8
613.2 431
614 431.8
615 431.8
615.2 432
616 432.8
617 432.8
617.2 433
618 433.8
618.2 434
619 434.8
619.2 435
620 435.8
620.2 436
620.2 437
621 437.8
621.2 438
621.2 439
621.2 440
621.2 441
621.2 442
621.2 443
621.2 444
621.2 445
621 445.2
620.2 446
620.2 447
620.2 448
620 448.2
619.2 449
619 449.2
618.2 450
618 450.2
617.2 451
617 451.2
616 451.2
615.2 452
615 452.2
614 452.2
613.2 453
613 453.2
612 453.2
611 453.2
610.2 454
610 454.2
609 454.2
608 454.2
607 454.2
606 454.2
605.2 455
605 455.2
};
\addplot [red]
table {%
3811 585.2
3810 585.2
3809 585.2
3808 585.2
3807.8 585
3807 584.2
3806 584.2
3805 584.2
3804 584.2
3803.8 584
3803 583.2
3802 583.2
3801.8 583
3801 582.2
3800 582.2
3799.8 582
3799 581.2
3798 581.2
3797.8 581
3797 580.2
3796 580.2
3795.8 580
3795 579.2
3794.8 579
3794 578.2
3793.8 578
3793 577.2
3792.8 577
3792 576.2
3791.8 576
3791 575.2
3790.8 575
3790.8 574
3790.8 573
3790 572.2
3789.8 572
3789.8 571
3789.8 570
3789.8 569
3790 568.8
3790.8 568
3790.8 567
3791 566.8
3791.8 566
3791.8 565
3792 564.8
3792.8 564
3792.8 563
3793 562.8
3793.8 562
3794 561.8
3794.8 561
3795 560.8
3795.8 560
3796 559.8
3796.8 559
3797 558.8
3798 558.8
3798.8 558
3799 557.8
3800 557.8
3800.8 557
3801 556.8
3802 556.8
3803 556.8
3803.8 556
3804 555.8
3805 555.8
3806 555.8
3807 555.8
3808 555.8
3809 555.8
3810 555.8
3811 555.8
3812 555.8
3813 555.8
3814 555.8
3815 555.8
3816 555.8
3816.2 556
3817 556.8
3818 556.8
3818.2 557
3819 557.8
3820 557.8
3820.2 558
3821 558.8
3821.2 559
3822 559.8
3822.2 560
3823 560.8
3823.2 561
3824 561.8
3824.2 562
3824.2 563
3824.2 564
3825 564.8
3825.2 565
3825.2 566
3825.2 567
3825.2 568
3825.2 569
3825 569.2
3824.2 570
3824.2 571
3824 571.2
3823.2 572
3823.2 573
3823 573.2
3822.2 574
3822 574.2
3821.2 575
3821 575.2
3820.2 576
3820.2 577
3820 577.2
3819.2 578
3819 578.2
3818.2 579
3818 579.2
3817.2 580
3817.2 581
3817 581.2
3816.2 582
3816 582.2
3815.2 583
3815 583.2
3814 583.2
3813.2 584
3813 584.2
3812 584.2
3811.2 585
3811 585.2
};
\addplot [red]
table {%
397 691.2
396 691.2
395 691.2
394 691.2
393 691.2
392 691.2
391 691.2
390 691.2
389 691.2
388 691.2
387 691.2
386 691.2
385 691.2
384.8 691
384 690.2
383 690.2
382 690.2
381 690.2
380 690.2
379.8 690
379 689.2
378 689.2
377 689.2
376.8 689
376 688.2
375 688.2
374 688.2
373.8 688
373 687.2
372 687.2
371.8 687
371 686.2
370 686.2
369.8 686
369 685.2
368.8 685
368 684.2
367.8 684
367 683.2
366 683.2
365.8 683
365 682.2
364.8 682
364 681.2
363.8 681
363.8 680
363 679.2
362.8 679
362 678.2
361.8 678
361.8 677
361 676.2
360.8 676
360 675.2
359.8 675
359.8 674
359 673.2
358.8 673
358.8 672
358.8 671
358 670.2
357.8 670
357.8 669
357.8 668
357 667.2
356.8 667
356.8 666
356.8 665
356.8 664
356.8 663
356.8 662
356.8 661
356.8 660
356.8 659
356 658.2
355.8 658
356 657.8
356.8 657
356.8 656
356.8 655
356.8 654
357 653.8
357.8 653
357.8 652
357.8 651
357.8 650
358 649.8
358.8 649
359 648.8
359.8 648
359.8 647
360 646.8
360.8 646
360.8 645
361 644.8
361.8 644
361.8 643
362 642.8
362.8 642
363 641.8
363.8 641
364 640.8
364.8 640
365 639.8
365.8 639
366 638.8
366.8 638
367 637.8
367.8 637
368 636.8
369 636.8
369.8 636
370 635.8
371 635.8
371.8 635
372 634.8
372.8 634
373 633.8
374 633.8
374.8 633
375 632.8
376 632.8
376.8 632
377 631.8
378 631.8
378.8 631
379 630.8
380 630.8
381 630.8
381.8 630
382 629.8
383 629.8
384 629.8
384.8 629
385 628.8
386 628.8
387 628.8
388 628.8
389 628.8
390 628.8
391 628.8
392 628.8
393 628.8
393.8 628
394 627.8
395 627.8
396 627.8
397 627.8
398 627.8
399 627.8
400 627.8
401 627.8
401.8 627
402 626.8
403 626.8
404 626.8
405 626.8
406 626.8
406.8 626
407 625.8
408 625.8
408.8 625
409 624.8
410 624.8
411 624.8
411.8 624
412 623.8
413 623.8
414 623.8
414.8 623
415 622.8
416 622.8
416.8 622
417 621.8
418 621.8
418.8 621
419 620.8
420 620.8
421 620.8
421.8 620
422 619.8
423 619.8
423.8 619
424 618.8
425 618.8
426 618.8
426.8 618
427 617.8
428 617.8
429 617.8
430 617.8
430.8 617
431 616.8
432 616.8
433 616.8
433.8 616
434 615.8
435 615.8
436 615.8
436.8 615
437 614.8
438 614.8
439 614.8
440 614.8
440.8 614
441 613.8
442 613.8
443 613.8
444 613.8
445 613.8
446 613.8
447 613.8
447.2 614
448 614.8
449 614.8
450 614.8
451 614.8
452 614.8
453 614.8
453.2 615
454 615.8
455 615.8
456 615.8
457 615.8
458 615.8
458.2 616
459 616.8
460 616.8
460.2 617
461 617.8
461.2 618
462 618.8
462.2 619
462.2 620
463 620.8
463.2 621
464 621.8
464.2 622
464.2 623
465 623.8
465.2 624
465.2 625
466 625.8
466.2 626
467 626.8
467.2 627
467.2 628
468 628.8
468.2 629
468.2 630
468.2 631
468.2 632
468.2 633
468.2 634
468.2 635
469 635.8
469.2 636
469.2 637
469.2 638
469.2 639
469.2 640
469.2 641
469.2 642
469.2 643
469.2 644
469.2 645
469.2 646
469.2 647
469.2 648
469.2 649
469.2 650
469.2 651
469 651.2
468.2 652
468.2 653
468.2 654
468.2 655
468.2 656
468.2 657
468.2 658
468.2 659
469 659.8
469.8 659
470 658.8
471 658.8
471.8 658
472 657.8
473 657.8
473.8 657
474 656.8
475 656.8
476 656.8
477 656.8
478 656.8
479 656.8
480 656.8
481 656.8
481.2 657
482 657.8
483 657.8
484 657.8
485 657.8
485.2 658
486 658.8
487 658.8
487.2 659
488 659.8
489 659.8
490 659.8
490.2 660
491 660.8
491.8 660
491 659.2
490.8 659
490.8 658
490.8 657
490 656.2
489.8 656
489.8 655
489 654.2
488.8 654
488 653.2
487.8 653
487.8 652
487 651.2
486.8 651
486.8 650
486 649.2
485.8 649
485.8 648
485 647.2
484.8 647
484 646.2
483.8 646
483.8 645
483 644.2
482.8 644
482.8 643
482.8 642
482 641.2
481.8 641
481.8 640
481 639.2
480.8 639
480.8 638
480.8 637
480 636.2
479.8 636
479.8 635
479.8 634
479 633.2
478.8 633
478.8 632
478.8 631
478 630.2
477.8 630
478 629.8
478.8 629
478.8 628
479 627.8
479.8 627
479.8 626
480 625.8
480.8 625
481 624.8
481.8 624
481.8 623
482 622.8
482.8 622
482.8 621
483 620.8
483.8 620
483.8 619
484 618.8
484.8 618
485 617.8
485.8 617
486 616.8
487 616.8
488 616.8
488.8 616
489 615.8
490 615.8
491 615.8
491.8 615
492 614.8
493 614.8
494 614.8
494.8 614
495 613.8
496 613.8
496.8 613
497 612.8
498 612.8
499 612.8
499.8 612
500 611.8
501 611.8
502 611.8
503 611.8
504 611.8
505 611.8
505.8 611
506 610.8
507 610.8
508 610.8
509 610.8
510 610.8
511 610.8
511.8 610
512 609.8
513 609.8
514 609.8
515 609.8
516 609.8
517 609.8
518 609.8
518.8 609
519 608.8
520 608.8
521 608.8
522 608.8
523 608.8
524 608.8
525 608.8
526 608.8
526.8 608
527 607.8
528 607.8
529 607.8
530 607.8
531 607.8
532 607.8
533 607.8
534 607.8
535 607.8
536 607.8
536.2 608
537 608.8
538 608.8
539 608.8
540 608.8
541 608.8
541.2 609
542 609.8
543 609.8
544 609.8
544.2 610
545 610.8
546 610.8
547 610.8
547.2 611
548 611.8
549 611.8
549.2 612
550 612.8
551 612.8
552 612.8
552.2 613
553 613.8
554 613.8
554.2 614
555 614.8
556 614.8
557 614.8
557.2 615
558 615.8
559 615.8
559.2 616
560 616.8
561 616.8
561.2 617
562 617.8
563 617.8
563.2 618
564 618.8
565 618.8
565.2 619
566 619.8
566.2 620
567 620.8
567.2 621
568 621.8
569 621.8
569.2 622
569.2 623
570 623.8
570.2 624
571 624.8
571.2 625
572 625.8
572.2 626
572.2 627
573 627.8
573.2 628
574 628.8
574.2 629
574.2 630
575 630.8
575.2 631
575.2 632
576 632.8
576.2 633
576.2 634
576.2 635
577 635.8
577.2 636
577.2 637
577.2 638
577.2 639
578 639.8
578.2 640
578.2 641
578.2 642
578.2 643
578.2 644
578.2 645
578.2 646
578.2 647
578.2 648
578.2 649
578 649.2
577.2 650
577.2 651
577.2 652
577.2 653
577.2 654
577.2 655
577 655.2
576.2 656
576.2 657
576.2 658
576 658.2
575.2 659
575.2 660
575.2 661
575 661.2
574.2 662
574.2 663
574 663.2
573.2 664
573.2 665
573 665.2
572.2 666
572.2 667
572 667.2
571.2 668
571 668.2
570.2 669
570.2 670
570 670.2
569.2 671
569 671.2
568.2 672
568 672.2
567.2 673
567 673.2
566.2 674
566 674.2
565.2 675
565 675.2
564.2 676
564 676.2
563.2 677
563 677.2
562 677.2
561.2 678
561 678.2
560 678.2
559.2 679
559 679.2
558 679.2
557.2 680
557 680.2
556 680.2
555.2 681
555 681.2
554 681.2
553 681.2
552 681.2
551 681.2
550.2 682
550 682.2
549 682.2
548 682.2
547 682.2
546 682.2
545 682.2
544 682.2
543 682.2
542 682.2
541 682.2
540 682.2
539 682.2
538.2 683
538 683.2
537 683.2
536 683.2
535 683.2
534 683.2
533 683.2
532 683.2
531 683.2
530 683.2
529 683.2
528 683.2
527.2 684
527 684.2
526 684.2
525 684.2
524 684.2
523 684.2
522 684.2
521.2 685
521 685.2
520 685.2
519 685.2
518 685.2
517 685.2
516 685.2
515 685.2
514 685.2
513 685.2
512 685.2
511 685.2
510 685.2
509 685.2
508.8 685
508 684.2
507 684.2
506 684.2
505.8 684
505 683.2
504 683.2
503 683.2
502.2 684
502 684.2
501 684.2
500 684.2
499 684.2
498.2 685
498 685.2
497 685.2
496 685.2
495 685.2
494.2 686
494 686.2
493 686.2
492 686.2
491 686.2
490.2 687
490 687.2
489 687.2
488 687.2
487 687.2
486 687.2
485 687.2
484 687.2
483 687.2
482.2 688
482 688.2
481 688.2
480 688.2
479 688.2
478 688.2
477 688.2
476 688.2
475 688.2
474 688.2
473 688.2
472 688.2
471.8 688
471 687.2
470 687.2
469.8 687
469 686.2
468 686.2
467.8 686
467 685.2
466.8 685
466 684.2
465.8 684
465 683.2
464.8 683
464 682.2
463.8 682
463 681.2
462.8 681
462.8 680
462.8 679
462.8 678
462.8 677
462 676.2
461 676.2
460.2 677
460 677.2
459.2 678
459 678.2
458.2 679
458 679.2
457.2 680
457 680.2
456.2 681
456 681.2
455.2 682
455 682.2
454 682.2
453 682.2
452 682.2
451 682.2
450.2 683
450 683.2
449 683.2
448 683.2
447 683.2
446.2 684
446 684.2
445 684.2
444 684.2
443 684.2
442 684.2
441.2 685
441 685.2
440 685.2
439 685.2
438 685.2
437 685.2
436 685.2
435 685.2
434.2 686
434 686.2
433 686.2
432 686.2
431 686.2
430 686.2
429 686.2
428 686.2
427 686.2
426 686.2
425 686.2
424.2 687
424 687.2
423 687.2
422 687.2
421 687.2
420 687.2
419 687.2
418 687.2
417.2 688
417 688.2
416 688.2
415 688.2
414 688.2
413 688.2
412 688.2
411.2 689
411 689.2
410 689.2
409 689.2
408 689.2
407 689.2
406 689.2
405.2 690
405 690.2
404 690.2
403 690.2
402 690.2
401 690.2
400 690.2
399 690.2
398 690.2
397.2 691
397 691.2
};
\addplot [red]
table {%
3257 666.2
3256 666.2
3255.8 666
3255 665.2
3254 665.2
3253 665.2
3252 665.2
3251 665.2
3250 665.2
3249 665.2
3248 665.2
3247 665.2
3246 665.2
3245 665.2
3244 665.2
3243.8 665
3243 664.2
3242 664.2
3241 664.2
3240 664.2
3239 664.2
3238 664.2
3237 664.2
3236 664.2
3235 664.2
3234 664.2
3233 664.2
3232.8 664
3232 663.2
3231 663.2
3230 663.2
3229 663.2
3228 663.2
3227 663.2
3226 663.2
3225.8 663
3225 662.2
3224 662.2
3223 662.2
3222 662.2
3221 662.2
3220 662.2
3219 662.2
3218 662.2
3217 662.2
3216 662.2
3215 662.2
3214 662.2
3213 662.2
3212 662.2
3211 662.2
3210 662.2
3209 662.2
3208 662.2
3207.2 663
3207 663.2
3206 663.2
3205 663.2
3204 663.2
3203 663.2
3202 663.2
3201 663.2
3200 663.2
3199 663.2
3198 663.2
3197 663.2
3196 663.2
3195 663.2
3194 663.2
3193.8 663
3193 662.2
3192 662.2
3191 662.2
3190 662.2
3189 662.2
3188 662.2
3187.8 662
3187 661.2
3186 661.2
3185 661.2
3184.8 661
3184 660.2
3183 660.2
3182.8 660
3182 659.2
3181 659.2
3180.8 659
3180 658.2
3179 658.2
3178 658.2
3177.8 658
3177 657.2
3176 657.2
3175.8 657
3175 656.2
3174.8 656
3174 655.2
3173.8 655
3173 654.2
3172.8 654
3172 653.2
3171.8 653
3171 652.2
3170.8 652
3170 651.2
3169.8 651
3169.8 650
3169.8 649
3169.8 648
3169.8 647
3169.8 646
3169.8 645
3169.8 644
3170 643.8
3170.8 643
3170.8 642
3171 641.8
3171.8 641
3172 640.8
3172.8 640
3172.8 639
3173 638.8
3173.8 638
3174 637.8
3174.8 637
3175 636.8
3175.8 636
3176 635.8
3177 635.8
3177.8 635
3178 634.8
3178.8 634
3179 633.8
3179.8 633
3180 632.8
3181 632.8
3181.8 632
3182 631.8
3182.8 631
3183 630.8
3183.8 630
3184 629.8
3185 629.8
3185.8 629
3186 628.8
3187 628.8
3187.8 628
3188 627.8
3189 627.8
3190 627.8
3190.8 627
3191 626.8
3192 626.8
3193 626.8
3193.8 626
3194 625.8
3195 625.8
3196 625.8
3197 625.8
3198 625.8
3198.8 625
3199 624.8
3200 624.8
3201 624.8
3202 624.8
3203 624.8
3203.8 624
3204 623.8
3205 623.8
3206 623.8
3207 623.8
3208 623.8
3209 623.8
3210 623.8
3210.2 624
3211 624.8
3212 624.8
3213 624.8
3214 624.8
3215 624.8
3216 624.8
3217 624.8
3218 624.8
3219 624.8
3220 624.8
3221 624.8
3222 624.8
3222.2 625
3223 625.8
3224 625.8
3225 625.8
3226 625.8
3227 625.8
3228 625.8
3228.2 626
3229 626.8
3230 626.8
3231 626.8
3232 626.8
3232.2 627
3233 627.8
3234 627.8
3235 627.8
3236 627.8
3236.2 628
3237 628.8
3238 628.8
3239 628.8
3240 628.8
3240.2 629
3241 629.8
3242 629.8
3243 629.8
3244 629.8
3244.2 630
3245 630.8
3246 630.8
3247 630.8
3248 630.8
3248.2 631
3249 631.8
3250 631.8
3251 631.8
3252 631.8
3252.2 632
3253 632.8
3254 632.8
3255 632.8
3255.2 633
3256 633.8
3257 633.8
3258 633.8
3258.2 634
3259 634.8
3260 634.8
3261 634.8
3261.2 635
3262 635.8
3263 635.8
3263.2 636
3264 636.8
3265 636.8
3265.2 637
3266 637.8
3267 637.8
3267.2 638
3268 638.8
3269 638.8
3270 638.8
3270.2 639
3271 639.8
3272 639.8
3272.2 640
3273 640.8
3274 640.8
3274.2 641
3275 641.8
3276 641.8
3277 641.8
3277.2 642
3278 642.8
3279 642.8
3279.2 643
3280 643.8
3281 643.8
3281.2 644
3281.2 645
3281.2 646
3281.2 647
3281.2 648
3281.2 649
3281.2 650
3281.2 651
3281.2 652
3281 652.2
3280.2 653
3280.2 654
3280.2 655
3280.2 656
3280.2 657
3280.2 658
3280.2 659
3280 659.2
3279 659.2
3278 659.2
3277.2 660
3277 660.2
3276 660.2
3275 660.2
3274.2 661
3274 661.2
3273 661.2
3272 661.2
3271.2 662
3271 662.2
3270 662.2
3269 662.2
3268 662.2
3267.2 663
3267 663.2
3266 663.2
3265 663.2
3264.2 664
3264 664.2
3263 664.2
3262 664.2
3261.2 665
3261 665.2
3260 665.2
3259 665.2
3258 665.2
3257.2 666
3257 666.2
};
\addplot [red]
table {%
4224 696.2
4223 696.2
4222 696.2
4221 696.2
4220 696.2
4219 696.2
4218 696.2
4217 696.2
4216 696.2
4215 696.2
4214.8 696
4214 695.2
4213 695.2
4212 695.2
4211.8 695
4211 694.2
4210 694.2
4209.8 694
4209 693.2
4208 693.2
4207.8 693
4207 692.2
4206.8 692
4206 691.2
4205.8 691
4205 690.2
4204.8 690
4204 689.2
4203.8 689
4203 688.2
4202.8 688
4202 687.2
4201.8 687
4201 686.2
4200.8 686
4200 685.2
4199.8 685
4199.8 684
4199.8 683
4199 682.2
4198.8 682
4198.8 681
4198.8 680
4198 679.2
4197.8 679
4197.8 678
4197.8 677
4197.8 676
4197.8 675
4197.8 674
4198 673.8
4198.8 673
4198.8 672
4199 671.8
4199.8 671
4199.8 670
4199.8 669
4200 668.8
4200.8 668
4200.8 667
4201 666.8
4201.8 666
4201.8 665
4202 664.8
4202.8 664
4203 663.8
4203.8 663
4204 662.8
4204.8 662
4204.8 661
4205 660.8
4205.8 660
4206 659.8
4206.8 659
4207 658.8
4207.8 658
4208 657.8
4208.8 657
4209 656.8
4209.8 656
4210 655.8
4211 655.8
4211.8 655
4212 654.8
4213 654.8
4213.8 654
4214 653.8
4215 653.8
4215.8 653
4216 652.8
4217 652.8
4218 652.8
4219 652.8
4219.8 652
4220 651.8
4221 651.8
4222 651.8
4223 651.8
4224 651.8
4225 651.8
4226 651.8
4227 651.8
4228 651.8
4229 651.8
4230 651.8
4231 651.8
4232 651.8
4232.2 652
4233 652.8
4234 652.8
4235 652.8
4236 652.8
4237 652.8
4237.2 653
4238 653.8
4239 653.8
4239.2 654
4240 654.8
4240.2 655
4241 655.8
4242 655.8
4242.2 656
4243 656.8
4243.2 657
4244 657.8
4244.2 658
4244.2 659
4245 659.8
4245.2 660
4245.2 661
4246 661.8
4246.2 662
4246.2 663
4247 663.8
4247.2 664
4247.2 665
4247.2 666
4247.2 667
4247.2 668
4247.2 669
4248 669.8
4248.2 670
4248.2 671
4248.2 672
4248.2 673
4248.2 674
4248 674.2
4247.2 675
4247.2 676
4247.2 677
4247.2 678
4247 678.2
4246.2 679
4246.2 680
4246.2 681
4246 681.2
4245.2 682
4245.2 683
4245 683.2
4244.2 684
4244 684.2
4243.2 685
4243 685.2
4242.2 686
4242 686.2
4241.2 687
4241 687.2
4240.2 688
4240 688.2
4239 688.2
4238.2 689
4238 689.2
4237.2 690
4237 690.2
4236 690.2
4235.2 691
4235 691.2
4234 691.2
4233.2 692
4233 692.2
4232 692.2
4231.2 693
4231 693.2
4230 693.2
4229.2 694
4229 694.2
4228 694.2
4227.2 695
4227 695.2
4226 695.2
4225 695.2
4224.2 696
4224 696.2
};
\addplot [red]
table {%
468.8 661
468 660.2
467.2 661
468 661.8
468.8 661
};
\addplot [red]
table {%
3748 772.2
3747 772.2
3746 772.2
3745 772.2
3744 772.2
3743 772.2
3742 772.2
3741 772.2
3740 772.2
3739.8 772
3739 771.2
3738 771.2
3737 771.2
3736 771.2
3735 771.2
3734 771.2
3733 771.2
3732 771.2
3731 771.2
3730 771.2
3729 771.2
3728 771.2
3727 771.2
3726.8 771
3726 770.2
3725 770.2
3724 770.2
3723 770.2
3722 770.2
3721 770.2
3720 770.2
3719 770.2
3718.8 770
3718 769.2
3717 769.2
3716 769.2
3715 769.2
3714 769.2
3713 769.2
3712 769.2
3711 769.2
3710 769.2
3709 769.2
3708.8 769
3708 768.2
3707 768.2
3706 768.2
3705 768.2
3704 768.2
3703.8 768
3703 767.2
3702 767.2
3701 767.2
3700 767.2
3699 767.2
3698 767.2
3697.8 767
3697 766.2
3696 766.2
3695 766.2
3694 766.2
3693.8 766
3693 765.2
3692.8 765
3692 764.2
3691.8 764
3691 763.2
3690.8 763
3690 762.2
3689.8 762
3689 761.2
3688.8 761
3688 760.2
3687 760.2
3686.8 760
3686 759.2
3685.8 759
3685 758.2
3684.8 758
3684.8 757
3685 756.8
3685.8 756
3685.8 755
3685.8 754
3685.8 753
3686 752.8
3686.8 752
3686.8 751
3686 750.2
3685 750.2
3684 750.2
3683 750.2
3682 750.2
3681 750.2
3680 750.2
3679 750.2
3678 750.2
3677 750.2
3676 750.2
3675 750.2
3674 750.2
3673 750.2
3672 750.2
3671 750.2
3670 750.2
3669.8 750
3669 749.2
3668.8 749
3668 748.2
3667.8 748
3667.8 747
3667.8 746
3667 745.2
3666 745.2
3665.2 746
3665 746.2
3664 746.2
3663.2 747
3663 747.2
3662 747.2
3661 747.2
3660.2 748
3660 748.2
3659 748.2
3658 748.2
3657 748.2
3656.2 749
3656 749.2
3655 749.2
3654 749.2
3653 749.2
3652 749.2
3651 749.2
3650 749.2
3649 749.2
3648 749.2
3647 749.2
3646 749.2
3645 749.2
3644 749.2
3643 749.2
3642 749.2
3641 749.2
3640 749.2
3639 749.2
3638.8 749
3638 748.2
3637 748.2
3636 748.2
3635 748.2
3634 748.2
3633 748.2
3632 748.2
3631 748.2
3630 748.2
3629 748.2
3628 748.2
3627 748.2
3626 748.2
3625.2 749
3625 749.2
3624 749.2
3623.8 749
3623 748.2
3622 748.2
3621 748.2
3620 748.2
3619 748.2
3618 748.2
3617 748.2
3616 748.2
3615 748.2
3614 748.2
3613 748.2
3612.8 748
3612 747.2
3611 747.2
3610 747.2
3609 747.2
3608 747.2
3607.8 747
3607 746.2
3606 746.2
3605.8 746
3605 745.2
3604.8 745
3604 744.2
3603.8 744
3603 743.2
3602.8 743
3602 742.2
3601.8 742
3601 741.2
3600.8 741
3600 740.2
3599 740.2
3598.8 740
3599 739.8
3599.8 739
3600 738.8
3600.8 738
3601 737.8
3601.8 737
3602 736.8
3602.8 736
3603 735.8
3603.8 735
3604 734.8
3604.8 734
3605 733.8
3605.8 733
3606 732.8
3606.8 732
3607 731.8
3608 731.8
3609 731.8
3610 731.8
3611 731.8
3612 731.8
3613 731.8
3613.8 731
3614 730.8
3615 730.8
3616 730.8
3617 730.8
3618 730.8
3619 730.8
3620 730.8
3620.8 730
3621 729.8
3622 729.8
3623 729.8
3623.2 730
3624 730.8
3625 730.8
3626 730.8
3627 730.8
3628 730.8
3629 730.8
3630 730.8
3631 730.8
3632 730.8
3632.2 731
3633 731.8
3634 731.8
3635 731.8
3636 731.8
3637 731.8
3638 731.8
3639 731.8
3640 731.8
3641 731.8
3642 731.8
3643 731.8
3644 731.8
3644.2 732
3645 732.8
3646 732.8
3647 732.8
3648 732.8
3649 732.8
3650 732.8
3651 732.8
3651.2 733
3652 733.8
3653 733.8
3654 733.8
3654.2 734
3655 734.8
3656 734.8
3657 734.8
3658 734.8
3658.2 735
3659 735.8
3660 735.8
3661 735.8
3662 735.8
3662.2 736
3663 736.8
3664 736.8
3664.2 737
3665 737.8
3665.2 738
3666 738.8
3666.2 739
3667 739.8
3667.2 740
3667.2 741
3667.2 742
3668 742.8
3668.8 742
3668.8 741
3669 740.8
3669.8 740
3670 739.8
3670.8 739
3671 738.8
3672 738.8
3673 738.8
3673.8 738
3674 737.8
3675 737.8
3676 737.8
3676.2 738
3677 738.8
3678 738.8
3679 738.8
3680 738.8
3681 738.8
3682 738.8
3682.2 739
3683 739.8
3684 739.8
3684.2 740
3685 740.8
3686 740.8
3686.2 741
3687 741.8
3688 741.8
3688.2 742
3689 742.8
3689.2 743
3690 743.8
3691 743.8
3691.2 744
3691.2 745
3692 745.8
3693 745.8
3693.8 745
3694 744.8
3695 744.8
3695.8 744
3696 743.8
3697 743.8
3697.8 743
3698 742.8
3699 742.8
3700 742.8
3700.8 742
3701 741.8
3702 741.8
3703 741.8
3704 741.8
3705 741.8
3706 741.8
3706.8 741
3707 740.8
3708 740.8
3708.8 740
3708 739.2
3707.8 739
3707 738.2
3706.8 738
3706 737.2
3705.8 737
3705 736.2
3704.8 736
3704.8 735
3704 734.2
3703.8 734
3703 733.2
3702.8 733
3702 732.2
3701.8 732
3701.8 731
3701 730.2
3700.8 730
3700 729.2
3699.8 729
3699.8 728
3699 727.2
3698.8 727
3698 726.2
3697.8 726
3697 725.2
3696.8 725
3696.8 724
3696 723.2
3695.8 723
3695 722.2
3694.8 722
3694.8 721
3694.8 720
3694.8 719
3694.8 718
3694.8 717
3694.8 716
3694.8 715
3694.8 714
3694.8 713
3694.8 712
3694.8 711
3694.8 710
3694 709.2
3693.8 709
3693 708.2
3692 708.2
3691 708.2
3690.8 708
3690 707.2
3689 707.2
3688.8 707
3688 706.2
3687.8 706
3687 705.2
3686 705.2
3685.8 705
3685 704.2
3684.8 704
3684 703.2
3683.8 703
3683 702.2
3682.8 702
3682 701.2
3681 701.2
3680.8 701
3680 700.2
3679.8 700
3679 699.2
3678 699.2
3677 699.2
3676 699.2
3675 699.2
3674 699.2
3673 699.2
3672 699.2
3671 699.2
3670 699.2
3669 699.2
3668 699.2
3667 699.2
3666 699.2
3665 699.2
3664 699.2
3663 699.2
3662 699.2
3661 699.2
3660 699.2
3659.8 699
3659 698.2
3658 698.2
3657 698.2
3656 698.2
3655 698.2
3654 698.2
3653.8 698
3653 697.2
3652 697.2
3651 697.2
3650.8 697
3650 696.2
3649.2 697
3649 697.2
3648 697.2
3647 697.2
3646.2 698
3646 698.2
3645 698.2
3644 698.2
3643.2 699
3643 699.2
3642 699.2
3641.2 700
3641 700.2
3640 700.2
3639.2 701
3639 701.2
3638 701.2
3637 701.2
3636.2 702
3636 702.2
3635.2 703
3635 703.2
3634 703.2
3633.2 704
3633 704.2
3632 704.2
3631.2 705
3631 705.2
3630 705.2
3629.2 706
3629.2 707
3629.2 708
3629 708.2
3628.2 709
3628.2 710
3628.2 711
3628 711.2
3627.2 712
3627.2 713
3627 713.2
3626.2 714
3626 714.2
3625.2 715
3625.2 716
3625 716.2
3624 716.2
3623.2 717
3623 717.2
3622 717.2
3621 717.2
3620.2 718
3620 718.2
3619.8 718
3619 717.2
3618 717.2
3617 717.2
3616 717.2
3615 717.2
3614 717.2
3613 717.2
3612.8 717
3612 716.2
3611 716.2
3610.8 716
3610 715.2
3609.8 715
3609 714.2
3608.2 715
3608 715.2
3607.2 716
3607 716.2
3606.2 717
3606 717.2
3605.2 718
3605 718.2
3604.2 719
3604.2 720
3604 720.2
3603.2 721
3603.2 722
3603 722.2
3602.2 723
3602.2 724
3602 724.2
3601.2 725
3601.2 726
3601 726.2
3600.2 727
3600 727.2
3599.2 728
3599 728.2
3598.2 729
3598 729.2
3597.2 730
3597 730.2
3596.2 731
3596 731.2
3595.2 732
3595 732.2
3594.2 733
3594 733.2
3593.2 734
3593 734.2
3592 734.2
3591.2 735
3591 735.2
3590.2 736
3590 736.2
3589 736.2
3588.2 737
3588 737.2
3587 737.2
3586.2 738
3586 738.2
3585 738.2
3584 738.2
3583 738.2
3582 738.2
3581 738.2
3580 738.2
3579.8 738
3579 737.2
3578.8 737
3578 736.2
3577.8 736
3577.8 735
3577 734.2
3576.8 734
3576 733.2
3575.8 733
3575.8 732
3575.8 731
3576 730.8
3576.8 730
3576.8 729
3576.8 728
3576.8 727
3576.8 726
3576 725.2
3575 725.2
3574 725.2
3573 725.2
3572 725.2
3571.2 726
3571 726.2
3570 726.2
3569 726.2
3568 726.2
3567 726.2
3566 726.2
3565 726.2
3564 726.2
3563 726.2
3562 726.2
3561 726.2
3560.8 726
3560 725.2
3559 725.2
3558.8 725
3558 724.2
3557 724.2
3556.8 724
3556 723.2
3555.8 723
3555 722.2
3554.8 722
3554 721.2
3553.8 721
3553.8 720
3553 719.2
3552.8 719
3552 718.2
3551.8 718
3551 717.2
3550.8 717
3550 716.2
3549.8 716
3549.8 715
3549 714.2
3548.8 714
3548 713.2
3547.8 713
3547 712.2
3546.8 712
3546.8 711
3546 710.2
3545.8 710
3545.8 709
3545 708.2
3544.8 708
3545 707.8
3545.8 707
3545.8 706
3545.8 705
3545.8 704
3545.8 703
3546 702.8
3546.8 702
3546.8 701
3547 700.8
3547.8 700
3548 699.8
3548.8 699
3549 698.8
3549.8 698
3550 697.8
3550.8 697
3551 696.8
3551.8 696
3552 695.8
3553 695.8
3553.8 695
3554 694.8
3554.8 694
3555 693.8
3556 693.8
3556.8 693
3557 692.8
3558 692.8
3559 692.8
3560 692.8
3560.8 692
3561 691.8
3562 691.8
3563 691.8
3564 691.8
3565 691.8
3566 691.8
3567 691.8
3568 691.8
3569 691.8
3570 691.8
3570.2 692
3571 692.8
3572 692.8
3573 692.8
3573.2 693
3574 693.8
3575 693.8
3575.2 694
3576 694.8
3577 694.8
3577.2 695
3578 695.8
3579 695.8
3579.2 696
3580 696.8
3580.2 697
3581 697.8
3581.2 698
3582 698.8
3582.2 699
3583 699.8
3583.2 700
3584 700.8
3584.2 701
3584.2 702
3584.2 703
3585 703.8
3585.2 704
3585.2 705
3585.2 706
3586 706.8
3587 706.8
3587.8 706
3588 705.8
3588.8 705
3588.8 704
3589 703.8
3589.8 703
3590 702.8
3590.8 702
3591 701.8
3591.8 701
3592 700.8
3592.8 700
3593 699.8
3593.8 699
3594 698.8
3594.8 698
3595 697.8
3595.8 697
3596 696.8
3597 696.8
3597.8 696
3598 695.8
3599 695.8
3599.8 695
3600 694.8
3601 694.8
3602 694.8
3603 694.8
3604 694.8
3605 694.8
3605.8 694
3606 693.8
3607 693.8
3608 693.8
3609 693.8
3610 693.8
3611 693.8
3612 693.8
3612.2 694
3613 694.8
3614 694.8
3614.2 695
3614.2 696
3615 696.8
3616 696.8
3617 696.8
3617.8 696
3617.8 695
3617.8 694
3618 693.8
3618.8 693
3618.8 692
3618.8 691
3619 690.8
3619.8 690
3620 689.8
3620.8 689
3621 688.8
3621.8 688
3622 687.8
3622.8 687
3623 686.8
3624 686.8
3624.8 686
3625 685.8
3625.8 685
3626 684.8
3627 684.8
3628 684.8
3629 684.8
3629.8 684
3630 683.8
3631 683.8
3631.8 683
3632 682.8
3633 682.8
3634 682.8
3634.8 682
3635 681.8
3636 681.8
3637 681.8
3638 681.8
3639 681.8
3640 681.8
3641 681.8
3642 681.8
3643 681.8
3644 681.8
3645 681.8
3645.2 682
3646 682.8
3647 682.8
3648 682.8
3648.8 682
3649 681.8
3650 681.8
3650.8 681
3651 680.8
3652 680.8
3653 680.8
3654 680.8
3655 680.8
3656 680.8
3657 680.8
3658 680.8
3659 680.8
3660 680.8
3661 680.8
3662 680.8
3663 680.8
3664 680.8
3665 680.8
3666 680.8
3667 680.8
3668 680.8
3669 680.8
3670 680.8
3671 680.8
3672 680.8
3673 680.8
3673.2 681
3674 681.8
3675 681.8
3676 681.8
3677 681.8
3678 681.8
3678.2 682
3679 682.8
3680 682.8
3681 682.8
3682 682.8
3683 682.8
3684 682.8
3685 682.8
3685.2 683
3686 683.8
3686.2 684
3687 684.8
3687.2 685
3688 685.8
3688.2 686
3689 686.8
3689.8 686
3690 685.8
3691 685.8
3692 685.8
3693 685.8
3694 685.8
3695 685.8
3696 685.8
3697 685.8
3698 685.8
3699 685.8
3700 685.8
3701 685.8
3702 685.8
3702.8 685
3703 684.8
3704 684.8
3705 684.8
3706 684.8
3707 684.8
3708 684.8
3709 684.8
3710 684.8
3711 684.8
3711.2 685
3712 685.8
3712.8 685
3713 684.8
3713.2 685
3714 685.8
3715 685.8
3716 685.8
3716.2 686
3717 686.8
3717.2 687
3718 687.8
3719 687.8
3719.2 688
3720 688.8
3720.2 689
3721 689.8
3721.2 690
3721.2 691
3722 691.8
3723 691.8
3724 691.8
3725 691.8
3726 691.8
3727 691.8
3727.8 691
3728 690.8
3729 690.8
3730 690.8
3731 690.8
3732 690.8
3733 690.8
3733.2 691
3734 691.8
3735 691.8
3736 691.8
3737 691.8
3738 691.8
3739 691.8
3740 691.8
3740.2 692
3741 692.8
3742 692.8
3743 692.8
3744 692.8
3745 692.8
3746 692.8
3747 692.8
3747.2 693
3748 693.8
3749 693.8
3750 693.8
3750.2 694
3751 694.8
3752 694.8
3753 694.8
3754 694.8
3754.2 695
3755 695.8
3756 695.8
3757 695.8
3757.2 696
3758 696.8
3759 696.8
3760 696.8
3760.2 697
3761 697.8
3762 697.8
3763 697.8
3763.2 698
3764 698.8
3765 698.8
3765.2 699
3766 699.8
3767 699.8
3768 699.8
3768.2 700
3769 700.8
3770 700.8
3771 700.8
3771.2 701
3772 701.8
3773 701.8
3774 701.8
3775 701.8
3776 701.8
3776.2 702
3777 702.8
3778 702.8
3779 702.8
3780 702.8
3780.2 703
3781 703.8
3782 703.8
3783 703.8
3784 703.8
3784.2 704
3785 704.8
3786 704.8
3787 704.8
3788 704.8
3788.2 705
3789 705.8
3790 705.8
3791 705.8
3791.2 706
3792 706.8
3793 706.8
3793.2 707
3794 707.8
3794.2 708
3795 708.8
3796 708.8
3796.2 709
3797 709.8
3797.2 710
3798 710.8
3799 710.8
3799.2 711
3800 711.8
3800.2 712
3801 712.8
3801.2 713
3802 713.8
3802.2 714
3803 714.8
3804 714.8
3804.2 715
3805 715.8
3805.2 716
3806 716.8
3806.2 717
3807 717.8
3808 717.8
3808.2 718
3809 718.8
3809.2 719
3810 719.8
3810.2 720
3811 720.8
3811.2 721
3812 721.8
3812.2 722
3813 722.8
3813.2 723
3814 723.8
3814.2 724
3815 724.8
3815.2 725
3816 725.8
3816.2 726
3817 726.8
3817.2 727
3818 727.8
3818.2 728
3819 728.8
3819.2 729
3820 729.8
3820.2 730
3821 730.8
3821.2 731
3822 731.8
3822.2 732
3823 732.8
3823.2 733
3824 733.8
3824.2 734
3825 734.8
3825.2 735
3826 735.8
3826.2 736
3827 736.8
3827.2 737
3828 737.8
3828.2 738
3828.2 739
3829 739.8
3829.2 740
3830 740.8
3830.2 741
3830.2 742
3831 742.8
3831.2 743
3832 743.8
3832.2 744
3832.2 745
3833 745.8
3833.2 746
3833.2 747
3834 747.8
3834.2 748
3835 748.8
3835.2 749
3835.2 750
3835.2 751
3835 751.2
3834.2 752
3834.2 753
3834.2 754
3834.2 755
3834.2 756
3834 756.2
3833.2 757
3833.2 758
3833.2 759
3833.2 760
3833 760.2
3832.2 761
3832 761.2
3831 761.2
3830 761.2
3829.2 762
3829 762.2
3828 762.2
3827 762.2
3826.2 763
3826 763.2
3825 763.2
3824 763.2
3823.2 764
3823 764.2
3822 764.2
3821 764.2
3820.2 765
3820 765.2
3819 765.2
3818 765.2
3817 765.2
3816 765.2
3815.2 766
3815 766.2
3814 766.2
3813 766.2
3812 766.2
3811 766.2
3810 766.2
3809 766.2
3808 766.2
3807 766.2
3806.2 767
3806 767.2
3805 767.2
3804 767.2
3803 767.2
3802.2 768
3802 768.2
3801 768.2
3800 768.2
3799 768.2
3798 768.2
3797.2 769
3797 769.2
3796 769.2
3795 769.2
3794 769.2
3793 769.2
3792 769.2
3791.2 770
3791 770.2
3790 770.2
3789 770.2
3788 770.2
3787 770.2
3786.2 771
3786 771.2
3785 771.2
3784 771.2
3783 771.2
3782 771.2
3781 771.2
3780 771.2
3779 771.2
3778 771.2
3777 771.2
3776 771.2
3775 771.2
3774 771.2
3773 771.2
3772.8 771
3772 770.2
3771 770.2
3770 770.2
3769.8 770
3769 769.2
3768 769.2
3767 769.2
3766.8 769
3766 768.2
3765 768.2
3764.8 768
3764 767.2
3763 767.2
3762.2 768
3762 768.2
3761 768.2
3760.2 769
3760 769.2
3759 769.2
3758 769.2
3757.2 770
3757 770.2
3756 770.2
3755 770.2
3754 770.2
3753.2 771
3753 771.2
3752 771.2
3751 771.2
3750 771.2
3749 771.2
3748.2 772
3748 772.2
};
\addplot [red]
table {%
4335 707.2
4334 707.2
4333.8 707
4333 706.2
4332 706.2
4331.8 706
4331.8 705
4331 704.2
4330.8 704
4330.8 703
4330.8 702
4330.8 701
4330 700.2
4329.8 700
4329.8 699
4329.8 698
4329.8 697
4330 696.8
4330.8 696
4330.8 695
4330.8 694
4331 693.8
4331.8 693
4332 692.8
4332.8 692
4333 691.8
4334 691.8
4335 691.8
4335.2 692
4336 692.8
4336.2 693
4336.2 694
4337 694.8
4337.2 695
4337.2 696
4338 696.8
4338.2 697
4338.2 698
4338.2 699
4338.2 700
4339 700.8
4339.2 701
4339 701.2
4338.2 702
4338.2 703
4338.2 704
4338.2 705
4338 705.2
4337.2 706
4337 706.2
4336 706.2
4335.2 707
4335 707.2
};
\addplot [red]
table {%
4324 735.2
4323 735.2
4322 735.2
4321 735.2
4320 735.2
4319 735.2
4318 735.2
4317 735.2
4316 735.2
4315 735.2
4314 735.2
4313 735.2
4312 735.2
4311 735.2
4310.8 735
4310 734.2
4309.8 734
4309 733.2
4308.8 733
4308 732.2
4307.8 732
4307.8 731
4307.8 730
4307 729.2
4306 729.2
4305.8 729
4305 728.2
4304.8 728
4304 727.2
4303.8 727
4303.8 726
4303.8 725
4303 724.2
4302.8 724
4302.8 723
4303 722.8
4303.8 722
4303.8 721
4303 720.2
4302.2 721
4302 721.2
4301 721.2
4300.8 721
4300 720.2
4299 720.2
4298 720.2
4297.8 720
4297 719.2
4296.8 719
4296 718.2
4295 718.2
4294.8 718
4294 717.2
4293.8 717
4293 716.2
4292.8 716
4292 715.2
4291.8 715
4291.8 714
4291 713.2
4290.8 713
4290.8 712
4290.8 711
4290.8 710
4290.8 709
4290.8 708
4291 707.8
4291.8 707
4292 706.8
4292.8 706
4293 705.8
4293.8 705
4294 704.8
4295 704.8
4296 704.8
4296.8 704
4297 703.8
4298 703.8
4299 703.8
4300 703.8
4300.8 703
4301 702.8
4302 702.8
4302.2 703
4303 703.8
4303.2 704
4303.2 705
4303.2 706
4303.2 707
4304 707.8
4304.2 708
4304.2 709
4305 709.8
4305.2 710
4305.2 711
4306 711.8
4306.2 712
4306.2 713
4306.2 714
4307 714.8
4307.2 715
4308 715.8
4309 715.8
4310 715.8
4311 715.8
4312 715.8
4312.2 716
4313 716.8
4313.2 717
4314 717.8
4314.2 718
4314.2 719
4315 719.8
4315.8 719
4316 718.8
4316.8 718
4317 717.8
4318 717.8
4319 717.8
4319.8 717
4320 716.8
4320.2 717
4321 717.8
4322 717.8
4323 717.8
4324 717.8
4324.2 718
4325 718.8
4325.2 719
4326 719.8
4326.2 720
4327 720.8
4327.2 721
4328 721.8
4328.2 722
4328.2 723
4328.2 724
4328.2 725
4329 725.8
4329.2 726
4329 726.2
4328.2 727
4328.2 728
4328.2 729
4328.2 730
4328.2 731
4328 731.2
4327.2 732
4327 732.2
4326.2 733
4326 733.2
4325.2 734
4325 734.2
4324.2 735
4324 735.2
};
\addplot [red]
table {%
4336 723.2
4335 723.2
4334.8 723
4334 722.2
4333 722.2
4332.8 722
4332 721.2
4331.8 721
4331.8 720
4331.8 719
4331.8 718
4331.8 717
4332 716.8
4332.8 716
4332.8 715
4333 714.8
4333.8 714
4334 713.8
4335 713.8
4336 713.8
4337 713.8
4338 713.8
4339 713.8
4339.2 714
4340 714.8
4340.2 715
4341 715.8
4341.2 716
4341.2 717
4341.2 718
4341.2 719
4341 719.2
4340.2 720
4340.2 721
4340 721.2
4339.2 722
4339 722.2
4338 722.2
4337 722.2
4336.2 723
4336 723.2
};
\addplot [red]
table {%
6292 877.2
6291 877.2
6290 877.2
6289 877.2
6288 877.2
6287 877.2
6286 877.2
6285 877.2
6284 877.2
6283 877.2
6282 877.2
6281 877.2
6280 877.2
6279 877.2
6278.8 877
6278 876.2
6277 876.2
6276 876.2
6275 876.2
6274 876.2
6273 876.2
6272.8 876
6272 875.2
6271 875.2
6270.8 875
6270.8 874
6270 873.2
6269.8 873
6269.8 872
6269.8 871
6269 870.2
6268.8 870
6268.8 869
6268 868.2
6267.8 868
6267.8 867
6268 866.8
6268.8 866
6269 865.8
6269.8 865
6270 864.8
6270.8 864
6271 863.8
6271.8 863
6272 862.8
6272.8 862
6273 861.8
6273.8 861
6274 860.8
6274.8 860
6275 859.8
6275.8 859
6276 858.8
6276.8 858
6277 857.8
6278 857.8
6278.8 857
6279 856.8
6280 856.8
6280.8 856
6281 855.8
6282 855.8
6282.8 855
6283 854.8
6284 854.8
6284.8 854
6285 853.8
6286 853.8
6286.8 853
6287 852.8
6288 852.8
6288.8 852
6289 851.8
6290 851.8
6290.8 851
6291 850.8
6292 850.8
6293 850.8
6293.8 850
6294 849.8
6295 849.8
6295.8 849
6296 848.8
6297 848.8
6297.8 848
6298 847.8
6299 847.8
6300 847.8
6300.8 847
6301 846.8
6302 846.8
6302.8 846
6303 845.8
6304 845.8
6305 845.8
6305.8 845
6306 844.8
6307 844.8
6308 844.8
6308.8 844
6309 843.8
6310 843.8
6310.2 844
6311 844.8
6311.8 844
6312 843.8
6312.8 843
6312 842.2
6311.8 842
6311 841.2
6310.8 841
6310 840.2
6309.8 840
6309 839.2
6308.8 839
6308 838.2
6307.8 838
6307 837.2
6306.8 837
6306 836.2
6305.8 836
6305 835.2
6304.8 835
6304 834.2
6303.8 834
6303 833.2
6302.8 833
6302 832.2
6301.8 832
6301 831.2
6300.8 831
6300 830.2
6299.8 830
6299 829.2
6298.8 829
6298.8 828
6298.8 827
6298 826.2
6297.8 826
6297.8 825
6297.8 824
6297.8 823
6297 822.2
6296.8 822
6296.8 821
6296.8 820
6296.8 819
6296.8 818
6296 817.2
6295.8 817
6295.8 816
6295.8 815
6296 814.8
6296.8 814
6296.8 813
6297 812.8
6297.8 812
6297.8 811
6298 810.8
6298.8 810
6298.8 809
6299 808.8
6299.8 808
6299.8 807
6300 806.8
6300.8 806
6300.8 805
6301 804.8
6301.8 804
6301.8 803
6302 802.8
6302.8 802
6303 801.8
6303.8 801
6304 800.8
6304.8 800
6305 799.8
6305.8 799
6305.8 798
6306 797.8
6306.8 797
6307 796.8
6307.8 796
6308 795.8
6308.8 795
6309 794.8
6309.8 794
6310 793.8
6310.8 793
6311 792.8
6311.8 792
6312 791.8
6312.8 791
6313 790.8
6313.8 790
6314 789.8
6315 789.8
6315.8 789
6316 788.8
6316.8 788
6317 787.8
6317.8 787
6318 786.8
6318.8 786
6319 785.8
6320 785.8
6320.8 785
6321 784.8
6322 784.8
6322.8 784
6323 783.8
6324 783.8
6325 783.8
6325.8 783
6326 782.8
6327 782.8
6328 782.8
6328.8 782
6329 781.8
6330 781.8
6331 781.8
6331.8 781
6332 780.8
6333 780.8
6334 780.8
6335 780.8
6335.8 780
6336 779.8
6337 779.8
6338 779.8
6339 779.8
6340 779.8
6341 779.8
6341.8 779
6342 778.8
6343 778.8
6344 778.8
6345 778.8
6346 778.8
6347 778.8
6348 778.8
6348.8 778
6349 777.8
6350 777.8
6351 777.8
6352 777.8
6353 777.8
6354 777.8
6355 777.8
6356 777.8
6357 777.8
6358 777.8
6359 777.8
6360 777.8
6361 777.8
6362 777.8
6363 777.8
6364 777.8
6365 777.8
6366 777.8
6367 777.8
6368 777.8
6369 777.8
6370 777.8
6371 777.8
6372 777.8
6373 777.8
6373.2 778
6374 778.8
6375 778.8
6376 778.8
6376.2 779
6377 779.8
6378 779.8
6378.2 780
6379 780.8
6380 780.8
6381 780.8
6381.2 781
6382 781.8
6382.2 782
6383 782.8
6383.2 783
6384 783.8
6384.2 784
6385 784.8
6385.2 785
6386 785.8
6386.2 786
6386.2 787
6387 787.8
6387.2 788
6388 788.8
6388.2 789
6388.2 790
6389 790.8
6389.2 791
6389.2 792
6389.2 793
6390 793.8
6390.2 794
6390.2 795
6390.2 796
6391 796.8
6391.2 797
6391.2 798
6391.2 799
6392 799.8
6392.2 800
6392.2 801
6393 801.8
6393.2 802
6393.2 803
6393.2 804
6394 804.8
6394.2 805
6394.2 806
6394.2 807
6395 807.8
6395.2 808
6395.2 809
6395.2 810
6395.2 811
6395.2 812
6395.2 813
6396 813.8
6396.2 814
6396.2 815
6396.2 816
6396.2 817
6396.2 818
6396 818.2
6395.2 819
6395.2 820
6395.2 821
6395.2 822
6395.2 823
6395 823.2
6394.2 824
6394.2 825
6394 825.2
6393.2 826
6393.2 827
6393 827.2
6392.2 828
6392.2 829
6392 829.2
6391.2 830
6391.2 831
6391 831.2
6390.2 832
6390 832.2
6389.2 833
6389 833.2
6388.2 834
6388 834.2
6387.2 835
6387 835.2
6386.2 836
6386 836.2
6385.2 837
6385.2 838
6385 838.2
6384.2 839
6384 839.2
6383.2 840
6383 840.2
6382.2 841
6382 841.2
6381.2 842
6381 842.2
6380.2 843
6380 843.2
6379.2 844
6379 844.2
6378.2 845
6378 845.2
6377.2 846
6377 846.2
6376.2 847
6376.2 848
6376 848.2
6375.2 849
6375 849.2
6374.2 850
6374 850.2
6373.2 851
6373 851.2
6372.2 852
6372 852.2
6371.2 853
6371 853.2
6370 853.2
6369 853.2
6368 853.2
6367.2 854
6367 854.2
6366 854.2
6365 854.2
6364 854.2
6363.2 855
6363 855.2
6362 855.2
6361 855.2
6360 855.2
6359.2 856
6359 856.2
6358 856.2
6357 856.2
6356 856.2
6355.2 857
6355 857.2
6354 857.2
6353 857.2
6352 857.2
6351.2 858
6351 858.2
6350 858.2
6349 858.2
6348 858.2
6347 858.2
6346 858.2
6345 858.2
6344 858.2
6343 858.2
6342 858.2
6341 858.2
6340 858.2
6339.2 859
6339.2 860
6339.2 861
6339.2 862
6339.2 863
6339 863.2
6338.2 864
6338 864.2
6337.2 865
6337 865.2
6336.2 866
6336 866.2
6335.2 867
6335 867.2
6334 867.2
6333.2 868
6333 868.2
6332 868.2
6331 868.2
6330.2 869
6330 869.2
6329 869.2
6328 869.2
6327.2 870
6327 870.2
6326 870.2
6325 870.2
6324.2 871
6324 871.2
6323 871.2
6322 871.2
6321 871.2
6320.2 872
6320 872.2
6319 872.2
6318 872.2
6317 872.2
6316 872.2
6315.2 873
6315 873.2
6314 873.2
6313 873.2
6312 873.2
6311.2 874
6311 874.2
6310 874.2
6309 874.2
6308.2 875
6308 875.2
6307 875.2
6306 875.2
6305 875.2
6304 875.2
6303 875.2
6302 875.2
6301.2 876
6301 876.2
6300 876.2
6299 876.2
6298 876.2
6297 876.2
6296 876.2
6295 876.2
6294 876.2
6293 876.2
6292.2 877
6292 877.2
};
\addplot [red]
table {%
3168 802.2
3167 802.2
3166.8 802
3166 801.2
3165 801.2
3164 801.2
3163 801.2
3162 801.2
3161.8 801
3161 800.2
3160 800.2
3159 800.2
3158 800.2
3157 800.2
3156.8 800
3156 799.2
3155 799.2
3154.8 799
3154 798.2
3153 798.2
3152.8 798
3152 797.2
3151.2 798
3151 798.2
3150.2 799
3150.2 800
3150 800.2
3149 800.2
3148.2 801
3148 801.2
3147 801.2
3146.8 801
3146 800.2
3145 800.2
3144.2 801
3144 801.2
3143 801.2
3142 801.2
3141 801.2
3140 801.2
3139 801.2
3138.8 801
3138 800.2
3137 800.2
3136 800.2
3135 800.2
3134 800.2
3133 800.2
3132 800.2
3131 800.2
3130 800.2
3129 800.2
3128 800.2
3127 800.2
3126 800.2
3125 800.2
3124 800.2
3123 800.2
3122 800.2
3121 800.2
3120 800.2
3119 800.2
3118 800.2
3117 800.2
3116 800.2
3115.8 800
3115 799.2
3114.8 799
3114 798.2
3113 798.2
3112.8 798
3112 797.2
3111.8 797
3111 796.2
3110 796.2
3109.8 796
3109 795.2
3108 795.2
3107.8 795
3107 794.2
3106.8 794
3107 793.8
3108 793.8
3109 793.8
3110 793.8
3111 793.8
3112 793.8
3113 793.8
3113.8 793
3114 792.8
3115 792.8
3116 792.8
3117 792.8
3118 792.8
3119 792.8
3120 792.8
3121 792.8
3121.8 792
3122 791.8
3123 791.8
3124 791.8
3125 791.8
3126 791.8
3127 791.8
3128 791.8
3128.8 791
3129 790.8
3130 790.8
3131 790.8
3132 790.8
3133 790.8
3134 790.8
3135 790.8
3136 790.8
3137 790.8
3138 790.8
3139 790.8
3139.2 791
3140 791.8
3141 791.8
3141.8 791
3142 790.8
3143 790.8
3144 790.8
3145 790.8
3146 790.8
3146.8 790
3147 789.8
3148 789.8
3148.2 790
3149 790.8
3150 790.8
3150.8 790
3151 789.8
3151.2 790
3152 790.8
3152.2 791
3153 791.8
3154 791.8
3154.2 792
3155 792.8
3156 792.8
3156.2 793
3157 793.8
3158 793.8
3159 793.8
3159.2 794
3160 794.8
3161 794.8
3161.2 795
3162 795.8
3163 795.8
3163.2 796
3164 796.8
3165 796.8
3165.2 797
3166 797.8
3167 797.8
3167.2 798
3167.2 799
3167.2 800
3167.2 801
3168 801.8
3168.2 802
3168 802.2
};
\addplot [red]
table {%
3079 869.2
3078 869.2
3077 869.2
3076 869.2
3075.8 869
3075 868.2
3074 868.2
3073 868.2
3072 868.2
3071 868.2
3070 868.2
3069 868.2
3068.8 868
3068 867.2
3067 867.2
3066 867.2
3065.8 867
3065 866.2
3064 866.2
3063 866.2
3062.8 866
3062 865.2
3061 865.2
3060 865.2
3059.8 865
3059 864.2
3058 864.2
3057 864.2
3056.8 864
3056 863.2
3055 863.2
3054.8 863
3054 862.2
3053 862.2
3052.8 862
3052 861.2
3051 861.2
3050.8 861
3050 860.2
3049.8 860
3049 859.2
3048.8 859
3048 858.2
3047.8 858
3047 857.2
3046 857.2
3045.8 857
3045.8 856
3045 855.2
3044.8 855
3044 854.2
3043.8 854
3043 853.2
3042.8 853
3042 852.2
3041.8 852
3041 851.2
3040.8 851
3040 850.2
3039.8 850
3039 849.2
3038.8 849
3038 848.2
3037 848.2
3036.8 848
3036 847.2
3035.8 847
3035 846.2
3034.8 846
3034 845.2
3033.8 845
3033 844.2
3032.8 844
3032 843.2
3031 843.2
3030.8 843
3030 842.2
3029.8 842
3029 841.2
3028.8 841
3028 840.2
3027.8 840
3027 839.2
3026.8 839
3026 838.2
3025.8 838
3025 837.2
3024.8 837
3024 836.2
3023.8 836
3023 835.2
3022.8 835
3022 834.2
3021.8 834
3021.8 833
3021 832.2
3020.8 832
3020.8 831
3020 830.2
3019.8 830
3019.8 829
3019.8 828
3019 827.2
3018.8 827
3018.8 826
3018 825.2
3017.8 825
3018 824.8
3018.8 824
3018.8 823
3018.8 822
3018.8 821
3019 820.8
3019.8 820
3019.8 819
3019.8 818
3019.8 817
3020 816.8
3020.8 816
3021 815.8
3021.8 815
3022 814.8
3022.8 814
3022.8 813
3023 812.8
3023.8 812
3024 811.8
3024.8 811
3025 810.8
3025.8 810
3026 809.8
3026.8 809
3027 808.8
3028 808.8
3028.8 808
3029 807.8
3029.8 807
3030 806.8
3030.8 806
3031 805.8
3031.8 805
3032 804.8
3033 804.8
3033.8 804
3034 803.8
3035 803.8
3035.8 803
3036 802.8
3037 802.8
3037.8 802
3038 801.8
3039 801.8
3039.8 801
3040 800.8
3041 800.8
3042 800.8
3043 800.8
3044 800.8
3045 800.8
3045.8 800
3046 799.8
3047 799.8
3048 799.8
3049 799.8
3050 799.8
3051 799.8
3051.2 800
3052 800.8
3053 800.8
3054 800.8
3055 800.8
3056 800.8
3057 800.8
3058 800.8
3059 800.8
3060 800.8
3061 800.8
3062 800.8
3063 800.8
3064 800.8
3065 800.8
3066 800.8
3067 800.8
3068 800.8
3068.8 800
3069 799.8
3070 799.8
3071 799.8
3071.8 799
3072 798.8
3073 798.8
3073.8 798
3074 797.8
3075 797.8
3076 797.8
3076.8 797
3077 796.8
3078 796.8
3079 796.8
3079.8 796
3080 795.8
3081 795.8
3082 795.8
3082.8 795
3083 794.8
3084 794.8
3085 794.8
3085.8 794
3086 793.8
3087 793.8
3088 793.8
3088.8 793
3089 792.8
3090 792.8
3091 792.8
3092 792.8
3093 792.8
3094 792.8
3095 792.8
3096 792.8
3097 792.8
3098 792.8
3099 792.8
3100 792.8
3101 792.8
3102 792.8
3103 792.8
3103.2 793
3104 793.8
3104.2 794
3104.2 795
3105 795.8
3105.2 796
3105.2 797
3106 797.8
3106.2 798
3106.2 799
3107 799.8
3107.2 800
3107.2 801
3108 801.8
3108.2 802
3108.2 803
3108.2 804
3108 804.2
3107.2 805
3107.2 806
3107 806.2
3106.2 807
3106.2 808
3106 808.2
3105.2 809
3105 809.2
3104.2 810
3104.2 811
3104 811.2
3103.2 812
3103.2 813
3103 813.2
3102.2 814
3102.2 815
3102 815.2
3101.2 816
3101.2 817
3101 817.2
3100.2 818
3100 818.2
3099.2 819
3099.2 820
3099 820.2
3098.2 821
3098 821.2
3097.2 822
3097 822.2
3096.2 823
3096.2 824
3096 824.2
3095.2 825
3095 825.2
3094.2 826
3094 826.2
3093.2 827
3093.2 828
3093 828.2
3092.2 829
3092.2 830
3092.2 831
3092.2 832
3092.2 833
3092 833.2
3091.2 834
3092 834.8
3092.2 835
3093 835.8
3093.2 836
3094 836.8
3094.2 837
3094.2 838
3095 838.8
3095.2 839
3096 839.8
3096.2 840
3097 840.8
3097.2 841
3097.2 842
3098 842.8
3098.2 843
3099 843.8
3099.2 844
3099.2 845
3100 845.8
3100.2 846
3101 846.8
3101.2 847
3101.2 848
3102 848.8
3102.2 849
3103 849.8
3103.2 850
3103.2 851
3104 851.8
3104.2 852
3104.2 853
3104.2 854
3104.2 855
3104.2 856
3105 856.8
3105.2 857
3105.2 858
3105.2 859
3105.2 860
3105.2 861
3106 861.8
3106.2 862
3106.2 863
3106 863.2
3105.2 864
3105 864.2
3104 864.2
3103.2 865
3103 865.2
3102.2 866
3102 866.2
3101 866.2
3100.2 867
3100 867.2
3099 867.2
3098.2 868
3098 868.2
3097 868.2
3096 868.2
3095 868.2
3094 868.2
3093 868.2
3092 868.2
3091 868.2
3090 868.2
3089 868.2
3088 868.2
3087 868.2
3086 868.2
3085 868.2
3084 868.2
3083 868.2
3082 868.2
3081 868.2
3080 868.2
3079.2 869
3079 869.2
};
\addplot [red]
table {%
4140 830.2
4139 830.2
4138 830.2
4137 830.2
4136 830.2
4135 830.2
4134 830.2
4133 830.2
4132 830.2
4131 830.2
4130 830.2
4129 830.2
4128 830.2
4127 830.2
4126 830.2
4125 830.2
4124 830.2
4123 830.2
4122 830.2
4121 830.2
4120 830.2
4119 830.2
4118.8 830
4118 829.2
4117 829.2
4116 829.2
4115 829.2
4114 829.2
4113 829.2
4112 829.2
4111 829.2
4110 829.2
4109 829.2
4108 829.2
4107 829.2
4106 829.2
4105 829.2
4104 829.2
4103 829.2
4102 829.2
4101.8 829
4101 828.2
4100.2 829
4100 829.2
4099 829.2
4098 829.2
4097 829.2
4096.8 829
4096 828.2
4095 828.2
4094 828.2
4093 828.2
4092 828.2
4091 828.2
4090 828.2
4089 828.2
4088 828.2
4087 828.2
4086 828.2
4085 828.2
4084 828.2
4083.8 828
4083 827.2
4082 827.2
4081 827.2
4080 827.2
4079.8 827
4079 826.2
4078 826.2
4077 826.2
4076 826.2
4075 826.2
4074 826.2
4073.8 826
4073 825.2
4072 825.2
4071 825.2
4070 825.2
4069 825.2
4068 825.2
4067 825.2
4066 825.2
4065 825.2
4064.8 825
4064 824.2
4063 824.2
4062 824.2
4061 824.2
4060 824.2
4059 824.2
4058 824.2
4057 824.2
4056 824.2
4055 824.2
4054 824.2
4053 824.2
4052.8 824
4052 823.2
4051 823.2
4050 823.2
4049 823.2
4048 823.2
4047.8 823
4047 822.2
4046 822.2
4045.8 822
4045 821.2
4044 821.2
4043 821.2
4042.8 821
4042 820.2
4041 820.2
4040.8 820
4040 819.2
4039 819.2
4038 819.2
4037.8 819
4037 818.2
4036 818.2
4035.8 818
4035 817.2
4034 817.2
4033.8 817
4033 816.2
4032 816.2
4031 816.2
4030.8 816
4030 815.2
4029 815.2
4028.8 815
4028 814.2
4027 814.2
4026 814.2
4025.8 814
4026 813.8
4026.8 813
4027 812.8
4027.8 812
4028 811.8
4028.8 811
4029 810.8
4029.8 810
4030 809.8
4030.8 809
4031 808.8
4031.8 808
4032 807.8
4032.8 807
4032.8 806
4033 805.8
4033.8 805
4034 804.8
4034.8 804
4035 803.8
4035.8 803
4036 802.8
4036.8 802
4037 801.8
4038 801.8
4039 801.8
4040 801.8
4041 801.8
4042 801.8
4043 801.8
4044 801.8
4044.8 801
4045 800.8
4046 800.8
4047 800.8
4048 800.8
4049 800.8
4050 800.8
4051 800.8
4052 800.8
4053 800.8
4054 800.8
4054.8 800
4055 799.8
4056 799.8
4057 799.8
4058 799.8
4059 799.8
4060 799.8
4061 799.8
4062 799.8
4063 799.8
4064 799.8
4065 799.8
4066 799.8
4067 799.8
4068 799.8
4069 799.8
4070 799.8
4071 799.8
4072 799.8
4073 799.8
4074 799.8
4074.8 799
4075 798.8
4076 798.8
4077 798.8
4078 798.8
4079 798.8
4080 798.8
4081 798.8
4082 798.8
4083 798.8
4084 798.8
4085 798.8
4086 798.8
4087 798.8
4088 798.8
4088.8 798
4089 797.8
4090 797.8
4091 797.8
4092 797.8
4092.2 798
4093 798.8
4094 798.8
4095 798.8
4096 798.8
4097 798.8
4098 798.8
4099 798.8
4100 798.8
4101 798.8
4102 798.8
4103 798.8
4104 798.8
4105 798.8
4106 798.8
4106.2 799
4107 799.8
4108 799.8
4109 799.8
4110 799.8
4111 799.8
4112 799.8
4113 799.8
4114 799.8
4114.2 800
4115 800.8
4116 800.8
4117 800.8
4118 800.8
4118.2 801
4119 801.8
4120 801.8
4121 801.8
4122 801.8
4123 801.8
4124 801.8
4125 801.8
4126 801.8
4127 801.8
4128 801.8
4129 801.8
4130 801.8
4131 801.8
4132 801.8
4133 801.8
4134 801.8
4135 801.8
4136 801.8
4137 801.8
4138 801.8
4138.2 802
4139 802.8
4140 802.8
4141 802.8
4142 802.8
4143 802.8
4144 802.8
4145 802.8
4146 802.8
4147 802.8
4148 802.8
4149 802.8
4150 802.8
4151 802.8
4152 802.8
4153 802.8
4154 802.8
4155 802.8
4155.2 803
4156 803.8
4157 803.8
4157.2 804
4158 804.8
4159 804.8
4159.2 805
4160 805.8
4161 805.8
4161.2 806
4162 806.8
4162.2 807
4163 807.8
4164 807.8
4164.2 808
4165 808.8
4166 808.8
4166.2 809
4167 809.8
4168 809.8
4168.2 810
4169 810.8
4170 810.8
4170.2 811
4171 811.8
4172 811.8
4172.2 812
4173 812.8
4174 812.8
4174.2 813
4175 813.8
4176 813.8
4176.2 814
4176 814.2
4175.2 815
4175 815.2
4174.2 816
4174 816.2
4173.2 817
4173 817.2
4172 817.2
4171.2 818
4171 818.2
4170.2 819
4170 819.2
4169.2 820
4169 820.2
4168.2 821
4168 821.2
4167.2 822
4167 822.2
4166.2 823
4166 823.2
4165.2 824
4165 824.2
4164.2 825
4164 825.2
4163.2 826
4163 826.2
4162 826.2
4161 826.2
4160.2 827
4160 827.2
4159 827.2
4158 827.2
4157 827.2
4156 827.2
4155 827.2
4154.2 828
4154 828.2
4153 828.2
4152 828.2
4151 828.2
4150 828.2
4149 828.2
4148 828.2
4147.2 829
4147 829.2
4146 829.2
4145 829.2
4144 829.2
4143 829.2
4142 829.2
4141 829.2
4140.2 830
4140 830.2
};
\addplot [red]
table {%
4242 840.2
4241 840.2
4240 840.2
4239 840.2
4238 840.2
4237 840.2
4236 840.2
4235 840.2
4234 840.2
4233 840.2
4232 840.2
4231.8 840
4231 839.2
4230 839.2
4229 839.2
4228 839.2
4227.8 839
4227 838.2
4226 838.2
4225 838.2
4224.8 838
4224 837.2
4223 837.2
4222.8 837
4222 836.2
4221 836.2
4220 836.2
4219.8 836
4219 835.2
4218.8 835
4218 834.2
4217 834.2
4216.8 834
4216 833.2
4215.8 833
4215.8 832
4215 831.2
4214.8 831
4214 830.2
4213.8 830
4213.8 829
4213 828.2
4212.8 828
4212.8 827
4212 826.2
4211.8 826
4211.8 825
4211 824.2
4210.8 824
4210.8 823
4210.8 822
4210.8 821
4210.8 820
4210.8 819
4210.8 818
4210.8 817
4211 816.8
4211.8 816
4212 815.8
4212.8 815
4213 814.8
4214 814.8
4214.8 814
4215 813.8
4216 813.8
4217 813.8
4217.8 813
4218 812.8
4219 812.8
4220 812.8
4221 812.8
4222 812.8
4223 812.8
4224 812.8
4225 812.8
4226 812.8
4227 812.8
4227.8 812
4228 811.8
4228.2 812
4229 812.8
4230 812.8
4231 812.8
4232 812.8
4233 812.8
4234 812.8
4235 812.8
4236 812.8
4237 812.8
4238 812.8
4238.2 813
4239 813.8
4240 813.8
4241 813.8
4242 813.8
4242.2 814
4243 814.8
4244 814.8
4245 814.8
4246 814.8
4246.2 815
4247 815.8
4248 815.8
4248.2 816
4249 816.8
4250 816.8
4251 816.8
4251.2 817
4252 817.8
4253 817.8
4254 817.8
4255 817.8
4255.2 818
4256 818.8
4257 818.8
4258 818.8
4259 818.8
4260 818.8
4261 818.8
4261.2 819
4262 819.8
4263 819.8
4264 819.8
4265 819.8
4266 819.8
4266.2 820
4267 820.8
4267.2 821
4268 821.8
4268.2 822
4269 822.8
4269.2 823
4270 823.8
4270.2 824
4271 824.8
4271.2 825
4272 825.8
4272.2 826
4272 826.2
4271.2 827
4271.2 828
4271 828.2
4270.2 829
4270.2 830
4270 830.2
4269.2 831
4269.2 832
4269 832.2
4268.2 833
4268 833.2
4267 833.2
4266 833.2
4265.2 834
4265 834.2
4264 834.2
4263 834.2
4262.2 835
4262 835.2
4261 835.2
4260.2 836
4260 836.2
4259 836.2
4258 836.2
4257 836.2
4256.2 837
4256 837.2
4255 837.2
4254 837.2
4253 837.2
4252.2 838
4252 838.2
4251 838.2
4250 838.2
4249 838.2
4248.2 839
4248 839.2
4247 839.2
4246 839.2
4245 839.2
4244 839.2
4243 839.2
4242.2 840
4242 840.2
};
\addplot [red]
table {%
5745 885.2
5744 885.2
5743 885.2
5742 885.2
5741 885.2
5740 885.2
5739 885.2
5738 885.2
5737 885.2
5736 885.2
5735 885.2
5734 885.2
5733 885.2
5732.8 885
5732 884.2
5731 884.2
5730 884.2
5729 884.2
5728.8 884
5728 883.2
5727 883.2
5726 883.2
5725 883.2
5724 883.2
5723 883.2
5722 883.2
5721 883.2
5720 883.2
5719 883.2
5718 883.2
5717.8 883
5717 882.2
5716 882.2
5715 882.2
5714.8 882
5714 881.2
5713.8 881
5713.8 880
5713 879.2
5712.8 879
5712 878.2
5711.8 878
5712 877.8
5712.8 877
5712.8 876
5713 875.8
5713.8 875
5713.8 874
5714 873.8
5714.8 873
5714.8 872
5714.8 871
5714.8 870
5714.8 869
5714.8 868
5714.8 867
5715 866.8
5715.8 866
5715.8 865
5715.8 864
5715.8 863
5715.8 862
5715.8 861
5716 860.8
5716.8 860
5716.8 859
5716.8 858
5717 857.8
5717.8 857
5718 856.8
5718.8 856
5719 855.8
5719.8 855
5720 854.8
5721 854.8
5722 854.8
5722.8 854
5723 853.8
5724 853.8
5724.2 854
5725 854.8
5726 854.8
5727 854.8
5728 854.8
5729 854.8
5729.2 855
5730 855.8
5731 855.8
5732 855.8
5732.2 856
5733 856.8
5734 856.8
5734.2 857
5735 857.8
5735.2 858
5735.2 859
5736 859.8
5736.2 860
5737 860.8
5738 860.8
5739 860.8
5740 860.8
5740.8 860
5741 859.8
5742 859.8
5743 859.8
5744 859.8
5745 859.8
5746 859.8
5746.8 859
5747 858.8
5748 858.8
5749 858.8
5750 858.8
5751 858.8
5752 858.8
5753 858.8
5754 858.8
5755 858.8
5756 858.8
5757 858.8
5758 858.8
5759 858.8
5760 858.8
5761 858.8
5762 858.8
5763 858.8
5764 858.8
5765 858.8
5766 858.8
5766.8 858
5766 857.2
5765.8 857
5765 856.2
5764.8 856
5764.8 855
5764 854.2
5763.8 854
5763.8 853
5763 852.2
5762.8 852
5762.8 851
5762 850.2
5761.8 850
5761 849.2
5760.8 849
5760 848.2
5759.8 848
5759 847.2
5758.8 847
5758.8 846
5758 845.2
5757.8 845
5757 844.2
5756.8 844
5756.8 843
5756 842.2
5755.8 842
5755.8 841
5755.8 840
5755 839.2
5754.8 839
5754.8 838
5754.8 837
5754 836.2
5753.8 836
5753.8 835
5753.8 834
5753.8 833
5753.8 832
5753.8 831
5753.8 830
5753.8 829
5753.8 828
5754 827.8
5754.8 827
5754.8 826
5754.8 825
5755 824.8
5755.8 824
5756 823.8
5756.8 823
5757 822.8
5757.8 822
5758 821.8
5758.8 821
5759 820.8
5759.8 820
5760 819.8
5761 819.8
5762 819.8
5762.8 819
5763 818.8
5764 818.8
5764.8 818
5765 817.8
5766 817.8
5767 817.8
5768 817.8
5769 817.8
5769.8 817
5770 816.8
5771 816.8
5772 816.8
5773 816.8
5774 816.8
5774.8 816
5775 815.8
5776 815.8
5777 815.8
5778 815.8
5778.8 815
5779 814.8
5780 814.8
5781 814.8
5782 814.8
5783 814.8
5784 814.8
5785 814.8
5786 814.8
5787 814.8
5788 814.8
5789 814.8
5790 814.8
5791 814.8
5791.2 815
5792 815.8
5793 815.8
5794 815.8
5795 815.8
5796 815.8
5797 815.8
5798 815.8
5799 815.8
5800 815.8
5801 815.8
5801.2 816
5802 816.8
5803 816.8
5804 816.8
5805 816.8
5806 816.8
5807 816.8
5808 816.8
5808.2 817
5809 817.8
5810 817.8
5811 817.8
5812 817.8
5812.2 818
5813 818.8
5814 818.8
5815 818.8
5816 818.8
5816.2 819
5817 819.8
5818 819.8
5818.2 820
5819 820.8
5819.2 821
5820 821.8
5821 821.8
5821.2 822
5822 822.8
5823 822.8
5823.2 823
5824 823.8
5824.2 824
5825 824.8
5826 824.8
5826.2 825
5826.2 826
5827 826.8
5827.2 827
5828 827.8
5828.2 828
5829 828.8
5829.2 829
5830 829.8
5830.2 830
5830.2 831
5831 831.8
5831.2 832
5832 832.8
5832.2 833
5833 833.8
5833.2 834
5833.2 835
5833.2 836
5834 836.8
5834.2 837
5834.2 838
5835 838.8
5835.2 839
5835.2 840
5835.2 841
5836 841.8
5836.2 842
5836.2 843
5837 843.8
5837.2 844
5837.2 845
5837.2 846
5837.2 847
5837.2 848
5837.2 849
5837.2 850
5837.2 851
5837.2 852
5837.2 853
5837.2 854
5837.2 855
5837.2 856
5837.2 857
5837 857.2
5836.2 858
5836.2 859
5836 859.2
5835.2 860
5835.2 861
5835.2 862
5835 862.2
5834.2 863
5834.2 864
5834 864.2
5833.2 865
5833.2 866
5833 866.2
5832.2 867
5832.2 868
5832 868.2
5831.2 869
5831 869.2
5830 869.2
5829.2 870
5829 870.2
5828.2 871
5828 871.2
5827 871.2
5826.2 872
5826 872.2
5825.2 873
5825 873.2
5824 873.2
5823.2 874
5823 874.2
5822 874.2
5821.2 875
5821 875.2
5820 875.2
5819 875.2
5818 875.2
5817.2 876
5817 876.2
5816 876.2
5815 876.2
5814 876.2
5813 876.2
5812.2 877
5812 877.2
5811 877.2
5810 877.2
5809 877.2
5808 877.2
5807 877.2
5806.2 878
5806 878.2
5805 878.2
5804 878.2
5803 878.2
5802 878.2
5801.2 879
5801 879.2
5800 879.2
5799 879.2
5798 879.2
5797.2 880
5797 880.2
5796 880.2
5795 880.2
5794.2 881
5794 881.2
5793 881.2
5792 881.2
5791.2 882
5791 882.2
5790 882.2
5789 882.2
5788.8 882
5788 881.2
5787 881.2
5786 881.2
5785 881.2
5784 881.2
5783 881.2
5782 881.2
5781 881.2
5780.2 882
5780 882.2
5779 882.2
5778 882.2
5777 882.2
5776 882.2
5775 882.2
5774 882.2
5773 882.2
5772 882.2
5771 882.2
5770.2 883
5770 883.2
5769 883.2
5768 883.2
5767 883.2
5766 883.2
5765 883.2
5764 883.2
5763 883.2
5762 883.2
5761 883.2
5760 883.2
5759 883.2
5758 883.2
5757.2 884
5757 884.2
5756 884.2
5755 884.2
5754 884.2
5753 884.2
5752 884.2
5751 884.2
5750 884.2
5749 884.2
5748 884.2
5747 884.2
5746 884.2
5745.2 885
5745 885.2
};
\addplot [red]
table {%
5730 851.2
5729 851.2
5728 851.2
5727 851.2
5726 851.2
5725 851.2
5724 851.2
5723 851.2
5722 851.2
5721 851.2
5720.8 851
5720 850.2
5719 850.2
5718.8 850
5718 849.2
5717 849.2
5716.8 849
5716 848.2
5715.8 848
5715.8 847
5715.8 846
5715 845.2
5714.8 845
5714.8 844
5714.8 843
5714.8 842
5714.8 841
5714.8 840
5714.8 839
5714.8 838
5715 837.8
5715.8 837
5715.8 836
5715 835.2
5714 835.2
5713 835.2
5712 835.2
5711 835.2
5710 835.2
5709 835.2
5708 835.2
5707 835.2
5706 835.2
5705 835.2
5704 835.2
5703 835.2
5702.8 835
5702 834.2
5701 834.2
5700 834.2
5699.8 834
5699 833.2
5698 833.2
5697 833.2
5696 833.2
5695 833.2
5694 833.2
5693 833.2
5692 833.2
5691.8 833
5691 832.2
5690 832.2
5689 832.2
5688 832.2
5687 832.2
5686 832.2
5685 832.2
5684 832.2
5683 832.2
5682 832.2
5681 832.2
5680 832.2
5679 832.2
5678.8 832
5678 831.2
5677 831.2
5676 831.2
5675.8 831
5675 830.2
5674 830.2
5673.8 830
5673 829.2
5672 829.2
5671.8 829
5671 828.2
5670 828.2
5669.8 828
5669 827.2
5668 827.2
5667.8 827
5667 826.2
5666 826.2
5665.8 826
5666 825.8
5666.8 825
5667 824.8
5667.8 824
5667.8 823
5668 822.8
5668.8 822
5669 821.8
5669.8 821
5670 820.8
5670.8 820
5671 819.8
5671.8 819
5672 818.8
5673 818.8
5674 818.8
5675 818.8
5675.8 818
5676 817.8
5677 817.8
5678 817.8
5679 817.8
5680 817.8
5681 817.8
5682 817.8
5683 817.8
5683.8 817
5684 816.8
5685 816.8
5686 816.8
5687 816.8
5688 816.8
5689 816.8
5690 816.8
5691 816.8
5692 816.8
5693 816.8
5694 816.8
5695 816.8
5696 816.8
5697 816.8
5698 816.8
5699 816.8
5700 816.8
5700.2 817
5701 817.8
5702 817.8
5703 817.8
5704 817.8
5705 817.8
5706 817.8
5707 817.8
5708 817.8
5709 817.8
5710 817.8
5710.2 818
5711 818.8
5712 818.8
5713 818.8
5714 818.8
5715 818.8
5716 818.8
5717 818.8
5718 818.8
5719 818.8
5720 818.8
5720.2 819
5721 819.8
5722 819.8
5723 819.8
5723.2 820
5724 820.8
5725 820.8
5725.8 820
5726 819.8
5726.8 819
5727 818.8
5728 818.8
5729 818.8
5730 818.8
5730.8 818
5731 817.8
5732 817.8
5733 817.8
5734 817.8
5735 817.8
5736 817.8
5737 817.8
5738 817.8
5739 817.8
5740 817.8
5740.2 818
5741 818.8
5742 818.8
5743 818.8
5744 818.8
5745 818.8
5745.2 819
5746 819.8
5746.2 820
5747 820.8
5747.2 821
5748 821.8
5748.2 822
5749 822.8
5749.2 823
5749.2 824
5750 824.8
5750.2 825
5750.2 826
5751 826.8
5751.2 827
5751.2 828
5751.2 829
5751.2 830
5752 830.8
5752.2 831
5752.2 832
5752.2 833
5752.2 834
5752.2 835
5752.2 836
5752.2 837
5752.2 838
5752.2 839
5752.2 840
5752.2 841
5752 841.2
5751.2 842
5751.2 843
5751 843.2
5750.2 844
5750 844.2
5749.2 845
5749.2 846
5749 846.2
5748 846.2
5747.2 847
5747 847.2
5746.2 848
5746 848.2
5745 848.2
5744 848.2
5743 848.2
5742.2 849
5742 849.2
5741 849.2
5740 849.2
5739 849.2
5738.2 850
5738 850.2
5737 850.2
5736 850.2
5735 850.2
5734 850.2
5733 850.2
5732 850.2
5731 850.2
5730.2 851
5730 851.2
};
\addplot [red]
table {%
6227 885.2
6226 885.2
6225 885.2
6224 885.2
6223 885.2
6222 885.2
6221 885.2
6220 885.2
6219 885.2
6218 885.2
6217 885.2
6216 885.2
6215 885.2
6214 885.2
6213 885.2
6212 885.2
6211.8 885
6211 884.2
6210 884.2
6209 884.2
6208 884.2
6207 884.2
6206 884.2
6205.8 884
6205 883.2
6204 883.2
6203 883.2
6202.8 883
6202 882.2
6201 882.2
6200 882.2
6199.8 882
6199 881.2
6198 881.2
6197.8 881
6197 880.2
6196.8 880
6196 879.2
6195.8 879
6195 878.2
6194.8 878
6194 877.2
6193.8 877
6193 876.2
6192.8 876
6192 875.2
6191.8 875
6191.8 874
6191 873.2
6190.8 873
6190.8 872
6190.8 871
6190.8 870
6190.8 869
6190 868.2
6189.8 868
6189.8 867
6190 866.8
6190.8 866
6190.8 865
6190.8 864
6190.8 863
6190.8 862
6191 861.8
6191.8 861
6191.8 860
6191.8 859
6192 858.8
6192.8 858
6192.8 857
6193 856.8
6193.8 856
6193.8 855
6194 854.8
6194.8 854
6194.8 853
6195 852.8
6195.8 852
6196 851.8
6196.8 851
6196.8 850
6197 849.8
6197.8 849
6198 848.8
6198.8 848
6199 847.8
6199.8 847
6200 846.8
6200.8 846
6200.8 845
6201 844.8
6201.8 844
6202 843.8
6202.8 843
6203 842.8
6203.8 842
6203.8 841
6203.8 840
6204 839.8
6204.8 839
6204.8 838
6205 837.8
6205.8 837
6205.8 836
6206 835.8
6206.8 835
6206.8 834
6206.8 833
6207 832.8
6207.8 832
6207.8 831
6208 830.8
6208.8 830
6209 829.8
6209.8 829
6210 828.8
6210.8 828
6210.8 827
6211 826.8
6211.8 826
6212 825.8
6212.8 825
6213 824.8
6213.8 824
6214 823.8
6214.8 823
6215 822.8
6215.8 822
6216 821.8
6216.8 821
6217 820.8
6217.8 820
6218 819.8
6219 819.8
6219.8 819
6220 818.8
6221 818.8
6221.8 818
6222 817.8
6223 817.8
6224 817.8
6225 817.8
6225.8 817
6226 816.8
6227 816.8
6228 816.8
6229 816.8
6230 816.8
6230.2 817
6231 817.8
6232 817.8
6233 817.8
6233.2 818
6234 818.8
6235 818.8
6236 818.8
6236.2 819
6237 819.8
6238 819.8
6238.2 820
6239 820.8
6240 820.8
6240.2 821
6241 821.8
6242 821.8
6242.2 822
6243 822.8
6244 822.8
6244.2 823
6245 823.8
6245.2 824
6246 824.8
6246.2 825
6247 825.8
6248 825.8
6248.2 826
6249 826.8
6249.2 827
6250 827.8
6250.2 828
6251 828.8
6251.2 829
6252 829.8
6252.2 830
6252.2 831
6253 831.8
6253.2 832
6254 832.8
6254.2 833
6255 833.8
6255.2 834
6255.2 835
6256 835.8
6256.2 836
6257 836.8
6257.2 837
6258 837.8
6258.2 838
6258.2 839
6259 839.8
6259.2 840
6260 840.8
6260.2 841
6260.2 842
6261 842.8
6261.2 843
6261.2 844
6262 844.8
6262.2 845
6262.2 846
6263 846.8
6263.2 847
6263.2 848
6264 848.8
6264.2 849
6264.2 850
6265 850.8
6265.2 851
6265.2 852
6265.2 853
6265.2 854
6265.2 855
6265.2 856
6265.2 857
6266 857.8
6266.2 858
6266.2 859
6266.2 860
6266 860.2
6265.2 861
6265.2 862
6265.2 863
6265.2 864
6265 864.2
6264.2 865
6264.2 866
6264.2 867
6264.2 868
6264 868.2
6263.2 869
6263.2 870
6263 870.2
6262.2 871
6262.2 872
6262 872.2
6261.2 873
6261.2 874
6261 874.2
6260.2 875
6260 875.2
6259.2 876
6259 876.2
6258.2 877
6258 877.2
6257.2 878
6257 878.2
6256 878.2
6255.2 879
6255 879.2
6254 879.2
6253.2 880
6253 880.2
6252 880.2
6251 880.2
6250.2 881
6250 881.2
6249 881.2
6248 881.2
6247.2 882
6247 882.2
6246 882.2
6245 882.2
6244.2 883
6244 883.2
6243 883.2
6242 883.2
6241 883.2
6240 883.2
6239.2 884
6239 884.2
6238 884.2
6237 884.2
6236 884.2
6235 884.2
6234 884.2
6233 884.2
6232 884.2
6231 884.2
6230 884.2
6229 884.2
6228 884.2
6227.2 885
6227 885.2
};
\addplot [red]
table {%
3141 872.2
3140 872.2
3139 872.2
3138 872.2
3137 872.2
3136 872.2
3135.8 872
3135 871.2
3134 871.2
3133 871.2
3132 871.2
3131.8 871
3131 870.2
3130 870.2
3129 870.2
3128 870.2
3127.8 870
3127 869.2
3126 869.2
3125 869.2
3124 869.2
3123 869.2
3122 869.2
3121.8 869
3121 868.2
3120 868.2
3119.8 868
3119 867.2
3118 867.2
3117 867.2
3116.8 867
3116 866.2
3115 866.2
3114.8 866
3114.8 865
3114.8 864
3114.8 863
3114 862.2
3113.8 862
3113.8 861
3113.8 860
3114 859.8
3114.8 859
3114.8 858
3114.8 857
3115 856.8
3115.8 856
3115.8 855
3116 854.8
3116.8 854
3117 853.8
3117.8 853
3118 852.8
3118.8 852
3118.8 851
3119 850.8
3119.8 850
3120 849.8
3120.8 849
3121 848.8
3121.8 848
3122 847.8
3122.8 847
3123 846.8
3123.8 846
3124 845.8
3124.8 845
3125 844.8
3125.8 844
3125.8 843
3126 842.8
3126.8 842
3127 841.8
3128 841.8
3128.8 841
3129 840.8
3129.8 840
3130 839.8
3130.8 839
3131 838.8
3131.8 838
3132 837.8
3133 837.8
3133.8 837
3134 836.8
3135 836.8
3135.8 836
3136 835.8
3137 835.8
3138 835.8
3139 835.8
3139.8 835
3140 834.8
3141 834.8
3142 834.8
3143 834.8
3144 834.8
3144.8 834
3145 833.8
3146 833.8
3147 833.8
3147.8 833
3148 832.8
3149 832.8
3150 832.8
3150.8 832
3151 831.8
3152 831.8
3153 831.8
3153.8 831
3154 830.8
3155 830.8
3155.8 830
3156 829.8
3157 829.8
3157.8 829
3158 828.8
3159 828.8
3159.8 828
3160 827.8
3161 827.8
3162 827.8
3162.8 827
3163 826.8
3164 826.8
3164.8 826
3165 825.8
3166 825.8
3166.8 825
3167 824.8
3168 824.8
3169 824.8
3169.8 824
3170 823.8
3171 823.8
3171.8 823
3172 822.8
3173 822.8
3174 822.8
3174.8 822
3175 821.8
3176 821.8
3176.8 821
3177 820.8
3178 820.8
3179 820.8
3180 820.8
3181 820.8
3182 820.8
3183 820.8
3184 820.8
3185 820.8
3186 820.8
3187 820.8
3188 820.8
3189 820.8
3189.8 820
3190 819.8
3191 819.8
3192 819.8
3193 819.8
3194 819.8
3195 819.8
3196 819.8
3196.2 820
3197 820.8
3198 820.8
3198.2 821
3199 821.8
3200 821.8
3200.2 822
3201 822.8
3202 822.8
3202.2 823
3203 823.8
3204 823.8
3204.2 824
3205 824.8
3206 824.8
3206.2 825
3207 825.8
3208 825.8
3209 825.8
3209.2 826
3210 826.8
3210.2 827
3211 827.8
3211.2 828
3211.2 829
3212 829.8
3212.2 830
3213 830.8
3213.2 831
3213.2 832
3214 832.8
3214.2 833
3215 833.8
3215.2 834
3215.2 835
3216 835.8
3216.2 836
3217 836.8
3217.2 837
3217.2 838
3218 838.8
3218.2 839
3218 839.2
3217.2 840
3217 840.2
3216.2 841
3216 841.2
3215.2 842
3215 842.2
3214.2 843
3214 843.2
3213.2 844
3213 844.2
3212.2 845
3212 845.2
3211.2 846
3211 846.2
3210.2 847
3210 847.2
3209.2 848
3209 848.2
3208.2 849
3208 849.2
3207.2 850
3207 850.2
3206 850.2
3205.2 851
3205 851.2
3204.2 852
3204 852.2
3203.2 853
3203 853.2
3202.2 854
3202 854.2
3201 854.2
3200 854.2
3199 854.2
3198.2 855
3198 855.2
3197 855.2
3196 855.2
3195 855.2
3194.2 856
3194 856.2
3193 856.2
3192 856.2
3191 856.2
3190 856.2
3189.2 857
3189 857.2
3188 857.2
3187 857.2
3186 857.2
3185.2 858
3185 858.2
3184 858.2
3183 858.2
3182 858.2
3181.2 859
3181 859.2
3180 859.2
3179 859.2
3178 859.2
3177.2 860
3177 860.2
3176 860.2
3175 860.2
3174 860.2
3173 860.2
3172.2 861
3172 861.2
3171 861.2
3170 861.2
3169 861.2
3168 861.2
3167 861.2
3166.2 862
3166 862.2
3165 862.2
3164 862.2
3163.2 863
3163 863.2
3162.2 864
3162 864.2
3161.2 865
3161 865.2
3160 865.2
3159.2 866
3159 866.2
3158.2 867
3158 867.2
3157 867.2
3156.2 868
3156 868.2
3155 868.2
3154 868.2
3153.2 869
3153 869.2
3152 869.2
3151.2 870
3151 870.2
3150 870.2
3149 870.2
3148 870.2
3147 870.2
3146.2 871
3146 871.2
3145 871.2
3144 871.2
3143 871.2
3142 871.2
3141.2 872
3141 872.2
};
\addplot [red]
table {%
5622 889.2
5621.8 889
5621 888.2
5620 888.2
5619 888.2
5618.2 889
5618 889.2
5617 889.2
5616 889.2
5615 889.2
5614 889.2
5613 889.2
5612 889.2
5611 889.2
5610 889.2
5609 889.2
5608 889.2
5607 889.2
5606 889.2
5605 889.2
5604 889.2
5603 889.2
5602 889.2
5601 889.2
5600 889.2
5599 889.2
5598 889.2
5597 889.2
5596 889.2
5595 889.2
5594.8 889
5594 888.2
5593 888.2
5592 888.2
5591 888.2
5590 888.2
5589 888.2
5588.8 888
5588 887.2
5587 887.2
5586 887.2
5585 887.2
5584 887.2
5583 887.2
5582.8 887
5582 886.2
5581 886.2
5580 886.2
5579 886.2
5578 886.2
5577 886.2
5576 886.2
5575 886.2
5574 886.2
5573 886.2
5572 886.2
5571 886.2
5570 886.2
5569 886.2
5568 886.2
5567.8 886
5567 885.2
5566 885.2
5565.8 885
5565 884.2
5564 884.2
5563.8 884
5563 883.2
5562 883.2
5561.8 883
5561 882.2
5560.8 882
5561 881.8
5561.8 881
5561.8 880
5561.8 879
5561.8 878
5561 877.2
5560.2 878
5560 878.2
5559.2 879
5559 879.2
5558.2 880
5558 880.2
5557 880.2
5556.2 881
5556 881.2
5555 881.2
5554.2 882
5554 882.2
5553 882.2
5552.2 883
5552 883.2
5551 883.2
5550.2 884
5550 884.2
5549.2 885
5549 885.2
5548 885.2
5547 885.2
5546.2 886
5546 886.2
5545 886.2
5544.2 887
5544 887.2
5543 887.2
5542 887.2
5541.2 888
5541 888.2
5540 888.2
5539 888.2
5538 888.2
5537 888.2
5536 888.2
5535 888.2
5534 888.2
5533.8 888
5533 887.2
5532 887.2
5531 887.2
5530.8 887
5530 886.2
5529 886.2
5528.8 886
5528 885.2
5527 885.2
5526.8 885
5526 884.2
5525.8 884
5525 883.2
5524.8 883
5524 882.2
5523.8 882
5523 881.2
5522.8 881
5522 880.2
5521.8 880
5521 879.2
5520.8 879
5520.8 878
5520 877.2
5519.8 877
5519 876.2
5518.8 876
5518 875.2
5517.8 875
5517.8 874
5517 873.2
5516.8 873
5516.8 872
5516 871.2
5515.8 871
5515.8 870
5515 869.2
5514.8 869
5514 868.2
5513.8 868
5513.8 867
5513.8 866
5513 865.2
5512.8 865
5512.8 864
5512.8 863
5512 862.2
5511.8 862
5511.8 861
5511.8 860
5511.8 859
5511.8 858
5511.8 857
5511.8 856
5511.8 855
5512 854.8
5512.8 854
5512.8 853
5512.8 852
5513 851.8
5513.8 851
5513.8 850
5514 849.8
5514.8 849
5514.8 848
5515 847.8
5515.8 847
5515.8 846
5516 845.8
5516.8 845
5517 844.8
5517.8 844
5518 843.8
5518.8 843
5518.8 842
5519 841.8
5520 841.8
5520.8 841
5521 840.8
5521.8 840
5522 839.8
5522.8 839
5523 838.8
5523.8 838
5524 837.8
5524.8 837
5525 836.8
5526 836.8
5526.8 836
5527 835.8
5528 835.8
5528.8 835
5529 834.8
5530 834.8
5530.8 834
5531 833.8
5532 833.8
5532.8 833
5533 832.8
5534 832.8
5535 832.8
5535.8 832
5536 831.8
5537 831.8
5538 831.8
5538.8 831
5539 830.8
5540 830.8
5541 830.8
5542 830.8
5543 830.8
5543.8 830
5544 829.8
5545 829.8
5546 829.8
5546.8 829
5547 828.8
5548 828.8
5549 828.8
5550 828.8
5550.8 828
5551 827.8
5552 827.8
5553 827.8
5554 827.8
5555 827.8
5555.8 827
5556 826.8
5557 826.8
5558 826.8
5559 826.8
5560 826.8
5560.8 826
5561 825.8
5562 825.8
5563 825.8
5564 825.8
5565 825.8
5565.8 825
5566 824.8
5567 824.8
5568 824.8
5569 824.8
5570 824.8
5571 824.8
5571.8 824
5572 823.8
5573 823.8
5574 823.8
5575 823.8
5576 823.8
5576.8 823
5577 822.8
5578 822.8
5579 822.8
5580 822.8
5581 822.8
5582 822.8
5583 822.8
5583.8 822
5584 821.8
5585 821.8
5586 821.8
5587 821.8
5588 821.8
5589 821.8
5590 821.8
5590.8 821
5591 820.8
5592 820.8
5593 820.8
5594 820.8
5595 820.8
5596 820.8
5597 820.8
5597.8 820
5598 819.8
5599 819.8
5600 819.8
5601 819.8
5601.2 820
5602 820.8
5603 820.8
5604 820.8
5605 820.8
5606 820.8
5607 820.8
5608 820.8
5608.2 821
5609 821.8
5610 821.8
5611 821.8
5612 821.8
5613 821.8
5614 821.8
5615 821.8
5615.2 822
5616 822.8
5617 822.8
5618 822.8
5619 822.8
5620 822.8
5621 822.8
5622 822.8
5622.2 823
5623 823.8
5623.2 824
5623.2 825
5624 825.8
5624.2 826
5624.2 827
5625 827.8
5625.2 828
5625.2 829
5626 829.8
5626.2 830
5626.2 831
5627 831.8
5627.2 832
5627.2 833
5628 833.8
5628.2 834
5628.2 835
5629 835.8
5629.2 836
5629.2 837
5629.2 838
5629 838.2
5628 838.2
5627.2 839
5627 839.2
5626.2 840
5626 840.2
5625 840.2
5624.2 841
5624 841.2
5623 841.2
5622.2 842
5622 842.2
5621.2 843
5621 843.2
5620 843.2
5619.2 844
5619 844.2
5618 844.2
5617.2 845
5617 845.2
5616.2 846
5616 846.2
5615 846.2
5614.2 847
5614 847.2
5613 847.2
5612.2 848
5612 848.2
5611.2 849
5611 849.2
5610 849.2
5609.2 850
5609 850.2
5608 850.2
5607.2 851
5607 851.2
5606.2 852
5606 852.2
5605 852.2
5604.2 853
5604 853.2
5603 853.2
5602.2 854
5602 854.2
5601 854.2
5600 854.2
5599 854.2
5598.2 855
5598 855.2
5597 855.2
5596 855.2
5595.2 856
5595 856.2
5594 856.2
5593 856.2
5592 856.2
5591.2 857
5591 857.2
5590 857.2
5589 857.2
5588 857.2
5587.2 858
5588 858.8
5589 858.8
5590 858.8
5591 858.8
5592 858.8
5593 858.8
5594 858.8
5595 858.8
5595.8 858
5596 857.8
5597 857.8
5598 857.8
5598.2 858
5599 858.8
5600 858.8
5601 858.8
5602 858.8
5603 858.8
5604 858.8
5605 858.8
5606 858.8
5606.2 859
5607 859.8
5608 859.8
5609 859.8
5610 859.8
5611 859.8
5612 859.8
5613 859.8
5614 859.8
5614.8 859
5615 858.8
5616 858.8
5617 858.8
5617.8 858
5618 857.8
5619 857.8
5620 857.8
5621 857.8
5622 857.8
5623 857.8
5624 857.8
5625 857.8
5626 857.8
5627 857.8
5628 857.8
5629 857.8
5629.8 857
5630 856.8
5631 856.8
5631.8 856
5632 855.8
5632.8 855
5633 854.8
5633.8 854
5634 853.8
5634.8 853
5635 852.8
5636 852.8
5637 852.8
5638 852.8
5638.8 852
5639 851.8
5639.8 851
5640 850.8
5641 850.8
5641.8 850
5642 849.8
5643 849.8
5644 849.8
5645 849.8
5645.2 850
5645.2 851
5646 851.8
5646.2 852
5646.2 853
5647 853.8
5647.2 854
5647.2 855
5647.2 856
5647.2 857
5647.2 858
5647.2 859
5647.2 860
5647.2 861
5647.2 862
5647.2 863
5647 863.2
5646.2 864
5646.2 865
5646.2 866
5646.2 867
5646 867.2
5645.2 868
5645 868.2
5644 868.2
5643.2 869
5643 869.2
5642 869.2
5641 869.2
5640 869.2
5639.8 869
5639 868.2
5638 868.2
5637.2 869
5637 869.2
5636 869.2
5635 869.2
5634.8 869
5634 868.2
5633 868.2
5632 868.2
5631 868.2
5630 868.2
5629 868.2
5628 868.2
5627 868.2
5626 868.2
5625 868.2
5624.2 869
5624 869.2
5623 869.2
5622.8 869
5622 868.2
5621 868.2
5620 868.2
5619 868.2
5618 868.2
5617 868.2
5616 868.2
5615.2 869
5615.2 870
5615.2 871
5615 871.2
5614.2 872
5614 872.2
5613.2 873
5614 873.8
5614.8 873
5615 872.8
5616 872.8
5617 872.8
5617.8 872
5618 871.8
5619 871.8
5620 871.8
5621 871.8
5621.8 871
5622 870.8
5623 870.8
5624 870.8
5625 870.8
5625.2 871
5626 871.8
5627 871.8
5628 871.8
5629 871.8
5630 871.8
5630.2 872
5631 872.8
5632 872.8
5633 872.8
5633.2 873
5634 873.8
5635 873.8
5636 873.8
5636.2 874
5637 874.8
5638 874.8
5638.2 875
5639 875.8
5640 875.8
5640.2 876
5641 876.8
5642 876.8
5643 876.8
5643.2 877
5644 877.8
5645 877.8
5645.2 878
5646 878.8
5647 878.8
5647.2 879
5648 879.8
5649 879.8
5649.2 880
5649.2 881
5649 881.2
5648.2 882
5648.2 883
5648.2 884
5648.2 885
5648 885.2
5647 885.2
5646 885.2
5645 885.2
5644.2 886
5644 886.2
5643 886.2
5642 886.2
5641 886.2
5640.2 887
5640 887.2
5639 887.2
5638 887.2
5637 887.2
5636 887.2
5635 887.2
5634 887.2
5633 887.2
5632 887.2
5631 887.2
5630 887.2
5629 887.2
5628 887.2
5627.2 888
5627 888.2
5626 888.2
5625 888.2
5624 888.2
5623 888.2
5622.2 889
5622 889.2
};
\addplot [red]
table {%
4280 890.2
4279 890.2
4278.8 890
4278 889.2
4277 889.2
4276 889.2
4275 889.2
4274 889.2
4273 889.2
4272 889.2
4271 889.2
4270 889.2
4269 889.2
4268.8 889
4268 888.2
4267 888.2
4266 888.2
4265 888.2
4264 888.2
4263 888.2
4262.8 888
4262 887.2
4261 887.2
4260 887.2
4259 887.2
4258 887.2
4257 887.2
4256 887.2
4255.8 887
4255 886.2
4254 886.2
4253 886.2
4252 886.2
4251 886.2
4250 886.2
4249.8 886
4249 885.2
4248 885.2
4247 885.2
4246 885.2
4245 885.2
4244 885.2
4243.8 885
4243 884.2
4242 884.2
4241.8 884
4241 883.2
4240 883.2
4239.8 883
4239 882.2
4238 882.2
4237.8 882
4237 881.2
4236 881.2
4235.8 881
4235 880.2
4234 880.2
4233.8 880
4233 879.2
4232 879.2
4231.8 879
4231 878.2
4230 878.2
4229.8 878
4230 877.8
4230.8 877
4230.8 876
4230.8 875
4231 874.8
4231.8 874
4231.8 873
4231.8 872
4232 871.8
4232.8 871
4232.8 870
4232.8 869
4233 868.8
4234 868.8
4235 868.8
4235.8 868
4236 867.8
4237 867.8
4238 867.8
4239 867.8
4240 867.8
4240.8 867
4241 866.8
4242 866.8
4243 866.8
4244 866.8
4245 866.8
4245.8 866
4246 865.8
4247 865.8
4248 865.8
4249 865.8
4250 865.8
4251 865.8
4252 865.8
4253 865.8
4254 865.8
4255 865.8
4256 865.8
4257 865.8
4258 865.8
4259 865.8
4260 865.8
4261 865.8
4262 865.8
4263 865.8
4264 865.8
4265 865.8
4266 865.8
4267 865.8
4268 865.8
4269 865.8
4270 865.8
4271 865.8
4272 865.8
4272.2 866
4273 866.8
4274 866.8
4275 866.8
4276 866.8
4277 866.8
4278 866.8
4279 866.8
4280 866.8
4281 866.8
4281.8 866
4281 865.2
4280 865.2
4279 865.2
4278 865.2
4277 865.2
4276 865.2
4275 865.2
4274 865.2
4273 865.2
4272.8 865
4272 864.2
4271 864.2
4270 864.2
4269 864.2
4268 864.2
4267 864.2
4266 864.2
4265 864.2
4264 864.2
4263 864.2
4262.8 864
4262 863.2
4261 863.2
4260 863.2
4259 863.2
4258.8 863
4258 862.2
4257 862.2
4256 862.2
4255.8 862
4255 861.2
4254.8 861
4254.8 860
4254 859.2
4253.8 859
4253.8 858
4254 857.8
4254.8 857
4254.8 856
4255 855.8
4255.8 855
4256 854.8
4256.8 854
4257 853.8
4257.8 853
4258 852.8
4259 852.8
4259.8 852
4260 851.8
4261 851.8
4261.8 851
4262 850.8
4262.8 850
4263 849.8
4263.8 849
4264 848.8
4265 848.8
4266 848.8
4266.8 848
4267 847.8
4267.8 847
4267.8 846
4267.8 845
4267 844.2
4266.8 844
4266.8 843
4267 842.8
4267.8 842
4267.8 841
4267.8 840
4267.8 839
4268 838.8
4268.8 838
4269 837.8
4269.8 837
4270 836.8
4270.8 836
4271 835.8
4271.8 835
4272 834.8
4273 834.8
4273.8 834
4274 833.8
4275 833.8
4275.8 833
4276 832.8
4277 832.8
4277.8 832
4278 831.8
4279 831.8
4279.8 831
4280 830.8
4281 830.8
4282 830.8
4283 830.8
4283.2 831
4284 831.8
4285 831.8
4286 831.8
4287 831.8
4288 831.8
4289 831.8
4290 831.8
4291 831.8
4292 831.8
4292.2 832
4293 832.8
4294 832.8
4295 832.8
4296 832.8
4297 832.8
4297.2 833
4298 833.8
4299 833.8
4300 833.8
4300.2 834
4301 834.8
4301.2 835
4302 835.8
4302.2 836
4303 836.8
4303.2 837
4303 837.2
4302.2 838
4302.2 839
4302.2 840
4302 840.2
4301.2 841
4301 841.2
4300.2 842
4300 842.2
4299.2 843
4299 843.2
4298.2 844
4298 844.2
4297 844.2
4296.2 845
4296 845.2
4295.2 846
4295 846.2
4294 846.2
4293.2 847
4293 847.2
4292 847.2
4291 847.2
4290 847.2
4289.2 848
4290 848.8
4291 848.8
4292 848.8
4292.2 849
4293 849.8
4294 849.8
4295 849.8
4295.2 850
4296 850.8
4297 850.8
4298 850.8
4298.2 851
4299 851.8
4300 851.8
4301 851.8
4301.2 852
4302 852.8
4302.8 852
4303 851.8
4303.8 851
4304 850.8
4304.8 850
4305 849.8
4306 849.8
4306.8 849
4307 848.8
4307.8 848
4308 847.8
4308.8 847
4309 846.8
4310 846.8
4311 846.8
4312 846.8
4313 846.8
4314 846.8
4315 846.8
4316 846.8
4317 846.8
4317.8 846
4318 845.8
4319 845.8
4320 845.8
4321 845.8
4322 845.8
4323 845.8
4324 845.8
4325 845.8
4326 845.8
4327 845.8
4328 845.8
4329 845.8
4330 845.8
4331 845.8
4332 845.8
4332.2 846
4333 846.8
4334 846.8
4335 846.8
4336 846.8
4337 846.8
4338 846.8
4339 846.8
4340 846.8
4341 846.8
4342 846.8
4343 846.8
4344 846.8
4345 846.8
4346 846.8
4346.2 847
4347 847.8
4348 847.8
4349 847.8
4350 847.8
4350.2 848
4351 848.8
4352 848.8
4353 848.8
4354 848.8
4354.2 849
4355 849.8
4356 849.8
4356.2 850
4357 850.8
4358 850.8
4359 850.8
4359.2 851
4360 851.8
4361 851.8
4361.2 852
4361 852.2
4360.2 853
4360.2 854
4360 854.2
4359.2 855
4359 855.2
4358.2 856
4358 856.2
4357 856.2
4356 856.2
4355 856.2
4354.2 857
4354 857.2
4353 857.2
4352 857.2
4351 857.2
4350 857.2
4349 857.2
4348 857.2
4347 857.2
4346 857.2
4345 857.2
4344 857.2
4343.2 858
4343 858.2
4342 858.2
4341 858.2
4340 858.2
4339.8 858
4339 857.2
4338 857.2
4337 857.2
4336 857.2
4335 857.2
4334 857.2
4333 857.2
4332 857.2
4331 857.2
4330 857.2
4329.8 857
4329 856.2
4328 856.2
4327 856.2
4326 856.2
4325 856.2
4324 856.2
4323 856.2
4322 856.2
4321 856.2
4320 856.2
4319 856.2
4318 856.2
4317 856.2
4316 856.2
4315 856.2
4314 856.2
4313.2 857
4313.2 858
4313.2 859
4313 859.2
4312.2 860
4312.2 861
4312.2 862
4312 862.2
4311.2 863
4311.2 864
4311 864.2
4310 864.2
4309 864.2
4308 864.2
4307.2 865
4307 865.2
4306 865.2
4305 865.2
4304 865.2
4303.2 866
4303 866.2
4302 866.2
4301 866.2
4300 866.2
4299 866.2
4298.2 867
4298 867.2
4297 867.2
4296 867.2
4295 867.2
4294.8 867
4294 866.2
4293 866.2
4292 866.2
4291 866.2
4290 866.2
4289 866.2
4288 866.2
4287 866.2
4286 866.2
4285.2 867
4286 867.8
4287 867.8
4288 867.8
4289 867.8
4290 867.8
4291 867.8
4291.2 868
4292 868.8
4293 868.8
4294 868.8
4294.2 869
4295 869.8
4296 869.8
4297 869.8
4298 869.8
4299 869.8
4299.2 870
4300 870.8
4301 870.8
4302 870.8
4303 870.8
4304 870.8
4305 870.8
4306 870.8
4307 870.8
4308 870.8
4309 870.8
4310 870.8
4310.2 871
4311 871.8
4312 871.8
4313 871.8
4314 871.8
4315 871.8
4315.2 872
4316 872.8
4316.2 873
4317 873.8
4318 873.8
4318.2 874
4319 874.8
4320 874.8
4320.2 875
4321 875.8
4322 875.8
4322.2 876
4323 876.8
4323.2 877
4324 877.8
4325 877.8
4325.2 878
4325 878.2
4324.2 879
4324.2 880
4324 880.2
4323.2 881
4323 881.2
4322.2 882
4322 882.2
4321.2 883
4321 883.2
4320.2 884
4320.2 885
4320 885.2
4319 885.2
4318.2 886
4318 886.2
4317 886.2
4316 886.2
4315 886.2
4314.2 887
4314 887.2
4313 887.2
4312 887.2
4311 887.2
4310.2 888
4310 888.2
4309 888.2
4308 888.2
4307 888.2
4306 888.2
4305.2 889
4305 889.2
4304 889.2
4303 889.2
4302 889.2
4301 889.2
4300 889.2
4299 889.2
4298 889.2
4297 889.2
4296 889.2
4295 889.2
4294 889.2
4293 889.2
4292 889.2
4291 889.2
4290 889.2
4289 889.2
4288 889.2
4287 889.2
4286 889.2
4285 889.2
4284 889.2
4283 889.2
4282 889.2
4281 889.2
4280.2 890
4280 890.2
};
\addplot [red]
table {%
5857 847.2
5856 847.2
5855.8 847
5855 846.2
5854 846.2
5853 846.2
5852.8 846
5852 845.2
5851 845.2
5850.8 845
5850.8 844
5850 843.2
5849.8 843
5849 842.2
5848.8 842
5848 841.2
5847.8 841
5847.8 840
5847 839.2
5846.8 839
5846.8 838
5846.8 837
5846.8 836
5847 835.8
5847.8 835
5848 834.8
5849 834.8
5850 834.8
5851 834.8
5852 834.8
5853 834.8
5854 834.8
5855 834.8
5855.2 835
5856 835.8
5857 835.8
5857.2 836
5858 836.8
5858.2 837
5859 837.8
5859.2 838
5859.2 839
5860 839.8
5860.2 840
5860.2 841
5860.2 842
5860.2 843
5860.2 844
5860 844.2
5859.2 845
5859 845.2
5858.2 846
5858 846.2
5857.2 847
5857 847.2
};
\addplot [red]
table {%
4150 875.2
4149 875.2
4148 875.2
4147 875.2
4146 875.2
4145 875.2
4144 875.2
4143 875.2
4142 875.2
4141 875.2
4140 875.2
4139 875.2
4138 875.2
4137 875.2
4136 875.2
4135 875.2
4134.8 875
4134 874.2
4133 874.2
4132 874.2
4131.8 874
4131 873.2
4130 873.2
4129.8 873
4129 872.2
4128 872.2
4127.8 872
4127 871.2
4126.8 871
4126 870.2
4125.8 870
4125.8 869
4125 868.2
4124.8 868
4124.8 867
4124.8 866
4124 865.2
4123.8 865
4123.8 864
4123.8 863
4124 862.8
4124.8 862
4124.8 861
4125 860.8
4125.8 860
4126 859.8
4126.8 859
4127 858.8
4127.8 858
4128 857.8
4128.8 857
4129 856.8
4129.8 856
4130 855.8
4131 855.8
4131.8 855
4132 854.8
4133 854.8
4134 854.8
4135 854.8
4136 854.8
4136.8 854
4137 853.8
4138 853.8
4139 853.8
4140 853.8
4140.2 854
4141 854.8
4142 854.8
4142.8 854
4142 853.2
4141.8 853
4141 852.2
4140.8 852
4140 851.2
4139.8 851
4139 850.2
4138.8 850
4138.8 849
4138.8 848
4138.8 847
4138.8 846
4138.8 845
4138.8 844
4138.8 843
4138.8 842
4139 841.8
4139.8 841
4140 840.8
4140.8 840
4141 839.8
4141.8 839
4142 838.8
4143 838.8
4143.8 838
4144 837.8
4145 837.8
4145.8 837
4146 836.8
4147 836.8
4147.8 836
4148 835.8
4149 835.8
4150 835.8
4151 835.8
4152 835.8
4153 835.8
4154 835.8
4155 835.8
4156 835.8
4157 835.8
4158 835.8
4159 835.8
4159.2 836
4160 836.8
4161 836.8
4162 836.8
4163 836.8
4164 836.8
4165 836.8
4166 836.8
4167 836.8
4168 836.8
4169 836.8
4170 836.8
4171 836.8
4172 836.8
4173 836.8
4174 836.8
4174.2 837
4175 837.8
4176 837.8
4177 837.8
4178 837.8
4179 837.8
4180 837.8
4181 837.8
4182 837.8
4183 837.8
4184 837.8
4185 837.8
4186 837.8
4186.2 838
4187 838.8
4188 838.8
4189 838.8
4190 838.8
4190.2 839
4191 839.8
4192 839.8
4193 839.8
4193.2 840
4194 840.8
4195 840.8
4196 840.8
4196.2 841
4197 841.8
4198 841.8
4199 841.8
4199.2 842
4200 842.8
4201 842.8
4202 842.8
4202.2 843
4203 843.8
4204 843.8
4205 843.8
4205.2 844
4206 844.8
4207 844.8
4208 844.8
4208.2 845
4209 845.8
4210 845.8
4211 845.8
4211.2 846
4211 846.2
4210.2 847
4210 847.2
4209.2 848
4209 848.2
4208.2 849
4208 849.2
4207.2 850
4207 850.2
4206.2 851
4206.2 852
4206 852.2
4205.2 853
4205 853.2
4204.2 854
4204 854.2
4203.2 855
4203 855.2
4202.2 856
4202 856.2
4201.2 857
4201 857.2
4200 857.2
4199 857.2
4198 857.2
4197 857.2
4196 857.2
4195.2 858
4195 858.2
4194 858.2
4193 858.2
4192 858.2
4191 858.2
4190 858.2
4189.2 859
4189 859.2
4188 859.2
4187 859.2
4186 859.2
4185 859.2
4184 859.2
4183.2 860
4183.2 861
4183.2 862
4184 862.8
4184.2 863
4184.2 864
4184 864.2
4183.2 865
4183 865.2
4182 865.2
4181.2 866
4181 866.2
4180 866.2
4179.2 867
4179 867.2
4178 867.2
4177.2 868
4177 868.2
4176.2 869
4176 869.2
4175 869.2
4174.2 870
4174 870.2
4173 870.2
4172.2 871
4172 871.2
4171 871.2
4170 871.2
4169.2 872
4169 872.2
4168 872.2
4167 872.2
4166.2 873
4166 873.2
4165 873.2
4164 873.2
4163 873.2
4162.2 874
4162 874.2
4161 874.2
4160 874.2
4159 874.2
4158 874.2
4157 874.2
4156 874.2
4155 874.2
4154 874.2
4153 874.2
4152 874.2
4151 874.2
4150.2 875
4150 875.2
};
\addplot [red]
table {%
4106 873.2
4105 873.2
4104.8 873
4104 872.2
4103 872.2
4102 872.2
4101 872.2
4100 872.2
4099 872.2
4098.8 872
4098 871.2
4097 871.2
4096 871.2
4095 871.2
4094 871.2
4093 871.2
4092.8 871
4092 870.2
4091 870.2
4090 870.2
4089 870.2
4088 870.2
4087 870.2
4086.8 870
4086 869.2
4085 869.2
4084 869.2
4083.8 869
4083 868.2
4082 868.2
4081 868.2
4080.8 868
4080 867.2
4079 867.2
4078 867.2
4077.8 867
4077 866.2
4076 866.2
4075.8 866
4075 865.2
4074 865.2
4073.8 865
4073 864.2
4072 864.2
4071.8 864
4071 863.2
4070 863.2
4069.8 863
4069 862.2
4068 862.2
4067.8 862
4067 861.2
4066 861.2
4065.8 861
4065 860.2
4064 860.2
4063 860.2
4062.8 860
4062 859.2
4061 859.2
4060.8 859
4060 858.2
4059 858.2
4058 858.2
4057.8 858
4057 857.2
4056 857.2
4055.8 857
4055 856.2
4054 856.2
4053.8 856
4053 855.2
4052.8 855
4052 854.2
4051.8 854
4051 853.2
4050.8 853
4050 852.2
4049.8 852
4049 851.2
4048.8 851
4048.8 850
4048 849.2
4047.8 849
4047 848.2
4046.8 848
4046 847.2
4045.8 847
4045.8 846
4045.8 845
4046 844.8
4046.8 844
4047 843.8
4047.8 843
4048 842.8
4048.8 842
4048.8 841
4049 840.8
4049.8 840
4050 839.8
4050.8 839
4051 838.8
4051.8 838
4052 837.8
4053 837.8
4054 837.8
4055 837.8
4056 837.8
4057 837.8
4058 837.8
4058.8 837
4059 836.8
4060 836.8
4061 836.8
4062 836.8
4063 836.8
4064 836.8
4065 836.8
4066 836.8
4067 836.8
4068 836.8
4069 836.8
4070 836.8
4071 836.8
4072 836.8
4073 836.8
4074 836.8
4075 836.8
4076 836.8
4077 836.8
4077.2 837
4078 837.8
4079 837.8
4080 837.8
4081 837.8
4082 837.8
4083 837.8
4084 837.8
4084.2 838
4085 838.8
4086 838.8
4087 838.8
4087.2 839
4088 839.8
4089 839.8
4089.2 840
4090 840.8
4091 840.8
4091.2 841
4092 841.8
4093 841.8
4094 841.8
4095 841.8
4095.2 842
4096 842.8
4097 842.8
4098 842.8
4099 842.8
4100 842.8
4100.2 843
4101 843.8
4102 843.8
4102.2 844
4103 844.8
4104 844.8
4104.2 845
4105 845.8
4105.2 846
4106 846.8
4107 846.8
4107.2 847
4108 847.8
4108.2 848
4109 848.8
4109.2 849
4110 849.8
4110.2 850
4111 850.8
4111.2 851
4112 851.8
4112.2 852
4113 852.8
4114 852.8
4114.2 853
4115 853.8
4115.2 854
4116 854.8
4116.2 855
4117 855.8
4117.2 856
4118 856.8
4118.2 857
4119 857.8
4119.2 858
4120 858.8
4120.2 859
4121 859.8
4121.2 860
4121.2 861
4122 861.8
4122.2 862
4122.2 863
4122.2 864
4122.2 865
4122.2 866
4122.2 867
4122.2 868
4122 868.2
4121 868.2
4120.2 869
4120 869.2
4119.2 870
4119 870.2
4118 870.2
4117.2 871
4117 871.2
4116 871.2
4115 871.2
4114 871.2
4113 871.2
4112.2 872
4112 872.2
4111 872.2
4110 872.2
4109 872.2
4108 872.2
4107 872.2
4106.2 873
4106 873.2
};
\addplot [red]
table {%
4252 856.2
4251 856.2
4250 856.2
4249 856.2
4248 856.2
4247 856.2
4246.8 856
4246.8 855
4246 854.2
4245.8 854
4245 853.2
4244.8 853
4244 852.2
4243.8 852
4243.8 851
4243 850.2
4242.8 850
4242.8 849
4242.8 848
4242 847.2
4241.8 847
4241.8 846
4241.8 845
4242 844.8
4242.8 844
4243 843.8
4244 843.8
4244.8 843
4245 842.8
4246 842.8
4247 842.8
4247.8 842
4248 841.8
4249 841.8
4250 841.8
4251 841.8
4251.8 841
4252 840.8
4253 840.8
4254 840.8
4255 840.8
4256 840.8
4257 840.8
4258 840.8
4258.2 841
4259 841.8
4260 841.8
4260.2 842
4261 842.8
4262 842.8
4262.2 843
4262.2 844
4263 844.8
4263.2 845
4263.2 846
4263.2 847
4263 847.2
4262.2 848
4262 848.2
4261.2 849
4261 849.2
4260.2 850
4260 850.2
4259 850.2
4258.2 851
4258 851.2
4257.2 852
4257 852.2
4256.2 853
4256 853.2
4255.2 854
4255 854.2
4254.2 855
4254 855.2
4253 855.2
4252.2 856
4252 856.2
};
\addplot [red]
table {%
5932 868.2
5931.8 868
5931 867.2
5930 867.2
5929 867.2
5928.8 867
5928 866.2
5927.8 866
5927 865.2
5926.8 865
5926.8 864
5926.8 863
5926.8 862
5926.8 861
5926.8 860
5926 859.2
5925.8 859
5925.8 858
5925.8 857
5926 856.8
5926.8 856
5926.8 855
5926.8 854
5927 853.8
5927.8 853
5928 852.8
5928.8 852
5929 851.8
5930 851.8
5931 851.8
5932 851.8
5933 851.8
5934 851.8
5935 851.8
5936 851.8
5936.2 852
5937 852.8
5937.2 853
5938 853.8
5938.2 854
5938.2 855
5938.2 856
5939 856.8
5939.2 857
5939.2 858
5939.2 859
5939.2 860
5939.2 861
5939 861.2
5938.2 862
5938.2 863
5938 863.2
5937.2 864
5937.2 865
5937 865.2
5936.2 866
5936 866.2
5935 866.2
5934.2 867
5934 867.2
5933 867.2
5932.2 868
5932 868.2
};
\addplot [red]
table {%
4144.8 855
4144 854.2
4143.2 855
4144 855.8
4144.8 855
};
\addplot [red]
table {%
4146.8 856
4146 855.2
4145.2 856
4146 856.8
4146.8 856
};
\addplot [red]
table {%
5586.8 859
5586 858.2
5585 858.2
5584.2 859
5585 859.8
5586 859.8
5586.8 859
};
\addplot [red]
table {%
5583.8 860
5583 859.2
5582 859.2
5581 859.2
5580.2 860
5581 860.8
5582 860.8
5583 860.8
5583.8 860
};
\addplot [red]
table {%
2274 874.2
2273 874.2
2272 874.2
2271 874.2
2270.8 874
2270 873.2
2269 873.2
2268.8 873
2268 872.2
2267.8 872
2267.8 871
2267 870.2
2266.8 870
2266.8 869
2266.8 868
2266.8 867
2266.8 866
2266.8 865
2267 864.8
2267.8 864
2267.8 863
2268 862.8
2268.8 862
2269 861.8
2270 861.8
2271 861.8
2272 861.8
2272.8 861
2273 860.8
2273.2 861
2274 861.8
2275 861.8
2276 861.8
2277 861.8
2278 861.8
2278.2 862
2278.2 863
2278.2 864
2278.2 865
2278.2 866
2278 866.2
2277.2 867
2277 867.2
2276.2 868
2276.2 869
2276 869.2
2275.2 870
2275.2 871
2275.2 872
2275 872.2
2274.2 873
2274.2 874
2274 874.2
};
\addplot [red]
table {%
5579.8 861
5579 860.2
5578.2 861
5579 861.8
5579.8 861
};
\addplot [red]
table {%
3208 876.2
3207 876.2
3206 876.2
3205.8 876
3205 875.2
3204 875.2
3203 875.2
3202.2 876
3202 876.2
3201.8 876
3201 875.2
3200 875.2
3199 875.2
3198.2 876
3198 876.2
3197.8 876
3197 875.2
3196 875.2
3195 875.2
3194 875.2
3193.8 875
3193 874.2
3192 874.2
3191.8 874
3191 873.2
3190.8 873
3190 872.2
3189.8 872
3189.8 871
3189.8 870
3189.8 869
3190 868.8
3190.8 868
3191 867.8
3191.8 867
3192 866.8
3192.8 866
3193 865.8
3194 865.8
3194.8 865
3195 864.8
3195.2 865
3196 865.8
3197 865.8
3197.8 865
3198 864.8
3199 864.8
3199.2 865
3200 865.8
3201 865.8
3202 865.8
3203 865.8
3204 865.8
3205 865.8
3205.2 866
3206 866.8
3207 866.8
3208 866.8
3208.2 867
3209 867.8
3210 867.8
3210.2 868
3210.2 869
3211 869.8
3211.2 870
3211.2 871
3211.2 872
3211.2 873
3211.2 874
3211 874.2
3210.2 875
3210 875.2
3209 875.2
3208.2 876
3208 876.2
};
\addplot [red]
table {%
5694 888.2
5693 888.2
5692 888.2
5691 888.2
5690 888.2
5689 888.2
5688 888.2
5687 888.2
5686 888.2
5685 888.2
5684 888.2
5683 888.2
5682 888.2
5681.8 888
5681 887.2
5680 887.2
5679 887.2
5678 887.2
5677 887.2
5676 887.2
5675.8 887
5675 886.2
5674.8 886
5674 885.2
5673.8 885
5673 884.2
5672.8 884
5673 883.8
5674 883.8
5675 883.8
5675.8 883
5676 882.8
5677 882.8
5678 882.8
5678.8 882
5679 881.8
5680 881.8
5681 881.8
5682 881.8
5683 881.8
5684 881.8
5685 881.8
5686 881.8
5687 881.8
5688 881.8
5689 881.8
5690 881.8
5690.8 881
5691 880.8
5692 880.8
5692.8 880
5693 879.8
5694 879.8
5695 879.8
5696 879.8
5697 879.8
5697.8 879
5698 878.8
5699 878.8
5700 878.8
5701 878.8
5702 878.8
5703 878.8
5704 878.8
5705 878.8
5706 878.8
5706.2 879
5707 879.8
5708 879.8
5709 879.8
5710 879.8
5711 879.8
5711.2 880
5712 880.8
5712.2 881
5713 881.8
5713.2 882
5714 882.8
5714.2 883
5714.2 884
5714 884.2
5713 884.2
5712.2 885
5712 885.2
5711 885.2
5710 885.2
5709.2 886
5709 886.2
5708 886.2
5707 886.2
5706 886.2
5705 886.2
5704 886.2
5703 886.2
5702 886.2
5701 886.2
5700.2 887
5700 887.2
5699 887.2
5698 887.2
5697 887.2
5696 887.2
5695 887.2
5694.2 888
5694 888.2
};
\addplot [red]
table {%
4517 946.2
4516 946.2
4515 946.2
4514 946.2
4513 946.2
4512 946.2
4511 946.2
4510 946.2
4509 946.2
4508 946.2
4507 946.2
4506 946.2
4505 946.2
4504 946.2
4503 946.2
4502 946.2
4501 946.2
4500 946.2
4499 946.2
4498 946.2
4497 946.2
4496 946.2
4495 946.2
4494 946.2
4493 946.2
4492 946.2
4491 946.2
4490 946.2
4489 946.2
4488.8 946
4488 945.2
4487 945.2
4486 945.2
4485 945.2
4484 945.2
4483 945.2
4482 945.2
4481.8 945
4481 944.2
4480 944.2
4479 944.2
4478 944.2
4477.8 944
4477 943.2
4476 943.2
4475 943.2
4474 943.2
4473 943.2
4472.8 943
4472 942.2
4471 942.2
4470.8 942
4470 941.2
4469 941.2
4468.8 941
4468 940.2
4467 940.2
4466 940.2
4465.8 940
4465 939.2
4464 939.2
4463.8 939
4463 938.2
4462 938.2
4461.8 938
4461 937.2
4460 937.2
4459.8 937
4459 936.2
4458.8 936
4458.8 935
4458 934.2
4457.8 934
4457 933.2
4456.8 933
4456 932.2
4455.8 932
4455 931.2
4454.8 931
4454 930.2
4453.8 930
4453 929.2
4452.8 929
4452 928.2
4451.8 928
4451.8 927
4451.8 926
4451 925.2
4450.8 925
4450.8 924
4450.8 923
4450.8 922
4450.8 921
4450.8 920
4450 919.2
4449.8 919
4449.8 918
4449.8 917
4449.8 916
4449.8 915
4449.8 914
4449.8 913
4449.8 912
4450 911.8
4450.8 911
4450.8 910
4450.8 909
4450.8 908
4450.8 907
4451 906.8
4451.8 906
4451.8 905
4451.8 904
4452 903.8
4452.8 903
4453 902.8
4453.8 902
4454 901.8
4454.8 901
4455 900.8
4455.8 900
4455.8 899
4456 898.8
4456.8 898
4457 897.8
4457.8 897
4458 896.8
4458.8 896
4459 895.8
4459.8 895
4460 894.8
4461 894.8
4461.8 894
4462 893.8
4463 893.8
4463.8 893
4464 892.8
4465 892.8
4465.8 892
4466 891.8
4467 891.8
4467.8 891
4468 890.8
4469 890.8
4470 890.8
4470.8 890
4471 889.8
4472 889.8
4472.8 889
4473 888.8
4474 888.8
4475 888.8
4476 888.8
4476.8 888
4477 887.8
4478 887.8
4479 887.8
4479.8 887
4480 886.8
4481 886.8
4482 886.8
4483 886.8
4483.8 886
4484 885.8
4485 885.8
4486 885.8
4487 885.8
4488 885.8
4488.8 885
4489 884.8
4490 884.8
4491 884.8
4492 884.8
4493 884.8
4494 884.8
4494.8 884
4495 883.8
4496 883.8
4497 883.8
4498 883.8
4499 883.8
4500 883.8
4501 883.8
4502 883.8
4503 883.8
4504 883.8
4505 883.8
4506 883.8
4507 883.8
4508 883.8
4508.8 883
4509 882.8
4510 882.8
4511 882.8
4512 882.8
4513 882.8
4514 882.8
4515 882.8
4516 882.8
4517 882.8
4518 882.8
4519 882.8
4519.8 882
4520 881.8
4521 881.8
4522 881.8
4523 881.8
4523.8 881
4524 880.8
4525 880.8
4526 880.8
4527 880.8
4528 880.8
4529 880.8
4530 880.8
4531 880.8
4531.8 880
4532 879.8
4533 879.8
4534 879.8
4535 879.8
4536 879.8
4537 879.8
4538 879.8
4538.2 880
4539 880.8
4540 880.8
4541 880.8
4542 880.8
4543 880.8
4544 880.8
4544.2 881
4545 881.8
4546 881.8
4547 881.8
4548 881.8
4549 881.8
4550 881.8
4551 881.8
4552 881.8
4553 881.8
4554 881.8
4555 881.8
4556 881.8
4557 881.8
4558 881.8
4559 881.8
4560 881.8
4561 881.8
4562 881.8
4563 881.8
4564 881.8
4565 881.8
4566 881.8
4567 881.8
4568 881.8
4568.2 882
4569 882.8
4570 882.8
4571 882.8
4572 882.8
4573 882.8
4574 882.8
4575 882.8
4576 882.8
4577 882.8
4578 882.8
4579 882.8
4580 882.8
4580.2 883
4581 883.8
4582 883.8
4583 883.8
4584 883.8
4585 883.8
4586 883.8
4587 883.8
4588 883.8
4589 883.8
4590 883.8
4590.2 884
4591 884.8
4592 884.8
4592.2 885
4593 885.8
4594 885.8
4595 885.8
4595.2 886
4596 886.8
4597 886.8
4597.2 887
4598 887.8
4599 887.8
4600 887.8
4600.2 888
4601 888.8
4602 888.8
4602.2 889
4603 889.8
4604 889.8
4604.2 890
4605 890.8
4606 890.8
4607 890.8
4607.2 891
4608 891.8
4609 891.8
4609.2 892
4610 892.8
4611 892.8
4612 892.8
4612.2 893
4613 893.8
4614 893.8
4614.2 894
4615 894.8
4616 894.8
4617 894.8
4617.2 895
4617.2 896
4617.2 897
4617.2 898
4617.2 899
4617.2 900
4617.2 901
4617.2 902
4617.2 903
4617.2 904
4617.2 905
4617.2 906
4617.2 907
4617.2 908
4617.2 909
4617.2 910
4617.2 911
4617.2 912
4617.2 913
4617.2 914
4617.2 915
4617.2 916
4617 916.2
4616 916.2
4615.2 917
4615 917.2
4614 917.2
4613.2 918
4613 918.2
4612 918.2
4611.2 919
4611 919.2
4610.2 920
4610 920.2
4609 920.2
4608.2 921
4608 921.2
4607 921.2
4606.2 922
4606 922.2
4605 922.2
4604.2 923
4604 923.2
4603 923.2
4602.2 924
4602 924.2
4601.2 925
4601 925.2
4600 925.2
4599.2 926
4599 926.2
4598 926.2
4597.2 927
4597 927.2
4596 927.2
4595.2 928
4595 928.2
4594 928.2
4593.2 929
4593 929.2
4592.2 930
4592 930.2
4591 930.2
4590.2 931
4590 931.2
4589 931.2
4588 931.2
4587.2 932
4587 932.2
4586 932.2
4585 932.2
4584.2 933
4584 933.2
4583 933.2
4582 933.2
4581.2 934
4581 934.2
4580 934.2
4579 934.2
4578.2 935
4578 935.2
4577 935.2
4576 935.2
4575 935.2
4574.2 936
4574 936.2
4573 936.2
4572 936.2
4571.2 937
4571 937.2
4570 937.2
4569 937.2
4568.2 938
4568 938.2
4567 938.2
4566 938.2
4565.2 939
4565 939.2
4564 939.2
4563 939.2
4562 939.2
4561.2 940
4561 940.2
4560 940.2
4559 940.2
4558.2 941
4558 941.2
4557 941.2
4556 941.2
4555 941.2
4554.2 942
4554 942.2
4553 942.2
4552 942.2
4551 942.2
4550.2 943
4550 943.2
4549 943.2
4548 943.2
4547 943.2
4546 943.2
4545 943.2
4544.2 944
4544 944.2
4543 944.2
4542 944.2
4541 944.2
4540 944.2
4539 944.2
4538 944.2
4537 944.2
4536.2 945
4536 945.2
4535 945.2
4534 945.2
4533 945.2
4532 945.2
4531 945.2
4530 945.2
4529 945.2
4528 945.2
4527 945.2
4526 945.2
4525 945.2
4524 945.2
4523 945.2
4522 945.2
4521 945.2
4520 945.2
4519 945.2
4518 945.2
4517.2 946
4517 946.2
};
\addplot [red]
table {%
2252 936.2
2251 936.2
2250 936.2
2249 936.2
2248 936.2
2247 936.2
2246.8 936
2246 935.2
2245 935.2
2244 935.2
2243 935.2
2242 935.2
2241 935.2
2240 935.2
2239 935.2
2238 935.2
2237 935.2
2236 935.2
2235 935.2
2234 935.2
2233 935.2
2232.8 935
2232 934.2
2231 934.2
2230 934.2
2229.8 934
2229 933.2
2228 933.2
2227.8 933
2227 932.2
2226 932.2
2225.8 932
2225 931.2
2224.8 931
2224 930.2
2223.8 930
2223 929.2
2222.8 929
2222.8 928
2222 927.2
2221.8 927
2221.8 926
2221 925.2
2220.8 925
2220.8 924
2220 923.2
2219.8 923
2219.8 922
2219.8 921
2219.8 920
2219.8 919
2219 918.2
2218.8 918
2218.8 917
2218.8 916
2218.8 915
2218.8 914
2218.8 913
2218.8 912
2219 911.8
2219.8 911
2219.8 910
2219.8 909
2219.8 908
2220 907.8
2220.8 907
2220.8 906
2220.8 905
2221 904.8
2221.8 904
2221.8 903
2221.8 902
2222 901.8
2222.8 901
2222.8 900
2222.8 899
2223 898.8
2223.8 898
2223.8 897
2224 896.8
2224.8 896
2224.8 895
2225 894.8
2225.8 894
2225.8 893
2226 892.8
2226.8 892
2227 891.8
2227.8 891
2227.8 890
2228 889.8
2228.8 889
2228.8 888
2229 887.8
2229.8 887
2230 886.8
2231 886.8
2231.8 886
2232 885.8
2232.8 885
2233 884.8
2233.8 884
2234 883.8
2235 883.8
2236 883.8
2236.8 883
2237 882.8
2238 882.8
2239 882.8
2240 882.8
2241 882.8
2242 882.8
2243 882.8
2244 882.8
2245 882.8
2246 882.8
2247 882.8
2248 882.8
2249 882.8
2250 882.8
2251 882.8
2252 882.8
2253 882.8
2253.2 883
2254 883.8
2255 883.8
2255.2 884
2256 884.8
2257 884.8
2257.2 885
2258 885.8
2258.2 886
2259 886.8
2259.2 887
2260 887.8
2260.2 888
2261 888.8
2261.2 889
2261.2 890
2262 890.8
2262.2 891
2262.2 892
2263 892.8
2263.2 893
2263.2 894
2264 894.8
2264.2 895
2264.2 896
2264.2 897
2264.2 898
2265 898.8
2265.2 899
2265.2 900
2265.2 901
2265.2 902
2265.2 903
2265.2 904
2266 904.8
2266.2 905
2266.2 906
2266.2 907
2266.2 908
2266.2 909
2266.2 910
2266.2 911
2266.2 912
2266 912.2
2265.2 913
2265.2 914
2265.2 915
2265.2 916
2265.2 917
2265 917.2
2264.2 918
2264.2 919
2264.2 920
2264.2 921
2264 921.2
2263.2 922
2263.2 923
2263.2 924
2263.2 925
2263 925.2
2262.2 926
2262.2 927
2262.2 928
2262 928.2
2261.2 929
2261.2 930
2261.2 931
2261.2 932
2261 932.2
2260 932.2
2259.2 933
2259 933.2
2258.2 934
2258 934.2
2257 934.2
2256.2 935
2256 935.2
2255 935.2
2254 935.2
2253 935.2
2252.2 936
2252 936.2
};
\addplot [red]
table {%
2213 898.2
2212 898.2
2211 898.2
2210.8 898
2210 897.2
2209 897.2
2208.8 897
2208 896.2
2207 896.2
2206.8 896
2206.8 895
2206 894.2
2205.8 894
2205 893.2
2204.8 893
2204.8 892
2204.8 891
2204.8 890
2204.8 889
2205 888.8
2205.8 888
2206 887.8
2206.8 887
2207 886.8
2207.8 886
2208 885.8
2209 885.8
2210 885.8
2210.8 885
2211 884.8
2212 884.8
2213 884.8
2214 884.8
2214.2 885
2215 885.8
2216 885.8
2216.2 886
2217 886.8
2217.2 887
2218 887.8
2218.2 888
2218.2 889
2218.2 890
2218.2 891
2218.2 892
2218 892.2
2217.2 893
2217.2 894
2217 894.2
2216.2 895
2216 895.2
2215.2 896
2215.2 897
2215 897.2
2214 897.2
2213.2 898
2213 898.2
};
\addplot [red]
table {%
4809 935.2
4808 935.2
4807 935.2
4806 935.2
4805 935.2
4804 935.2
4803 935.2
4802 935.2
4801 935.2
4800 935.2
4799 935.2
4798.8 935
4798 934.2
4797 934.2
4796 934.2
4795 934.2
4794 934.2
4793 934.2
4792 934.2
4791.8 934
4791 933.2
4790 933.2
4789 933.2
4788 933.2
4787 933.2
4786 933.2
4785 933.2
4784 933.2
4783 933.2
4782 933.2
4781 933.2
4780 933.2
4779 933.2
4778 933.2
4777 933.2
4776.2 934
4776 934.2
4775 934.2
4774 934.2
4773 934.2
4772 934.2
4771 934.2
4770 934.2
4769 934.2
4768 934.2
4767 934.2
4766 934.2
4765 934.2
4764 934.2
4763 934.2
4762 934.2
4761 934.2
4760 934.2
4759 934.2
4758 934.2
4757 934.2
4756 934.2
4755.8 934
4755 933.2
4754 933.2
4753 933.2
4752 933.2
4751.8 933
4751 932.2
4750 932.2
4749 932.2
4748 932.2
4747.8 932
4747 931.2
4746 931.2
4745 931.2
4744 931.2
4743.8 931
4743 930.2
4742 930.2
4741 930.2
4740 930.2
4739.8 930
4739 929.2
4738 929.2
4737 929.2
4736.8 929
4736 928.2
4735.8 928
4735 927.2
4734.8 927
4734 926.2
4733.8 926
4733.8 925
4733 924.2
4732.8 924
4732 923.2
4731.8 923
4731.8 922
4731 921.2
4730.8 921
4730 920.2
4729.8 920
4729 919.2
4728.8 919
4728.8 918
4728 917.2
4727.8 917
4727 916.2
4726.8 916
4727 915.8
4727.8 915
4728 914.8
4728.8 914
4728.8 913
4729 912.8
4729.8 912
4730 911.8
4730.8 911
4731 910.8
4731.8 910
4731.8 909
4732 908.8
4732.8 908
4733 907.8
4733.8 907
4734 906.8
4734.8 906
4734.8 905
4735 904.8
4735.8 904
4736 903.8
4736.8 903
4737 902.8
4738 902.8
4739 902.8
4739.8 902
4740 901.8
4741 901.8
4741.8 901
4742 900.8
4743 900.8
4744 900.8
4744.8 900
4745 899.8
4746 899.8
4746.8 899
4747 898.8
4748 898.8
4749 898.8
4749.8 898
4750 897.8
4751 897.8
4752 897.8
4752.8 897
4753 896.8
4754 896.8
4755 896.8
4756 896.8
4757 896.8
4758 896.8
4758.8 896
4759 895.8
4760 895.8
4761 895.8
4762 895.8
4763 895.8
4764 895.8
4765 895.8
4766 895.8
4767 895.8
4768 895.8
4769 895.8
4770 895.8
4771 895.8
4772 895.8
4773 895.8
4774 895.8
4775 895.8
4776 895.8
4777 895.8
4778 895.8
4779 895.8
4780 895.8
4781 895.8
4782 895.8
4783 895.8
4784 895.8
4785 895.8
4786 895.8
4786.8 895
4787 894.8
4788 894.8
4789 894.8
4790 894.8
4791 894.8
4792 894.8
4793 894.8
4794 894.8
4795 894.8
4796 894.8
4797 894.8
4798 894.8
4799 894.8
4800 894.8
4801 894.8
4802 894.8
4803 894.8
4804 894.8
4805 894.8
4806 894.8
4807 894.8
4808 894.8
4809 894.8
4810 894.8
4811 894.8
4812 894.8
4813 894.8
4814 894.8
4815 894.8
4816 894.8
4816.2 895
4817 895.8
4818 895.8
4819 895.8
4819.2 896
4820 896.8
4821 896.8
4821.2 897
4822 897.8
4823 897.8
4824 897.8
4824.2 898
4825 898.8
4826 898.8
4827 898.8
4827.2 899
4828 899.8
4829 899.8
4829.2 900
4830 900.8
4831 900.8
4831.2 901
4832 901.8
4833 901.8
4833.2 902
4834 902.8
4834.2 903
4835 903.8
4835.2 904
4835.2 905
4836 905.8
4836.2 906
4836.2 907
4837 907.8
4837.2 908
4837.2 909
4837.2 910
4837.2 911
4837.2 912
4837.2 913
4837.2 914
4837.2 915
4837.2 916
4837 916.2
4836.2 917
4836 917.2
4835.2 918
4835.2 919
4835 919.2
4834.2 920
4834 920.2
4833.2 921
4833.2 922
4833 922.2
4832.2 923
4832 923.2
4831.2 924
4831 924.2
4830 924.2
4829.2 925
4829 925.2
4828.2 926
4828 926.2
4827.2 927
4827 927.2
4826 927.2
4825.2 928
4825 928.2
4824.2 929
4824 929.2
4823 929.2
4822.2 930
4822 930.2
4821 930.2
4820.2 931
4820 931.2
4819 931.2
4818.2 932
4818 932.2
4817 932.2
4816 932.2
4815.2 933
4815 933.2
4814 933.2
4813 933.2
4812.2 934
4812 934.2
4811 934.2
4810 934.2
4809.2 935
4809 935.2
};
\addplot [red]
table {%
3911 959.2
3910 959.2
3909 959.2
3908 959.2
3907 959.2
3906 959.2
3905 959.2
3904 959.2
3903 959.2
3902 959.2
3901 959.2
3900 959.2
3899 959.2
3898 959.2
3897 959.2
3896.8 959
3896 958.2
3895 958.2
3894 958.2
3893.2 959
3893 959.2
3892 959.2
3891.8 959
3891 958.2
3890 958.2
3889 958.2
3888 958.2
3887 958.2
3886 958.2
3885.8 958
3885 957.2
3884 957.2
3883 957.2
3882 957.2
3881 957.2
3880 957.2
3879 957.2
3878 957.2
3877 957.2
3876 957.2
3875 957.2
3874 957.2
3873 957.2
3872 957.2
3871 957.2
3870 957.2
3869 957.2
3868 957.2
3867 957.2
3866 957.2
3865 957.2
3864.8 957
3864 956.2
3863 956.2
3862 956.2
3861 956.2
3860 956.2
3859 956.2
3858 956.2
3857 956.2
3856 956.2
3855 956.2
3854 956.2
3853.8 956
3853 955.2
3852 955.2
3851 955.2
3850 955.2
3849 955.2
3848.8 955
3848 954.2
3847.2 955
3847 955.2
3846 955.2
3845 955.2
3844 955.2
3843 955.2
3842 955.2
3841 955.2
3840 955.2
3839 955.2
3838 955.2
3837 955.2
3836 955.2
3835.8 955
3835 954.2
3834 954.2
3833.8 954
3833 953.2
3832 953.2
3831 953.2
3830 953.2
3829 953.2
3828.8 953
3828 952.2
3827 952.2
3826 952.2
3825 952.2
3824.8 952
3824 951.2
3823 951.2
3822.8 951
3822 950.2
3821 950.2
3820.8 950
3820 949.2
3819 949.2
3818.8 949
3818 948.2
3817 948.2
3816.8 948
3816 947.2
3815.8 947
3815 946.2
3814 946.2
3813.8 946
3813 945.2
3812.8 945
3812 944.2
3811 944.2
3810.8 944
3810 943.2
3809.8 943
3809 942.2
3808.8 942
3808 941.2
3807.8 941
3807 940.2
3806.8 940
3806 939.2
3805.8 939
3805.8 938
3805.8 937
3805 936.2
3804.8 936
3804.8 935
3804.8 934
3804.8 933
3804.8 932
3805 931.8
3805.8 931
3805.8 930
3806 929.8
3806.8 929
3807 928.8
3807.8 928
3808 927.8
3809 927.8
3809.8 927
3810 926.8
3811 926.8
3811.8 926
3812 925.8
3813 925.8
3814 925.8
3814.8 925
3815 924.8
3816 924.8
3817 924.8
3818 924.8
3819 924.8
3820 924.8
3821 924.8
3822 924.8
3823 924.8
3824 924.8
3825 924.8
3825.2 925
3826 925.8
3827 925.8
3828 925.8
3829 925.8
3830 925.8
3830.2 926
3831 926.8
3832 926.8
3832.2 927
3833 927.8
3834 927.8
3835 927.8
3836 927.8
3836.2 928
3837 928.8
3838 928.8
3838.2 929
3839 929.8
3840 929.8
3840.2 930
3841 930.8
3842 930.8
3842.2 931
3843 931.8
3844 931.8
3844.2 932
3845 932.8
3845.2 933
3846 933.8
3846.2 934
3847 934.8
3847.2 935
3848 935.8
3848.8 935
3849 934.8
3850 934.8
3851 934.8
3852 934.8
3852.8 934
3853 933.8
3854 933.8
3855 933.8
3855.8 933
3856 932.8
3857 932.8
3858 932.8
3858.8 932
3859 931.8
3860 931.8
3861 931.8
3862 931.8
3863 931.8
3864 931.8
3865 931.8
3866 931.8
3867 931.8
3867.8 931
3868 930.8
3869 930.8
3870 930.8
3871 930.8
3872 930.8
3873 930.8
3874 930.8
3875 930.8
3875.2 931
3876 931.8
3877 931.8
3878 931.8
3879 931.8
3880 931.8
3881 931.8
3882 931.8
3883 931.8
3884 931.8
3885 931.8
3886 931.8
3887 931.8
3888 931.8
3889 931.8
3890 931.8
3891 931.8
3892 931.8
3893 931.8
3894 931.8
3894.2 932
3895 932.8
3896 932.8
3897 932.8
3898 932.8
3899 932.8
3900 932.8
3901 932.8
3901.2 933
3902 933.8
3903 933.8
3904 933.8
3905 933.8
3906 933.8
3906.2 934
3907 934.8
3908 934.8
3908.2 935
3909 935.8
3910 935.8
3911 935.8
3911.2 936
3912 936.8
3913 936.8
3913.2 937
3914 937.8
3914.2 938
3915 938.8
3915.2 939
3916 939.8
3916.8 939
3916.8 938
3916.8 937
3916.8 936
3917 935.8
3917.8 935
3917.8 934
3918 933.8
3918.8 933
3918.8 932
3919 931.8
3920 931.8
3920.8 931
3921 930.8
3922 930.8
3922.8 930
3923 929.8
3924 929.8
3925 929.8
3926 929.8
3927 929.8
3928 929.8
3929 929.8
3930 929.8
3931 929.8
3932 929.8
3933 929.8
3934 929.8
3935 929.8
3936 929.8
3937 929.8
3938 929.8
3939 929.8
3940 929.8
3940.2 930
3941 930.8
3942 930.8
3943 930.8
3944 930.8
3945 930.8
3945.2 931
3946 931.8
3947 931.8
3948 931.8
3949 931.8
3949.2 932
3950 932.8
3951 932.8
3952 932.8
3952.2 933
3953 933.8
3954 933.8
3955 933.8
3955.2 934
3956 934.8
3956.2 935
3957 935.8
3957.2 936
3958 936.8
3958.2 937
3958.2 938
3959 938.8
3959.2 939
3959.2 940
3959.2 941
3959.2 942
3959.2 943
3959.2 944
3959.2 945
3959 945.2
3958.2 946
3958.2 947
3958 947.2
3957.2 948
3957 948.2
3956.2 949
3956 949.2
3955.2 950
3955 950.2
3954.2 951
3954 951.2
3953 951.2
3952.2 952
3952 952.2
3951.2 953
3951 953.2
3950 953.2
3949.2 954
3949 954.2
3948 954.2
3947.2 955
3947 955.2
3946 955.2
3945.2 956
3945 956.2
3944 956.2
3943.2 957
3943 957.2
3942 957.2
3941 957.2
3940 957.2
3939.2 958
3939 958.2
3938 958.2
3937 958.2
3936 958.2
3935 958.2
3934 958.2
3933 958.2
3932 958.2
3931 958.2
3930 958.2
3929 958.2
3928 958.2
3927 958.2
3926 958.2
3925.8 958
3925 957.2
3924 957.2
3923.8 957
3923 956.2
3922.8 956
3922 955.2
3921.8 955
3921 954.2
3920.8 954
3920 953.2
3919.2 954
3919.2 955
3919 955.2
3918.2 956
3918 956.2
3917 956.2
3916.2 957
3916 957.2
3915 957.2
3914.2 958
3914 958.2
3913 958.2
3912 958.2
3911.2 959
3911 959.2
};
\addplot [red]
table {%
590 1020.2
589 1020.2
588 1020.2
587 1020.2
586 1020.2
585 1020.2
584 1020.2
583 1020.2
582 1020.2
581 1020.2
580.8 1020
580 1019.2
579 1019.2
578 1019.2
577 1019.2
576 1019.2
575 1019.2
574 1019.2
573 1019.2
572.8 1019
572 1018.2
571 1018.2
570 1018.2
569 1018.2
568 1018.2
567.8 1018
567 1017.2
566 1017.2
565 1017.2
564 1017.2
563.8 1017
563 1016.2
562 1016.2
561 1016.2
560.8 1016
560 1015.2
559 1015.2
558 1015.2
557 1015.2
556.8 1015
556 1014.2
555 1014.2
554 1014.2
553.8 1014
553 1013.2
552 1013.2
551.8 1013
551 1012.2
550 1012.2
549 1012.2
548.8 1012
548 1011.2
547 1011.2
546 1011.2
545.8 1011
545 1010.2
544 1010.2
543 1010.2
542.8 1010
542 1009.2
541 1009.2
540 1009.2
539 1009.2
538.8 1009
538 1008.2
537 1008.2
536 1008.2
535.8 1008
535 1007.2
534 1007.2
533 1007.2
532 1007.2
531.8 1007
531 1006.2
530 1006.2
529 1006.2
528.8 1006
528 1005.2
527 1005.2
526 1005.2
525.8 1005
525 1004.2
524 1004.2
523 1004.2
522.8 1004
522 1003.2
521 1003.2
520 1003.2
519 1003.2
518.8 1003
518 1002.2
517 1002.2
516 1002.2
515 1002.2
514 1002.2
513.8 1002
513 1001.2
512 1001.2
511 1001.2
510 1001.2
509 1001.2
508.8 1001
508 1000.2
507 1000.2
506 1000.2
505.8 1000
505 999.2
504 999.2
503 999.2
502.8 999
502 998.2
501 998.2
500 998.2
499 998.2
498.8 998
498 997.2
497 997.2
496 997.2
495.8 997
495 996.2
494 996.2
493 996.2
492.8 996
492 995.2
491 995.2
490.8 995
490 994.2
489.8 994
489 993.2
488.8 993
488 992.2
487.8 992
487 991.2
486.8 991
486 990.2
485 990.2
484.8 990
484 989.2
483.8 989
483 988.2
482.8 988
482 987.2
481.8 987
481 986.2
480.8 986
480 985.2
479.8 985
479 984.2
478.8 984
478.8 983
478 982.2
477.8 982
477.8 981
477.8 980
477 979.2
476.8 979
476.8 978
476.8 977
476 976.2
475.8 976
475.8 975
475.8 974
475 973.2
474.8 973
474.8 972
474 971.2
473.8 971
473.8 970
473.8 969
473.8 968
474 967.8
474.8 967
474.8 966
474.8 965
474.8 964
474.8 963
475 962.8
475.8 962
475.8 961
475.8 960
475.8 959
475.8 958
475.8 957
476 956.8
476.8 956
476.8 955
476.8 954
477 953.8
477.8 953
478 952.8
478.8 952
479 951.8
479.8 951
480 950.8
480.8 950
481 949.8
481.8 949
482 948.8
482.8 948
483 947.8
483.8 947
484 946.8
484.8 946
485 945.8
485.8 945
485.8 944
486 943.8
486.8 943
487 942.8
488 942.8
488.8 942
489 941.8
490 941.8
490.8 941
491 940.8
492 940.8
492.8 940
493 939.8
494 939.8
494.8 939
495 938.8
495.8 938
496 937.8
497 937.8
497.8 937
498 936.8
499 936.8
499.8 936
500 935.8
501 935.8
501.8 935
502 934.8
503 934.8
504 934.8
505 934.8
505.8 934
506 933.8
507 933.8
508 933.8
509 933.8
509.8 933
510 932.8
511 932.8
512 932.8
513 932.8
513.8 932
514 931.8
515 931.8
516 931.8
517 931.8
518 931.8
519 931.8
520 931.8
521 931.8
522 931.8
523 931.8
524 931.8
524.8 931
525 930.8
526 930.8
527 930.8
528 930.8
529 930.8
530 930.8
531 930.8
531.2 931
532 931.8
533 931.8
534 931.8
535 931.8
536 931.8
537 931.8
538 931.8
539 931.8
540 931.8
541 931.8
542 931.8
542.2 932
543 932.8
544 932.8
545 932.8
546 932.8
547 932.8
548 932.8
549 932.8
549.2 933
550 933.8
551 933.8
552 933.8
553 933.8
554 933.8
554.2 934
555 934.8
556 934.8
557 934.8
558 934.8
558.2 935
559 935.8
560 935.8
561 935.8
562 935.8
563 935.8
563.2 936
564 936.8
565 936.8
566 936.8
567 936.8
568 936.8
569 936.8
569.2 937
570 937.8
571 937.8
572 937.8
573 937.8
573.2 938
574 938.8
575 938.8
575.2 939
576 939.8
577 939.8
577.2 940
578 940.8
579 940.8
579.2 941
580 941.8
581 941.8
582 941.8
582.2 942
583 942.8
584 942.8
584.2 943
585 943.8
586 943.8
586.2 944
587 944.8
588 944.8
589 944.8
589.2 945
590 945.8
591 945.8
591.2 946
592 946.8
593 946.8
593.2 947
594 947.8
595 947.8
595.2 948
596 948.8
596.2 949
597 949.8
598 949.8
598.2 950
599 950.8
600 950.8
600.2 951
601 951.8
601.2 952
602 952.8
602.2 953
603 953.8
603.2 954
604 954.8
604.2 955
605 955.8
605.2 956
606 956.8
607 956.8
607.2 957
608 957.8
608.2 958
609 958.8
609.2 959
609.2 960
610 960.8
610.2 961
610.2 962
611 962.8
611.2 963
611.2 964
612 964.8
612.2 965
612.2 966
613 966.8
613.2 967
613.2 968
614 968.8
614.2 969
614.2 970
615 970.8
615.2 971
615.2 972
615.2 973
615.2 974
615.2 975
616 975.8
616.2 976
616.2 977
616.2 978
616.2 979
616.2 980
617 980.8
617.2 981
617.2 982
617.2 983
617.2 984
617.2 985
617 985.2
616.2 986
616.2 987
616.2 988
616.2 989
616.2 990
616.2 991
616 991.2
615.2 992
615.2 993
615.2 994
615.2 995
615.2 996
615 996.2
614.2 997
614.2 998
614.2 999
614 999.2
613.2 1000
613.2 1001
613.2 1002
613 1002.2
612.2 1003
612.2 1004
612.2 1005
612 1005.2
611.2 1006
611.2 1007
611.2 1008
611 1008.2
610.2 1009
610 1009.2
609.2 1010
609 1010.2
608 1010.2
607.2 1011
607 1011.2
606.2 1012
606 1012.2
605.2 1013
605 1013.2
604.2 1014
604 1014.2
603.2 1015
603 1015.2
602.2 1016
602 1016.2
601.2 1017
601 1017.2
600 1017.2
599 1017.2
598.2 1018
598 1018.2
597 1018.2
596 1018.2
595 1018.2
594.2 1019
594 1019.2
593 1019.2
592 1019.2
591 1019.2
590.2 1020
590 1020.2
};
\addplot [red]
table {%
1533 958.2
1532 958.2
1531 958.2
1530.8 958
1530 957.2
1529 957.2
1528 957.2
1527.8 957
1527 956.2
1526.8 956
1526 955.2
1525.8 955
1525.8 954
1525 953.2
1524.8 953
1524.8 952
1524 951.2
1523.8 951
1523.8 950
1523.8 949
1523.8 948
1523.8 947
1523.8 946
1523.8 945
1523.8 944
1524 943.8
1524.8 943
1524.8 942
1525 941.8
1525.8 941
1526 940.8
1526.8 940
1527 939.8
1527.8 939
1528 938.8
1529 938.8
1530 938.8
1531 938.8
1532 938.8
1533 938.8
1533.2 939
1534 939.8
1535 939.8
1535.2 940
1536 940.8
1536.2 941
1537 941.8
1537.2 942
1538 942.8
1538.2 943
1539 943.8
1539.2 944
1539.2 945
1539.2 946
1539.2 947
1539.2 948
1539.2 949
1539.2 950
1539.2 951
1539.2 952
1539.2 953
1539 953.2
1538.2 954
1538.2 955
1538 955.2
1537.2 956
1537 956.2
1536.2 957
1536 957.2
1535 957.2
1534 957.2
1533.2 958
1533 958.2
};
\addplot [red]
table {%
1736 963.2
1735 963.2
1734 963.2
1733 963.2
1732 963.2
1731 963.2
1730 963.2
1729.8 963
1729 962.2
1728 962.2
1727 962.2
1726 962.2
1725 962.2
1724.8 962
1724 961.2
1723 961.2
1722 961.2
1721 961.2
1720.8 961
1720 960.2
1719 960.2
1718 960.2
1717 960.2
1716 960.2
1715.8 960
1715 959.2
1714 959.2
1713.8 959
1713 958.2
1712 958.2
1711.8 958
1711 957.2
1710.8 957
1710 956.2
1709.8 956
1709 955.2
1708 955.2
1707 955.2
1706 955.2
1705.8 955
1705 954.2
1704 954.2
1703 954.2
1702 954.2
1701 954.2
1700.8 954
1700 953.2
1699 953.2
1698 953.2
1697 953.2
1696 953.2
1695.8 953
1695 952.2
1694 952.2
1693 952.2
1692 952.2
1691 952.2
1690 952.2
1689.8 952
1689 951.2
1688 951.2
1687 951.2
1686 951.2
1685 951.2
1684.8 951
1684 950.2
1683 950.2
1682 950.2
1681 950.2
1680 950.2
1679.8 950
1679.8 949
1680 948.8
1680.8 948
1680.8 947
1680.8 946
1681 945.8
1681.8 945
1681.8 944
1681.8 943
1682 942.8
1682.8 942
1683 941.8
1684 941.8
1685 941.8
1686 941.8
1687 941.8
1688 941.8
1689 941.8
1690 941.8
1691 941.8
1691.8 941
1692 940.8
1693 940.8
1694 940.8
1695 940.8
1696 940.8
1697 940.8
1698 940.8
1699 940.8
1700 940.8
1701 940.8
1702 940.8
1702.8 940
1703 939.8
1704 939.8
1705 939.8
1706 939.8
1707 939.8
1708 939.8
1709 939.8
1710 939.8
1710.8 939
1711 938.8
1711.2 939
1712 939.8
1713 939.8
1714 939.8
1715 939.8
1716 939.8
1717 939.8
1717.2 940
1718 940.8
1719 940.8
1720 940.8
1721 940.8
1722 940.8
1723 940.8
1724 940.8
1725 940.8
1726 940.8
1726.2 941
1727 941.8
1728 941.8
1729 941.8
1729.2 942
1730 942.8
1731 942.8
1732 942.8
1733 942.8
1734 942.8
1734.2 943
1735 943.8
1736 943.8
1737 943.8
1738 943.8
1739 943.8
1740 943.8
1740.2 944
1741 944.8
1742 944.8
1743 944.8
1744 944.8
1745 944.8
1746 944.8
1747 944.8
1747.2 945
1748 945.8
1749 945.8
1750 945.8
1750.2 946
1751 946.8
1752 946.8
1753 946.8
1753.2 947
1754 947.8
1755 947.8
1755.2 948
1756 948.8
1757 948.8
1758 948.8
1758.2 949
1759 949.8
1760 949.8
1761 949.8
1761.2 950
1761.2 951
1761 951.2
1760.2 952
1760.2 953
1760.2 954
1760 954.2
1759.2 955
1759.2 956
1759 956.2
1758.2 957
1758 957.2
1757 957.2
1756.2 958
1756 958.2
1755 958.2
1754 958.2
1753.2 959
1753 959.2
1752 959.2
1751 959.2
1750 959.2
1749.2 960
1749 960.2
1748 960.2
1747 960.2
1746.2 961
1746 961.2
1745 961.2
1744 961.2
1743 961.2
1742.2 962
1742 962.2
1741 962.2
1740 962.2
1739 962.2
1738 962.2
1737 962.2
1736.2 963
1736 963.2
};
\addplot [red]
table {%
1615 1020.2
1614 1020.2
1613 1020.2
1612 1020.2
1611 1020.2
1610 1020.2
1609.8 1020
1609 1019.2
1608 1019.2
1607 1019.2
1606 1019.2
1605 1019.2
1604 1019.2
1603.8 1019
1603 1018.2
1602 1018.2
1601 1018.2
1600 1018.2
1599.8 1018
1599 1017.2
1598 1017.2
1597.8 1017
1597 1016.2
1596 1016.2
1595 1016.2
1594.8 1016
1594 1015.2
1593 1015.2
1592 1015.2
1591 1015.2
1590.8 1015
1590 1014.2
1589 1014.2
1588 1014.2
1587 1014.2
1586.8 1014
1586 1013.2
1585 1013.2
1584 1013.2
1583 1013.2
1582.8 1013
1582 1012.2
1581 1012.2
1580 1012.2
1579 1012.2
1578.8 1012
1578 1011.2
1577 1011.2
1576.8 1011
1576 1010.2
1575 1010.2
1574.8 1010
1574 1009.2
1573 1009.2
1572.8 1009
1572 1008.2
1571 1008.2
1570 1008.2
1569.8 1008
1569 1007.2
1568 1007.2
1567.8 1007
1567 1006.2
1566.8 1006
1566 1005.2
1565 1005.2
1564.8 1005
1564 1004.2
1563.8 1004
1563 1003.2
1562 1003.2
1561.8 1003
1561 1002.2
1560.8 1002
1560 1001.2
1559.8 1001
1559 1000.2
1558 1000.2
1557.8 1000
1557 999.2
1556.8 999
1556 998.2
1555 998.2
1554.8 998
1554.8 997
1554.8 996
1554 995.2
1553.8 995
1553.8 994
1553.8 993
1553 992.2
1552.8 992
1552.8 991
1552.8 990
1552 989.2
1551.8 989
1551.8 988
1551.8 987
1551.8 986
1552 985.8
1552.8 985
1553 984.8
1554 984.8
1554.8 984
1555 983.8
1555.8 983
1556 982.8
1556.8 982
1557 981.8
1558 981.8
1558.8 981
1559 980.8
1559.8 980
1560 979.8
1560.8 979
1561 978.8
1562 978.8
1563 978.8
1564 978.8
1564.8 978
1565 977.8
1566 977.8
1567 977.8
1568 977.8
1569 977.8
1570 977.8
1571 977.8
1572 977.8
1573 977.8
1574 977.8
1575 977.8
1576 977.8
1577 977.8
1578 977.8
1579 977.8
1580 977.8
1581 977.8
1582 977.8
1583 977.8
1584 977.8
1585 977.8
1586 977.8
1587 977.8
1588 977.8
1589 977.8
1590 977.8
1591 977.8
1592 977.8
1593 977.8
1594 977.8
1595 977.8
1596 977.8
1597 977.8
1598 977.8
1599 977.8
1600 977.8
1601 977.8
1602 977.8
1603 977.8
1604 977.8
1605 977.8
1606 977.8
1607 977.8
1608 977.8
1609 977.8
1610 977.8
1611 977.8
1611.2 978
1612 978.8
1613 978.8
1614 978.8
1615 978.8
1616 978.8
1617 978.8
1618 978.8
1619 978.8
1619.2 979
1620 979.8
1621 979.8
1621.2 980
1622 980.8
1623 980.8
1624 980.8
1624.2 981
1625 981.8
1626 981.8
1627 981.8
1628 981.8
1628.2 982
1629 982.8
1630 982.8
1631 982.8
1632 982.8
1633 982.8
1634 982.8
1635 982.8
1635.2 983
1636 983.8
1637 983.8
1638 983.8
1639 983.8
1640 983.8
1641 983.8
1642 983.8
1643 983.8
1644 983.8
1645 983.8
1646 983.8
1647 983.8
1648 983.8
1649 983.8
1650 983.8
1651 983.8
1652 983.8
1653 983.8
1654 983.8
1655 983.8
1655.2 984
1656 984.8
1657 984.8
1658 984.8
1659 984.8
1660 984.8
1661 984.8
1662 984.8
1663 984.8
1664 984.8
1665 984.8
1666 984.8
1667 984.8
1668 984.8
1669 984.8
1670 984.8
1671 984.8
1672 984.8
1672.2 985
1673 985.8
1674 985.8
1675 985.8
1675.2 986
1676 986.8
1676.2 987
1677 987.8
1677.2 988
1678 988.8
1678.2 989
1679 989.8
1680 989.8
1680.2 990
1681 990.8
1681.2 991
1682 991.8
1682.2 992
1683 992.8
1683.2 993
1684 993.8
1684.2 994
1685 994.8
1685.2 995
1686 995.8
1686.2 996
1687 996.8
1687.2 997
1688 997.8
1688.2 998
1688 998.2
1687.2 999
1687 999.2
1686 999.2
1685.2 1000
1685 1000.2
1684 1000.2
1683.2 1001
1683 1001.2
1682 1001.2
1681.2 1002
1681 1002.2
1680 1002.2
1679.2 1003
1679 1003.2
1678 1003.2
1677.2 1004
1677 1004.2
1676.2 1005
1676 1005.2
1675 1005.2
1674.2 1006
1674 1006.2
1673 1006.2
1672.2 1007
1672 1007.2
1671 1007.2
1670.2 1008
1670 1008.2
1669 1008.2
1668.2 1009
1668 1009.2
1667 1009.2
1666 1009.2
1665.2 1010
1665 1010.2
1664 1010.2
1663 1010.2
1662 1010.2
1661 1010.2
1660.2 1011
1660 1011.2
1659 1011.2
1658 1011.2
1657 1011.2
1656 1011.2
1655 1011.2
1654.2 1012
1654 1012.2
1653 1012.2
1652 1012.2
1651 1012.2
1650 1012.2
1649.2 1013
1649 1013.2
1648 1013.2
1647 1013.2
1646 1013.2
1645.2 1014
1645 1014.2
1644 1014.2
1643 1014.2
1642 1014.2
1641.2 1015
1641 1015.2
1640 1015.2
1639 1015.2
1638 1015.2
1637.2 1016
1637 1016.2
1636 1016.2
1635 1016.2
1634 1016.2
1633 1016.2
1632.2 1017
1632 1017.2
1631 1017.2
1630 1017.2
1629 1017.2
1628.2 1018
1628 1018.2
1627 1018.2
1626 1018.2
1625 1018.2
1624 1018.2
1623.2 1019
1623 1019.2
1622 1019.2
1621 1019.2
1620 1019.2
1619 1019.2
1618 1019.2
1617 1019.2
1616 1019.2
1615.2 1020
1615 1020.2
};
\addplot [red]
table {%
2314 990.2
2313.8 990
2313 989.2
2312 989.2
2311 989.2
2310 989.2
2309 989.2
2308 989.2
2307 989.2
2306 989.2
2305 989.2
2304 989.2
2303.8 989
2303 988.2
2302.8 988
2302 987.2
2301.8 987
2301.8 986
2301.8 985
2301.8 984
2302 983.8
2302.8 983
2303 982.8
2303.8 982
2304 981.8
2304.8 981
2305 980.8
2306 980.8
2306.8 980
2307 979.8
2308 979.8
2309 979.8
2309.8 979
2310 978.8
2311 978.8
2312 978.8
2313 978.8
2314 978.8
2315 978.8
2316 978.8
2317 978.8
2318 978.8
2318.2 979
2319 979.8
2320 979.8
2320.2 980
2321 980.8
2321.2 981
2321.2 982
2321.2 983
2322 983.8
2322.2 984
2322 984.2
2321.2 985
2321.2 986
2321.2 987
2321 987.2
2320.2 988
2320 988.2
2319.2 989
2319 989.2
2318 989.2
2317 989.2
2316 989.2
2315 989.2
2314.2 990
2314 990.2
};
\addplot [red]
table {%
1722 1020.2
1721 1020.2
1720 1020.2
1719 1020.2
1718 1020.2
1717 1020.2
1716 1020.2
1715 1020.2
1714 1020.2
1713 1020.2
1712 1020.2
1711 1020.2
1710 1020.2
1709 1020.2
1708 1020.2
1707 1020.2
1706.8 1020
1706 1019.2
1705 1019.2
1704 1019.2
1703 1019.2
1702 1019.2
1701 1019.2
1700 1019.2
1699 1019.2
1698 1019.2
1697 1019.2
1696.8 1019
1696 1018.2
1695 1018.2
1694.8 1018
1694 1017.2
1693 1017.2
1692.8 1017
1692 1016.2
1691 1016.2
1690.8 1016
1690.8 1015
1690.8 1014
1690.8 1013
1690.8 1012
1690.8 1011
1691 1010.8
1691.8 1010
1691.8 1009
1692 1008.8
1692.8 1008
1692.8 1007
1693 1006.8
1693.8 1006
1694 1005.8
1694.8 1005
1694.8 1004
1695 1003.8
1695.8 1003
1696 1002.8
1696.8 1002
1697 1001.8
1697.8 1001
1698 1000.8
1698.8 1000
1698.8 999
1699 998.8
1699.8 998
1699.8 997
1700 996.8
1700.8 996
1701 995.8
1701.8 995
1702 994.8
1702.8 994
1703 993.8
1704 993.8
1704.8 993
1705 992.8
1705.8 992
1706 991.8
1706.8 991
1707 990.8
1708 990.8
1708.8 990
1709 989.8
1709.8 989
1710 988.8
1711 988.8
1712 988.8
1712.8 988
1713 987.8
1714 987.8
1715 987.8
1716 987.8
1717 987.8
1718 987.8
1719 987.8
1719.2 988
1720 988.8
1721 988.8
1722 988.8
1722.2 989
1723 989.8
1724 989.8
1725 989.8
1725.2 990
1726 990.8
1726.2 991
1727 991.8
1728 991.8
1728.2 992
1729 992.8
1729.2 993
1730 993.8
1731 993.8
1731.2 994
1732 994.8
1733 994.8
1733.2 995
1734 995.8
1735 995.8
1736 995.8
1736.2 996
1737 996.8
1738 996.8
1739 996.8
1739.2 997
1740 997.8
1741 997.8
1742 997.8
1743 997.8
1743.2 998
1744 998.8
1745 998.8
1746 998.8
1746.2 999
1747 999.8
1748 999.8
1749 999.8
1750 999.8
1750.2 1000
1751 1000.8
1752 1000.8
1753 1000.8
1754 1000.8
1754.2 1001
1755 1001.8
1756 1001.8
1757 1001.8
1758 1001.8
1758.2 1002
1759 1002.8
1760 1002.8
1761 1002.8
1761.2 1003
1762 1003.8
1763 1003.8
1764 1003.8
1765 1003.8
1765.2 1004
1765.2 1005
1766 1005.8
1766.2 1006
1766.2 1007
1766.2 1008
1766.2 1009
1766.2 1010
1767 1010.8
1767.2 1011
1767.2 1012
1767.2 1013
1767.2 1014
1767 1014.2
1766 1014.2
1765.2 1015
1765 1015.2
1764 1015.2
1763 1015.2
1762.2 1016
1762 1016.2
1761 1016.2
1760.2 1017
1760 1017.2
1759 1017.2
1758 1017.2
1757.2 1018
1757 1018.2
1756 1018.2
1755.2 1019
1755 1019.2
1754 1019.2
1753 1019.2
1752 1019.2
1751 1019.2
1750 1019.2
1749 1019.2
1748 1019.2
1747 1019.2
1746 1019.2
1745 1019.2
1744.8 1019
1744 1018.2
1743 1018.2
1742 1018.2
1741 1018.2
1740 1018.2
1739 1018.2
1738 1018.2
1737 1018.2
1736 1018.2
1735 1018.2
1734 1018.2
1733 1018.2
1732 1018.2
1731 1018.2
1730 1018.2
1729 1018.2
1728.2 1019
1728 1019.2
1727 1019.2
1726 1019.2
1725 1019.2
1724 1019.2
1723 1019.2
1722.2 1020
1722 1020.2
};
\addplot [red]
table {%
2373 1035.2
2372 1035.2
2371 1035.2
2370.8 1035
2370 1034.2
2369 1034.2
2368.8 1034
2368 1033.2
2367 1033.2
2366.8 1033
2366.8 1032
2366 1031.2
2365.8 1031
2365.8 1030
2365.8 1029
2366 1028.8
2366.8 1028
2366.8 1027
2367 1026.8
2367.8 1026
2368 1025.8
2368.8 1025
2369 1024.8
2369.8 1024
2370 1023.8
2371 1023.8
2371.8 1023
2372 1022.8
2373 1022.8
2374 1022.8
2375 1022.8
2376 1022.8
2376.2 1023
2377 1023.8
2378 1023.8
2378.2 1024
2379 1024.8
2379.2 1025
2379.2 1026
2379.2 1027
2379.2 1028
2379 1028.2
2378.2 1029
2378.2 1030
2378 1030.2
2377.2 1031
2377 1031.2
2376.2 1032
2376.2 1033
2376 1033.2
2375.2 1034
2375 1034.2
2374 1034.2
2373.2 1035
2373 1035.2
};
\addplot [red]
table {%
780 1104.2
779 1104.2
778 1104.2
777 1104.2
776 1104.2
775.8 1104
775 1103.2
774 1103.2
773 1103.2
772.8 1103
772 1102.2
771.8 1102
771 1101.2
770.8 1101
770 1100.2
769.8 1100
769 1099.2
768.8 1099
768.8 1098
768.8 1097
768 1096.2
767.8 1096
768 1095.8
768.8 1095
768.8 1094
768 1093.2
767.8 1093
768 1092.8
768.8 1092
768.8 1091
769 1090.8
769.8 1090
770 1089.8
770.8 1089
771 1088.8
771.8 1088
772 1087.8
773 1087.8
774 1087.8
774.2 1088
775 1088.8
776 1088.8
777 1088.8
778 1088.8
778.2 1089
779 1089.8
779.2 1090
780 1090.8
780.2 1091
781 1091.8
781.2 1092
782 1092.8
782.2 1093
783 1093.8
783.2 1094
783.2 1095
784 1095.8
784.2 1096
784.2 1097
784.2 1098
784.2 1099
784.2 1100
784 1100.2
783.2 1101
783.2 1102
783 1102.2
782.2 1103
782 1103.2
781 1103.2
780.2 1104
780 1104.2
};
\addplot [red]
table {%
5017 1166.2
5016 1166.2
5015 1166.2
5014 1166.2
5013 1166.2
5012 1166.2
5011 1166.2
5010 1166.2
5009 1166.2
5008 1166.2
5007 1166.2
5006 1166.2
5005 1166.2
5004 1166.2
5003 1166.2
5002.8 1166
5002 1165.2
5001 1165.2
5000.8 1165
5000 1164.2
4999 1164.2
4998.8 1164
4998 1163.2
4997 1163.2
4996.8 1163
4996 1162.2
4995.8 1162
4995 1161.2
4994.8 1161
4994 1160.2
4993.8 1160
4993 1159.2
4992.8 1159
4992 1158.2
4991.8 1158
4991 1157.2
4990.8 1157
4990.8 1156
4990 1155.2
4989.8 1155
4989.8 1154
4989 1153.2
4988.8 1153
4988.8 1152
4988 1151.2
4987.8 1151
4987.8 1150
4987.8 1149
4987.8 1148
4987 1147.2
4986.8 1147
4986 1146.2
4985.2 1147
4985 1147.2
4984.2 1148
4984 1148.2
4983.2 1149
4983.2 1150
4983 1150.2
4982.2 1151
4982 1151.2
4981.2 1152
4981 1152.2
4980 1152.2
4979.2 1153
4979 1153.2
4978.2 1154
4978 1154.2
4977.2 1155
4977 1155.2
4976 1155.2
4975.2 1156
4975 1156.2
4974.2 1157
4974 1157.2
4973 1157.2
4972 1157.2
4971.2 1158
4971 1158.2
4970 1158.2
4969.2 1159
4969 1159.2
4968 1159.2
4967 1159.2
4966.2 1160
4966 1160.2
4965 1160.2
4964.2 1161
4964 1161.2
4963 1161.2
4962 1161.2
4961.2 1162
4961 1162.2
4960 1162.2
4959 1162.2
4958 1162.2
4957.2 1163
4957 1163.2
4956 1163.2
4955 1163.2
4954 1163.2
4953 1163.2
4952 1163.2
4951 1163.2
4950 1163.2
4949 1163.2
4948 1163.2
4947.8 1163
4947 1162.2
4946 1162.2
4945 1162.2
4944 1162.2
4943.8 1162
4943 1161.2
4942.8 1161
4942 1160.2
4941.8 1160
4941 1159.2
4940.8 1159
4940 1158.2
4939.8 1158
4939.8 1157
4939 1156.2
4938.8 1156
4938 1155.2
4937.8 1155
4937.8 1154
4937 1153.2
4936.8 1153
4936.8 1152
4936 1151.2
4935.8 1151
4935.8 1150
4935.8 1149
4935 1148.2
4934.8 1148
4934.8 1147
4934 1146.2
4933.8 1146
4933.8 1145
4933.8 1144
4933.8 1143
4933.8 1142
4933.8 1141
4933 1140.2
4932.8 1140
4932.8 1139
4932.8 1138
4932.8 1137
4932.8 1136
4932.8 1135
4932.8 1134
4933 1133.8
4933.8 1133
4933.8 1132
4933.8 1131
4933.8 1130
4933.8 1129
4934 1128.8
4934.8 1128
4934.8 1127
4934.8 1126
4934.8 1125
4935 1124.8
4935.8 1124
4935.8 1123
4935.8 1122
4936 1121.8
4936.8 1121
4936.8 1120
4937 1119.8
4937.8 1119
4938 1118.8
4938.8 1118
4938.8 1117
4939 1116.8
4939.8 1116
4940 1115.8
4941 1115.8
4941.8 1115
4942 1114.8
4942.8 1114
4943 1113.8
4943.8 1113
4944 1112.8
4945 1112.8
4945.8 1112
4946 1111.8
4947 1111.8
4948 1111.8
4948.8 1111
4949 1110.8
4950 1110.8
4950.8 1110
4951 1109.8
4952 1109.8
4953 1109.8
4953.8 1109
4954 1108.8
4955 1108.8
4956 1108.8
4957 1108.8
4957.8 1108
4958 1107.8
4959 1107.8
4960 1107.8
4961 1107.8
4961.8 1107
4962 1106.8
4963 1106.8
4964 1106.8
4965 1106.8
4966 1106.8
4967 1106.8
4968 1106.8
4969 1106.8
4970 1106.8
4971 1106.8
4971.2 1107
4972 1107.8
4972.8 1107
4972 1106.2
4971.8 1106
4971 1105.2
4970 1105.2
4969.8 1105
4969 1104.2
4968 1104.2
4967.8 1104
4967 1103.2
4966.8 1103
4966 1102.2
4965.8 1102
4965.8 1101
4965 1100.2
4964.8 1100
4964 1099.2
4963.8 1099
4964 1098.8
4964.8 1098
4964.8 1097
4965 1096.8
4965.8 1096
4966 1095.8
4966.8 1095
4967 1094.8
4968 1094.8
4968.8 1094
4969 1093.8
4969.2 1094
4970 1094.8
4971 1094.8
4972 1094.8
4973 1094.8
4973.2 1095
4974 1095.8
4975 1095.8
4975.2 1096
4976 1096.8
4977 1096.8
4977.2 1097
4978 1097.8
4979 1097.8
4979.2 1098
4980 1098.8
4980.2 1099
4981 1099.8
4981.2 1100
4981.2 1101
4981.2 1102
4982 1102.8
4982.2 1103
4982.2 1104
4982.2 1105
4982.2 1106
4982.2 1107
4982 1107.2
4981.2 1108
4981 1108.2
4980 1108.2
4979 1108.2
4978.2 1109
4979 1109.8
4980 1109.8
4980.2 1110
4981 1110.8
4981.2 1111
4982 1111.8
4983 1111.8
4983.2 1112
4984 1112.8
4984.2 1113
4985 1113.8
4985.2 1114
4985.2 1115
4986 1115.8
4986.2 1116
4987 1116.8
4987.2 1117
4987.2 1118
4988 1118.8
4988.8 1118
4989 1117.8
4989.8 1117
4990 1116.8
4990.8 1116
4991 1115.8
4991.8 1115
4992 1114.8
4993 1114.8
4994 1114.8
4995 1114.8
4996 1114.8
4997 1114.8
4998 1114.8
4999 1114.8
5000 1114.8
5000.2 1115
5001 1115.8
5002 1115.8
5003 1115.8
5004 1115.8
5004.2 1116
5005 1116.8
5006 1116.8
5007 1116.8
5007.2 1117
5008 1117.8
5009 1117.8
5009.2 1118
5010 1118.8
5010.2 1119
5011 1119.8
5012 1119.8
5012.2 1120
5013 1120.8
5014 1120.8
5014.2 1121
5015 1121.8
5016 1121.8
5017 1121.8
5018 1121.8
5019 1121.8
5020 1121.8
5021 1121.8
5022 1121.8
5023 1121.8
5024 1121.8
5025 1121.8
5026 1121.8
5027 1121.8
5027.8 1121
5028 1120.8
5029 1120.8
5030 1120.8
5031 1120.8
5032 1120.8
5033 1120.8
5033.8 1120
5034 1119.8
5035 1119.8
5036 1119.8
5037 1119.8
5038 1119.8
5039 1119.8
5040 1119.8
5041 1119.8
5042 1119.8
5043 1119.8
5044 1119.8
5045 1119.8
5046 1119.8
5047 1119.8
5048 1119.8
5049 1119.8
5050 1119.8
5051 1119.8
5052 1119.8
5053 1119.8
5054 1119.8
5055 1119.8
5056 1119.8
5056.2 1120
5057 1120.8
5058 1120.8
5059 1120.8
5059.2 1121
5060 1121.8
5061 1121.8
5062 1121.8
5063 1121.8
5063.2 1122
5064 1122.8
5065 1122.8
5066 1122.8
5066.2 1123
5067 1123.8
5068 1123.8
5069 1123.8
5069.2 1124
5070 1124.8
5071 1124.8
5072 1124.8
5072.2 1125
5072.2 1126
5072.2 1127
5072.2 1128
5072.2 1129
5072 1129.2
5071.2 1130
5071.2 1131
5071.2 1132
5071.2 1133
5071 1133.2
5070.2 1134
5070.2 1135
5070.2 1136
5070.2 1137
5070 1137.2
5069.2 1138
5069 1138.2
5068.2 1139
5068 1139.2
5067 1139.2
5066.2 1140
5066 1140.2
5065 1140.2
5064.2 1141
5064 1141.2
5063 1141.2
5062.2 1142
5062 1142.2
5061.2 1143
5061 1143.2
5060 1143.2
5059.2 1144
5059 1144.2
5058 1144.2
5057.2 1145
5057 1145.2
5056 1145.2
5055.2 1146
5055 1146.2
5054.2 1147
5054 1147.2
5053 1147.2
5052 1147.2
5051.2 1148
5051 1148.2
5050 1148.2
5049.2 1149
5049 1149.2
5048 1149.2
5047 1149.2
5046.2 1150
5046 1150.2
5045 1150.2
5044.2 1151
5044 1151.2
5043 1151.2
5042 1151.2
5041.2 1152
5041 1152.2
5040.2 1153
5040 1153.2
5039.2 1154
5039 1154.2
5038 1154.2
5037.2 1155
5037 1155.2
5036.2 1156
5036 1156.2
5035 1156.2
5034.2 1157
5034 1157.2
5033.2 1158
5033 1158.2
5032.2 1159
5032 1159.2
5031 1159.2
5030.2 1160
5030 1160.2
5029 1160.2
5028.2 1161
5028 1161.2
5027.2 1162
5027 1162.2
5026 1162.2
5025.2 1163
5025 1163.2
5024 1163.2
5023.2 1164
5023 1164.2
5022 1164.2
5021 1164.2
5020.2 1165
5020 1165.2
5019 1165.2
5018 1165.2
5017.2 1166
5017 1166.2
};
\addplot [red]
table {%
491 1154.2
490 1154.2
489 1154.2
488 1154.2
487 1154.2
486 1154.2
485 1154.2
484 1154.2
483 1154.2
482 1154.2
481 1154.2
480.8 1154
480 1153.2
479 1153.2
478.2 1154
478 1154.2
477.8 1154
477 1153.2
476 1153.2
475 1153.2
474 1153.2
473.2 1154
473 1154.2
472.8 1154
472 1153.2
471 1153.2
470 1153.2
469.8 1153
469 1152.2
468 1152.2
467 1152.2
466.8 1152
466 1151.2
465 1151.2
464 1151.2
463 1151.2
462.8 1151
462 1150.2
461 1150.2
460 1150.2
459.8 1150
459 1149.2
458 1149.2
457 1149.2
456 1149.2
455 1149.2
454 1149.2
453.8 1149
453 1148.2
452 1148.2
451 1148.2
450 1148.2
449 1148.2
448 1148.2
447.8 1148
447 1147.2
446 1147.2
445 1147.2
444.8 1147
444 1146.2
443 1146.2
442 1146.2
441 1146.2
440.8 1146
440 1145.2
439 1145.2
438 1145.2
437.8 1145
437 1144.2
436 1144.2
435 1144.2
434.8 1144
434 1143.2
433 1143.2
432 1143.2
431.8 1143
431 1142.2
430 1142.2
429 1142.2
428.8 1142
428 1141.2
427.8 1141
427.8 1140
427.8 1139
427 1138.2
426.8 1138
426 1137.2
425 1137.2
424 1137.2
423 1137.2
422 1137.2
421 1137.2
420 1137.2
419 1137.2
418 1137.2
417 1137.2
416 1137.2
415 1137.2
414 1137.2
413.8 1137
413 1136.2
412 1136.2
411 1136.2
410 1136.2
409 1136.2
408 1136.2
407 1136.2
406.8 1136
406 1135.2
405 1135.2
404 1135.2
403 1135.2
402.8 1135
402 1134.2
401 1134.2
400 1134.2
399 1134.2
398.8 1134
398 1133.2
397 1133.2
396 1133.2
395 1133.2
394.8 1133
394 1132.2
393 1132.2
392 1132.2
391 1132.2
390 1132.2
389 1132.2
388 1132.2
387 1132.2
386.8 1132
386 1131.2
385 1131.2
384 1131.2
383 1131.2
382 1131.2
381 1131.2
380.8 1131
380 1130.2
379 1130.2
378 1130.2
377 1130.2
376.8 1130
376 1129.2
375.8 1129
375 1128.2
374.8 1128
374 1127.2
373.8 1127
373 1126.2
372.8 1126
372 1125.2
371.8 1125
371 1124.2
370.8 1124
370.8 1123
370.8 1122
370.8 1121
370.8 1120
370.8 1119
370.8 1118
370.8 1117
371 1116.8
371.8 1116
372 1115.8
372.8 1115
373 1114.8
374 1114.8
374.8 1114
375 1113.8
375.8 1113
376 1112.8
376.8 1112
377 1111.8
378 1111.8
378.8 1111
379 1110.8
380 1110.8
381 1110.8
382 1110.8
383 1110.8
384 1110.8
385 1110.8
385.8 1110
386 1109.8
387 1109.8
388 1109.8
389 1109.8
390 1109.8
391 1109.8
392 1109.8
393 1109.8
393.2 1110
394 1110.8
395 1110.8
396 1110.8
397 1110.8
398 1110.8
399 1110.8
400 1110.8
401 1110.8
401.2 1111
402 1111.8
403 1111.8
404 1111.8
405 1111.8
406 1111.8
407 1111.8
407.2 1112
408 1112.8
409 1112.8
410 1112.8
411 1112.8
411.2 1113
412 1113.8
413 1113.8
414 1113.8
415 1113.8
416 1113.8
417 1113.8
417.2 1114
418 1114.8
419 1114.8
420 1114.8
421 1114.8
421.2 1115
422 1115.8
423 1115.8
424 1115.8
425 1115.8
426 1115.8
426.2 1116
427 1116.8
428 1116.8
429 1116.8
430 1116.8
431 1116.8
431.2 1117
432 1117.8
433 1117.8
434 1117.8
435 1117.8
436 1117.8
437 1117.8
437.2 1118
438 1118.8
439 1118.8
440 1118.8
440.2 1119
441 1119.8
442 1119.8
443 1119.8
444 1119.8
444.2 1120
445 1120.8
446 1120.8
447 1120.8
447.2 1121
448 1121.8
449 1121.8
450 1121.8
450.2 1122
451 1122.8
452 1122.8
453 1122.8
453.2 1123
454 1123.8
455 1123.8
456 1123.8
456.2 1124
457 1124.8
457.2 1125
457.2 1126
458 1126.8
459 1126.8
460 1126.8
461 1126.8
462 1126.8
463 1126.8
463.2 1127
464 1127.8
465 1127.8
466 1127.8
467 1127.8
468 1127.8
469 1127.8
470 1127.8
471 1127.8
472 1127.8
472.2 1128
473 1128.8
474 1128.8
474.2 1129
475 1129.8
476 1129.8
476.2 1130
477 1130.8
478 1130.8
478.2 1131
479 1131.8
480 1131.8
481 1131.8
482 1131.8
482.2 1132
483 1132.8
484 1132.8
484.2 1133
485 1133.8
486 1133.8
487 1133.8
487.2 1134
488 1134.8
489 1134.8
490 1134.8
490.2 1135
491 1135.8
492 1135.8
493 1135.8
493.2 1136
494 1136.8
495 1136.8
496 1136.8
496.2 1137
497 1137.8
498 1137.8
499 1137.8
499.2 1138
500 1138.8
500.2 1139
501 1139.8
501.2 1140
502 1140.8
502.2 1141
503 1141.8
503.2 1142
503.2 1143
503.2 1144
504 1144.8
504.2 1145
504.2 1146
504.2 1147
504 1147.2
503.2 1148
503 1148.2
502.2 1149
502 1149.2
501.2 1150
501 1150.2
500.2 1151
500 1151.2
499 1151.2
498.2 1152
498 1152.2
497 1152.2
496 1152.2
495.2 1153
495 1153.2
494 1153.2
493 1153.2
492 1153.2
491.2 1154
491 1154.2
};
\addplot [red]
table {%
1865 1231.2
1864 1231.2
1863 1231.2
1862 1231.2
1861 1231.2
1860 1231.2
1859 1231.2
1858 1231.2
1857 1231.2
1856 1231.2
1855 1231.2
1854 1231.2
1853 1231.2
1852.8 1231
1852 1230.2
1851 1230.2
1850 1230.2
1849 1230.2
1848 1230.2
1847 1230.2
1846 1230.2
1845 1230.2
1844.8 1230
1844 1229.2
1843 1229.2
1842 1229.2
1841 1229.2
1840 1229.2
1839.8 1229
1839 1228.2
1838 1228.2
1837 1228.2
1836 1228.2
1835.8 1228
1835 1227.2
1834 1227.2
1833 1227.2
1832.8 1227
1832 1226.2
1831 1226.2
1830 1226.2
1829.8 1226
1829 1225.2
1828 1225.2
1827 1225.2
1826.8 1225
1826 1224.2
1825 1224.2
1824.8 1224
1824 1223.2
1823.8 1223
1823 1222.2
1822 1222.2
1821.8 1222
1821 1221.2
1820 1221.2
1819.8 1221
1819 1220.2
1818.8 1220
1818 1219.2
1817 1219.2
1816.8 1219
1816 1218.2
1815 1218.2
1814.8 1218
1814 1217.2
1813 1217.2
1812 1217.2
1811.8 1217
1811 1216.2
1810 1216.2
1809.8 1216
1809 1215.2
1808 1215.2
1807 1215.2
1806 1215.2
1805 1215.2
1804 1215.2
1803.8 1215
1803 1214.2
1802 1214.2
1801 1214.2
1800 1214.2
1799 1214.2
1798 1214.2
1797 1214.2
1796 1214.2
1795.8 1214
1795 1213.2
1794 1213.2
1793 1213.2
1792 1213.2
1791 1213.2
1790 1213.2
1789 1213.2
1788 1213.2
1787 1213.2
1786.8 1213
1786 1212.2
1785 1212.2
1784 1212.2
1783 1212.2
1782 1212.2
1781 1212.2
1780 1212.2
1779 1212.2
1778 1212.2
1777.8 1212
1777 1211.2
1776 1211.2
1775 1211.2
1774 1211.2
1773 1211.2
1772 1211.2
1771 1211.2
1770 1211.2
1769 1211.2
1768.8 1211
1768 1210.2
1767 1210.2
1766 1210.2
1765 1210.2
1764 1210.2
1763 1210.2
1762 1210.2
1761 1210.2
1760.8 1210
1760 1209.2
1759 1209.2
1758.8 1209
1758 1208.2
1757.8 1208
1757 1207.2
1756.8 1207
1756 1206.2
1755 1206.2
1754.8 1206
1754 1205.2
1753 1205.2
1752 1205.2
1751.2 1206
1751 1206.2
1750 1206.2
1749 1206.2
1748 1206.2
1747.2 1207
1747 1207.2
1746 1207.2
1745 1207.2
1744 1207.2
1743 1207.2
1742 1207.2
1741 1207.2
1740 1207.2
1739 1207.2
1738 1207.2
1737 1207.2
1736 1207.2
1735 1207.2
1734.8 1207
1734 1206.2
1733 1206.2
1732 1206.2
1731 1206.2
1730 1206.2
1729 1206.2
1728 1206.2
1727 1206.2
1726 1206.2
1725 1206.2
1724.8 1206
1724 1205.2
1723 1205.2
1722 1205.2
1721 1205.2
1720 1205.2
1719 1205.2
1718 1205.2
1717.8 1205
1717 1204.2
1716 1204.2
1715 1204.2
1714.8 1204
1714 1203.2
1713 1203.2
1712 1203.2
1711.8 1203
1711 1202.2
1710 1202.2
1709 1202.2
1708.8 1202
1708 1201.2
1707 1201.2
1706 1201.2
1705.8 1201
1705 1200.2
1704 1200.2
1703 1200.2
1702.8 1200
1702 1199.2
1701 1199.2
1700 1199.2
1699.8 1199
1699 1198.2
1698 1198.2
1697 1198.2
1696.8 1198
1696 1197.2
1695 1197.2
1694 1197.2
1693.8 1197
1693 1196.2
1692 1196.2
1691.8 1196
1691 1195.2
1690 1195.2
1689 1195.2
1688.8 1195
1688 1194.2
1687 1194.2
1686.8 1194
1686 1193.2
1685 1193.2
1684.8 1193
1684 1192.2
1683 1192.2
1682.8 1192
1682 1191.2
1681 1191.2
1680.8 1191
1680 1190.2
1679 1190.2
1678.8 1190
1678 1189.2
1677 1189.2
1676.8 1189
1676 1188.2
1675 1188.2
1674.8 1188
1674 1187.2
1673 1187.2
1672.8 1187
1672 1186.2
1671 1186.2
1670 1186.2
1669 1186.2
1668.8 1186
1668 1185.2
1667 1185.2
1666.8 1185
1666 1184.2
1665 1184.2
1664 1184.2
1663.8 1184
1663 1183.2
1662 1183.2
1661.8 1183
1661 1182.2
1660 1182.2
1659.8 1182
1659 1181.2
1658.8 1181
1658 1180.2
1657 1180.2
1656.8 1180
1656 1179.2
1655 1179.2
1654.8 1179
1654 1178.2
1653.8 1178
1653 1177.2
1652 1177.2
1651.8 1177
1651 1176.2
1650.8 1176
1650 1175.2
1649.8 1175
1649 1174.2
1648.8 1174
1648 1173.2
1647.8 1173
1647 1172.2
1646.8 1172
1646 1171.2
1645.8 1171
1645.8 1170
1645 1169.2
1644.8 1169
1644.8 1168
1644 1167.2
1643.8 1167
1643.8 1166
1643 1165.2
1642.8 1165
1642.8 1164
1642.8 1163
1642.8 1162
1642.8 1161
1642.8 1160
1642.8 1159
1642 1158.2
1641.8 1158
1641.8 1157
1642 1156.8
1642.8 1156
1642.8 1155
1642.8 1154
1642.8 1153
1643 1152.8
1643.8 1152
1643.8 1151
1643.8 1150
1644 1149.8
1644.8 1149
1644.8 1148
1645 1147.8
1645.8 1147
1646 1146.8
1646.8 1146
1646.8 1145
1647 1144.8
1647.8 1144
1648 1143.8
1649 1143.8
1649.8 1143
1650 1142.8
1650.8 1142
1651 1141.8
1652 1141.8
1652.8 1141
1653 1140.8
1654 1140.8
1654.8 1140
1655 1139.8
1656 1139.8
1657 1139.8
1657.8 1139
1658 1138.8
1659 1138.8
1660 1138.8
1661 1138.8
1661.8 1138
1662 1137.8
1663 1137.8
1664 1137.8
1665 1137.8
1666 1137.8
1667 1137.8
1668 1137.8
1669 1137.8
1670 1137.8
1671 1137.8
1672 1137.8
1673 1137.8
1674 1137.8
1675 1137.8
1676 1137.8
1677 1137.8
1678 1137.8
1679 1137.8
1680 1137.8
1680.2 1138
1681 1138.8
1682 1138.8
1683 1138.8
1684 1138.8
1685 1138.8
1686 1138.8
1687 1138.8
1688 1138.8
1689 1138.8
1689.2 1139
1690 1139.8
1691 1139.8
1692 1139.8
1693 1139.8
1694 1139.8
1694.2 1140
1695 1140.8
1696 1140.8
1697 1140.8
1698 1140.8
1699 1140.8
1699.2 1141
1700 1141.8
1701 1141.8
1702 1141.8
1703 1141.8
1704 1141.8
1704.2 1142
1705 1142.8
1706 1142.8
1707 1142.8
1708 1142.8
1709 1142.8
1709.2 1143
1710 1143.8
1711 1143.8
1712 1143.8
1713 1143.8
1714 1143.8
1714.2 1144
1715 1144.8
1716 1144.8
1717 1144.8
1718 1144.8
1719 1144.8
1719.2 1145
1720 1145.8
1721 1145.8
1722 1145.8
1723 1145.8
1724 1145.8
1724.2 1146
1725 1146.8
1726 1146.8
1727 1146.8
1728 1146.8
1729 1146.8
1729.2 1147
1730 1147.8
1731 1147.8
1732 1147.8
1733 1147.8
1734 1147.8
1734.2 1148
1735 1148.8
1736 1148.8
1737 1148.8
1738 1148.8
1739 1148.8
1739.2 1149
1740 1149.8
1741 1149.8
1742 1149.8
1743 1149.8
1743.2 1150
1744 1150.8
1745 1150.8
1746 1150.8
1747 1150.8
1747.2 1151
1748 1151.8
1748.2 1152
1749 1152.8
1750 1152.8
1750.2 1153
1751 1153.8
1752 1153.8
1752.2 1154
1753 1154.8
1754 1154.8
1754.2 1155
1755 1155.8
1756 1155.8
1756.2 1156
1757 1156.8
1758 1156.8
1758.2 1157
1759 1157.8
1760 1157.8
1760.2 1158
1761 1158.8
1762 1158.8
1762.2 1159
1763 1159.8
1764 1159.8
1764.2 1160
1765 1160.8
1766 1160.8
1766.2 1161
1767 1161.8
1767.2 1162
1768 1162.8
1769 1162.8
1769.2 1163
1770 1163.8
1771 1163.8
1771.2 1164
1772 1164.8
1772.2 1165
1772.2 1166
1773 1166.8
1773.2 1167
1773.2 1168
1774 1168.8
1774.8 1168
1775 1167.8
1776 1167.8
1776.8 1167
1777 1166.8
1778 1166.8
1779 1166.8
1780 1166.8
1780.8 1166
1781 1165.8
1782 1165.8
1783 1165.8
1783.8 1165
1784 1164.8
1785 1164.8
1786 1164.8
1786.8 1164
1787 1163.8
1788 1163.8
1789 1163.8
1790 1163.8
1790.8 1163
1791 1162.8
1792 1162.8
1793 1162.8
1794 1162.8
1795 1162.8
1796 1162.8
1797 1162.8
1798 1162.8
1799 1162.8
1800 1162.8
1801 1162.8
1802 1162.8
1803 1162.8
1804 1162.8
1805 1162.8
1806 1162.8
1807 1162.8
1808 1162.8
1809 1162.8
1810 1162.8
1811 1162.8
1812 1162.8
1813 1162.8
1813.2 1163
1814 1163.8
1815 1163.8
1816 1163.8
1817 1163.8
1818 1163.8
1819 1163.8
1820 1163.8
1821 1163.8
1821.2 1164
1822 1164.8
1823 1164.8
1824 1164.8
1825 1164.8
1826 1164.8
1827 1164.8
1828 1164.8
1829 1164.8
1830 1164.8
1831 1164.8
1832 1164.8
1833 1164.8
1833.2 1165
1834 1165.8
1835 1165.8
1836 1165.8
1837 1165.8
1838 1165.8
1839 1165.8
1840 1165.8
1841 1165.8
1842 1165.8
1842.2 1166
1843 1166.8
1844 1166.8
1845 1166.8
1846 1166.8
1847 1166.8
1847.2 1167
1848 1167.8
1849 1167.8
1850 1167.8
1851 1167.8
1852 1167.8
1852.2 1168
1853 1168.8
1854 1168.8
1855 1168.8
1855.2 1169
1856 1169.8
1857 1169.8
1857.2 1170
1858 1170.8
1858.2 1171
1859 1171.8
1860 1171.8
1860.2 1172
1861 1172.8
1861.2 1173
1862 1173.8
1863 1173.8
1863.2 1174
1864 1174.8
1864.2 1175
1865 1175.8
1866 1175.8
1866.2 1176
1867 1176.8
1867.2 1177
1868 1177.8
1869 1177.8
1869.2 1178
1870 1178.8
1870.2 1179
1871 1179.8
1872 1179.8
1872.2 1180
1873 1180.8
1873.2 1181
1874 1181.8
1874.2 1182
1875 1182.8
1875.2 1183
1876 1183.8
1876.2 1184
1877 1184.8
1877.2 1185
1877.2 1186
1878 1186.8
1878.2 1187
1878.2 1188
1879 1188.8
1879.2 1189
1879.2 1190
1879.2 1191
1880 1191.8
1880.2 1192
1880.2 1193
1880.2 1194
1880.2 1195
1880.2 1196
1880.2 1197
1880.2 1198
1880.2 1199
1880.2 1200
1880.2 1201
1880.2 1202
1880.2 1203
1880.2 1204
1880.2 1205
1880 1205.2
1879.2 1206
1879.2 1207
1879.2 1208
1879.2 1209
1879 1209.2
1878.2 1210
1878.2 1211
1878.2 1212
1878.2 1213
1878 1213.2
1877.2 1214
1877.2 1215
1877.2 1216
1877.2 1217
1877 1217.2
1876.2 1218
1876.2 1219
1876 1219.2
1875.2 1220
1875.2 1221
1875 1221.2
1874.2 1222
1874.2 1223
1874 1223.2
1873.2 1224
1873.2 1225
1873 1225.2
1872.2 1226
1872 1226.2
1871.2 1227
1871 1227.2
1870.2 1228
1870 1228.2
1869 1228.2
1868.2 1229
1868 1229.2
1867 1229.2
1866.2 1230
1866 1230.2
1865.2 1231
1865 1231.2
};
\addplot [red]
table {%
485 1179.2
484 1179.2
483 1179.2
482 1179.2
481 1179.2
480 1179.2
479 1179.2
478 1179.2
477 1179.2
476 1179.2
475 1179.2
474 1179.2
473 1179.2
472 1179.2
471 1179.2
470 1179.2
469.8 1179
469 1178.2
468 1178.2
467 1178.2
466 1178.2
465.8 1178
465 1177.2
464 1177.2
463 1177.2
462 1177.2
461.8 1177
461 1176.2
460 1176.2
459 1176.2
458.2 1177
458 1177.2
457 1177.2
456 1177.2
455.8 1177
455 1176.2
454 1176.2
453 1176.2
452.8 1176
452 1175.2
451 1175.2
450.8 1175
450 1174.2
449 1174.2
448.8 1174
448 1173.2
447 1173.2
446 1173.2
445 1173.2
444.8 1173
444 1172.2
443 1172.2
442 1172.2
441 1172.2
440 1172.2
439 1172.2
438 1172.2
437 1172.2
436 1172.2
435 1172.2
434 1172.2
433.8 1172
433 1171.2
432 1171.2
431 1171.2
430 1171.2
429 1171.2
428 1171.2
427 1171.2
426 1171.2
425 1171.2
424 1171.2
423 1171.2
422 1171.2
421 1171.2
420 1171.2
419 1171.2
418.8 1171
418 1170.2
417 1170.2
416 1170.2
415 1170.2
414.8 1170
414 1169.2
413 1169.2
412 1169.2
411.8 1169
411 1168.2
410 1168.2
409 1168.2
408 1168.2
407.8 1168
407 1167.2
406 1167.2
405 1167.2
404.8 1167
404 1166.2
403 1166.2
402 1166.2
401 1166.2
400.8 1166
400 1165.2
399 1165.2
398 1165.2
397.8 1165
397 1164.2
396 1164.2
395 1164.2
394 1164.2
393.8 1164
393.8 1163
393.8 1162
394 1161.8
394.8 1161
394.8 1160
394.8 1159
395 1158.8
395.8 1158
395.8 1157
395.8 1156
395.8 1155
396 1154.8
396.8 1154
396.8 1153
396.8 1152
397 1151.8
398 1151.8
399 1151.8
400 1151.8
401 1151.8
401.8 1151
402 1150.8
403 1150.8
404 1150.8
405 1150.8
406 1150.8
406.8 1150
407 1149.8
408 1149.8
409 1149.8
410 1149.8
411 1149.8
412 1149.8
412.8 1149
413 1148.8
414 1148.8
415 1148.8
416 1148.8
417 1148.8
417.8 1148
418 1147.8
419 1147.8
420 1147.8
420.2 1148
421 1148.8
422 1148.8
423 1148.8
424 1148.8
425 1148.8
426 1148.8
427 1148.8
428 1148.8
429 1148.8
430 1148.8
430.2 1149
431 1149.8
432 1149.8
433 1149.8
434 1149.8
435 1149.8
436 1149.8
437 1149.8
438 1149.8
439 1149.8
439.2 1150
440 1150.8
441 1150.8
442 1150.8
443 1150.8
444 1150.8
445 1150.8
446 1150.8
447 1150.8
448 1150.8
448.2 1151
449 1151.8
450 1151.8
451 1151.8
452 1151.8
453 1151.8
454 1151.8
455 1151.8
456 1151.8
457 1151.8
458 1151.8
459 1151.8
459.2 1152
460 1152.8
461 1152.8
462 1152.8
463 1152.8
464 1152.8
465 1152.8
465.2 1153
466 1153.8
467 1153.8
468 1153.8
468.2 1154
469 1154.8
470 1154.8
471 1154.8
471.2 1155
472 1155.8
473 1155.8
474 1155.8
474.2 1156
475 1156.8
476 1156.8
477 1156.8
478 1156.8
478.2 1157
479 1157.8
480 1157.8
481 1157.8
482 1157.8
483 1157.8
483.2 1158
484 1158.8
485 1158.8
486 1158.8
486.2 1159
487 1159.8
488 1159.8
488.2 1160
489 1160.8
489.2 1161
490 1161.8
491 1161.8
491.2 1162
492 1162.8
493 1162.8
493.2 1163
494 1163.8
494.2 1164
494.2 1165
495 1165.8
495.2 1166
495.2 1167
495.2 1168
495.2 1169
495.2 1170
495.2 1171
495.2 1172
495 1172.2
494.2 1173
494.2 1174
494 1174.2
493.2 1175
493 1175.2
492.2 1176
492 1176.2
491.2 1177
491 1177.2
490 1177.2
489.2 1178
489 1178.2
488 1178.2
487 1178.2
486 1178.2
485.2 1179
485 1179.2
};
\addplot [red]
table {%
882 1203.2
881 1203.2
880 1203.2
879 1203.2
878 1203.2
877 1203.2
876 1203.2
875 1203.2
874 1203.2
873 1203.2
872 1203.2
871 1203.2
870.8 1203
870 1202.2
869 1202.2
868.8 1202
868 1201.2
867.8 1201
867 1200.2
866.8 1200
866.8 1199
866.8 1198
866.8 1197
866.8 1196
866.8 1195
866.8 1194
867 1193.8
867.8 1193
867.8 1192
868 1191.8
868.8 1191
868.8 1190
869 1189.8
869.8 1189
870 1188.8
870.8 1188
871 1187.8
871.8 1187
872 1186.8
872.8 1186
873 1185.8
873.8 1185
874 1184.8
874.8 1184
875 1183.8
876 1183.8
876.8 1183
877 1182.8
877.8 1182
878 1181.8
879 1181.8
879.8 1181
880 1180.8
880.8 1180
881 1179.8
881.8 1179
882 1178.8
882.8 1178
883 1177.8
883.8 1177
884 1176.8
884.8 1176
885 1175.8
886 1175.8
886.8 1175
887 1174.8
888 1174.8
888.8 1174
889 1173.8
890 1173.8
890.8 1173
891 1172.8
892 1172.8
892.8 1172
893 1171.8
894 1171.8
894.8 1171
895 1170.8
895.8 1170
896 1169.8
897 1169.8
897.8 1169
898 1168.8
899 1168.8
900 1168.8
900.8 1168
901 1167.8
902 1167.8
903 1167.8
903.8 1167
904 1166.8
905 1166.8
906 1166.8
907 1166.8
908 1166.8
908.8 1166
909 1165.8
910 1165.8
911 1165.8
912 1165.8
912.2 1166
913 1166.8
914 1166.8
915 1166.8
916 1166.8
917 1166.8
917.2 1167
918 1167.8
918.2 1168
919 1168.8
919.2 1169
919.2 1170
920 1170.8
920.2 1171
920.2 1172
920.2 1173
920 1173.2
919.2 1174
919 1174.2
918.2 1175
918.2 1176
918 1176.2
917.2 1177
917 1177.2
916.2 1178
916.2 1179
916 1179.2
915 1179.2
914.2 1180
914 1180.2
913.2 1181
913 1181.2
912.2 1182
912 1182.2
911.2 1183
911 1183.2
910.2 1184
910 1184.2
909 1184.2
908.2 1185
908 1185.2
907.2 1186
907 1186.2
906.2 1187
906.2 1188
906 1188.2
905.2 1189
905.2 1190
905 1190.2
904.2 1191
904.2 1192
904 1192.2
903.2 1193
903 1193.2
902.2 1194
902 1194.2
901.2 1195
901 1195.2
900.2 1196
900 1196.2
899.2 1197
899 1197.2
898.2 1198
898 1198.2
897 1198.2
896.2 1199
896 1199.2
895 1199.2
894 1199.2
893.2 1200
893 1200.2
892 1200.2
891 1200.2
890 1200.2
889.2 1201
889 1201.2
888 1201.2
887 1201.2
886.2 1202
886 1202.2
885 1202.2
884 1202.2
883 1202.2
882.2 1203
882 1203.2
};
\addplot [red]
table {%
1027 1241.2
1026 1241.2
1025 1241.2
1024 1241.2
1023 1241.2
1022 1241.2
1021 1241.2
1020 1241.2
1019.8 1241
1019 1240.2
1018 1240.2
1017 1240.2
1016 1240.2
1015 1240.2
1014 1240.2
1013 1240.2
1012 1240.2
1011.8 1240
1011 1239.2
1010 1239.2
1009 1239.2
1008 1239.2
1007 1239.2
1006 1239.2
1005 1239.2
1004.8 1239
1004 1238.2
1003 1238.2
1002 1238.2
1001 1238.2
1000 1238.2
999.8 1238
999 1237.2
998 1237.2
997 1237.2
996.8 1237
996 1236.2
995 1236.2
994 1236.2
993 1236.2
992.8 1236
992 1235.2
991 1235.2
990 1235.2
989.8 1235
989 1234.2
988 1234.2
987 1234.2
986 1234.2
985.8 1234
985 1233.2
984 1233.2
983 1233.2
982.8 1233
982 1232.2
981 1232.2
980 1232.2
979.8 1232
979 1231.2
978 1231.2
977.8 1231
977 1230.2
976 1230.2
975.8 1230
975 1229.2
974 1229.2
973.8 1229
973 1228.2
972 1228.2
971.8 1228
971 1227.2
970 1227.2
969.8 1227
969 1226.2
968 1226.2
967.8 1226
967 1225.2
966.8 1225
966 1224.2
965 1224.2
964.8 1224
964 1223.2
963 1223.2
962.8 1223
962 1222.2
961 1222.2
960.8 1222
960 1221.2
959 1221.2
958.8 1221
958 1220.2
957.8 1220
957 1219.2
956 1219.2
955.8 1219
955 1218.2
954 1218.2
953.8 1218
953 1217.2
952 1217.2
951.8 1217
951 1216.2
950 1216.2
949.8 1216
949 1215.2
948 1215.2
947.8 1215
947 1214.2
946.8 1214
946 1213.2
945 1213.2
944.8 1213
944 1212.2
943.8 1212
943 1211.2
942.8 1211
942 1210.2
941 1210.2
940.8 1210
940.8 1209
940 1208.2
939.8 1208
939.8 1207
939 1206.2
938.8 1206
938.8 1205
938 1204.2
937.8 1204
937.8 1203
937.8 1202
937.8 1201
937.8 1200
937.8 1199
937.8 1198
937.8 1197
937.8 1196
937.8 1195
937.8 1194
938 1193.8
938.8 1193
938.8 1192
938.8 1191
938 1190.2
937.8 1190
937.8 1189
937.8 1188
937.8 1187
937.8 1186
937 1185.2
936.8 1185
936.8 1184
936.8 1183
936.8 1182
936.8 1181
936.8 1180
936.8 1179
936.8 1178
937 1177.8
937.8 1177
937.8 1176
937.8 1175
937.8 1174
938 1173.8
938.8 1173
939 1172.8
940 1172.8
940.8 1172
941 1171.8
942 1171.8
942.8 1171
943 1170.8
944 1170.8
944.8 1170
945 1169.8
945.2 1170
946 1170.8
947 1170.8
948 1170.8
948.2 1171
949 1171.8
950 1171.8
951 1171.8
952 1171.8
952.2 1172
953 1172.8
954 1172.8
955 1172.8
955.2 1173
956 1173.8
957 1173.8
957.2 1174
958 1174.8
959 1174.8
959.2 1175
960 1175.8
961 1175.8
961.2 1176
962 1176.8
963 1176.8
964 1176.8
964.2 1177
965 1177.8
966 1177.8
967 1177.8
967.2 1178
968 1178.8
969 1178.8
969.2 1179
970 1179.8
971 1179.8
972 1179.8
972.2 1180
973 1180.8
974 1180.8
975 1180.8
975.2 1181
976 1181.8
977 1181.8
978 1181.8
978.2 1182
979 1182.8
980 1182.8
980.2 1183
981 1183.8
981.2 1184
982 1184.8
983 1184.8
983.2 1185
984 1185.8
985 1185.8
985.2 1186
986 1186.8
986.2 1187
987 1187.8
988 1187.8
988.2 1188
989 1188.8
990 1188.8
990.2 1189
991 1189.8
991.2 1190
992 1190.8
993 1190.8
993.2 1191
994 1191.8
995 1191.8
995.2 1192
996 1192.8
997 1192.8
997.2 1193
998 1193.8
999 1193.8
1000 1193.8
1000.2 1194
1001 1194.8
1002 1194.8
1003 1194.8
1003.2 1195
1004 1195.8
1005 1195.8
1006 1195.8
1006.2 1196
1007 1196.8
1008 1196.8
1009 1196.8
1009.2 1197
1010 1197.8
1011 1197.8
1012 1197.8
1012.2 1198
1013 1198.8
1014 1198.8
1015 1198.8
1015.2 1199
1016 1199.8
1017 1199.8
1018 1199.8
1018.2 1200
1019 1200.8
1020 1200.8
1021 1200.8
1021.2 1201
1022 1201.8
1023 1201.8
1024 1201.8
1024.2 1202
1025 1202.8
1026 1202.8
1027 1202.8
1027.2 1203
1028 1203.8
1029 1203.8
1030 1203.8
1030.2 1204
1031 1204.8
1031.2 1205
1032 1205.8
1032.2 1206
1033 1206.8
1034 1206.8
1034.2 1207
1035 1207.8
1035.2 1208
1036 1208.8
1036.2 1209
1037 1209.8
1037.2 1210
1038 1210.8
1038.2 1211
1039 1211.8
1039.2 1212
1040 1212.8
1040.2 1213
1041 1213.8
1042 1213.8
1042.2 1214
1043 1214.8
1043.2 1215
1044 1215.8
1044.2 1216
1045 1216.8
1045.2 1217
1046 1217.8
1046.2 1218
1047 1218.8
1047.2 1219
1048 1219.8
1048.2 1220
1048.2 1221
1048.2 1222
1048.2 1223
1048.2 1224
1048.2 1225
1048.2 1226
1048.2 1227
1048 1227.2
1047.2 1228
1047.2 1229
1047.2 1230
1047.2 1231
1047.2 1232
1047.2 1233
1047.2 1234
1047.2 1235
1047 1235.2
1046.2 1236
1046 1236.2
1045.2 1237
1045 1237.2
1044 1237.2
1043 1237.2
1042 1237.2
1041 1237.2
1040.2 1238
1040 1238.2
1039 1238.2
1038 1238.2
1037 1238.2
1036.2 1239
1036 1239.2
1035 1239.2
1034 1239.2
1033 1239.2
1032 1239.2
1031.2 1240
1031 1240.2
1030 1240.2
1029 1240.2
1028 1240.2
1027.2 1241
1027 1241.2
};
\addplot [red]
table {%
876 1257.2
875 1257.2
874 1257.2
873 1257.2
872 1257.2
871.8 1257
871 1256.2
870 1256.2
869 1256.2
868 1256.2
867 1256.2
866 1256.2
865 1256.2
864 1256.2
863 1256.2
862 1256.2
861 1256.2
860 1256.2
859 1256.2
858 1256.2
857 1256.2
856 1256.2
855 1256.2
854 1256.2
853 1256.2
852 1256.2
851.8 1256
851 1255.2
850 1255.2
849 1255.2
848 1255.2
847 1255.2
846 1255.2
845 1255.2
844 1255.2
843 1255.2
842 1255.2
841 1255.2
840 1255.2
839 1255.2
838 1255.2
837.8 1255
837 1254.2
836 1254.2
835 1254.2
834 1254.2
833 1254.2
832.8 1254
832 1253.2
831 1253.2
830 1253.2
829 1253.2
828 1253.2
827.8 1253
827 1252.2
826 1252.2
825 1252.2
824 1252.2
823.8 1252
823 1251.2
822 1251.2
821 1251.2
820 1251.2
819 1251.2
818.8 1251
818 1250.2
817 1250.2
816 1250.2
815 1250.2
814 1250.2
813.8 1250
813.8 1249
813 1248.2
812.8 1248
812 1247.2
811 1247.2
810 1247.2
809 1247.2
808 1247.2
807 1247.2
806 1247.2
805 1247.2
804 1247.2
803 1247.2
802 1247.2
801 1247.2
800 1247.2
799 1247.2
798 1247.2
797 1247.2
796 1247.2
795 1247.2
794 1247.2
793 1247.2
792 1247.2
791 1247.2
790 1247.2
789 1247.2
788 1247.2
787 1247.2
786.8 1247
786 1246.2
785 1246.2
784 1246.2
783 1246.2
782 1246.2
781 1246.2
780.8 1246
780 1245.2
779 1245.2
778 1245.2
777 1245.2
776 1245.2
775 1245.2
774 1245.2
773 1245.2
772.8 1245
772 1244.2
771 1244.2
770 1244.2
769 1244.2
768 1244.2
767 1244.2
766 1244.2
765 1244.2
764.8 1244
764 1243.2
763 1243.2
762 1243.2
761 1243.2
760 1243.2
759.8 1243
759 1242.2
758 1242.2
757 1242.2
756 1242.2
755.8 1242
755 1241.2
754 1241.2
753 1241.2
752 1241.2
751.8 1241
751 1240.2
750 1240.2
749 1240.2
748 1240.2
747 1240.2
746.8 1240
746 1239.2
745 1239.2
744 1239.2
743 1239.2
742 1239.2
741.8 1239
741 1238.2
740 1238.2
739 1238.2
738.8 1238
738 1237.2
737 1237.2
736 1237.2
735.8 1237
735 1236.2
734 1236.2
733.8 1236
733 1235.2
732.8 1235
732 1234.2
731 1234.2
730.8 1234
730 1233.2
729 1233.2
728.8 1233
728 1232.2
727 1232.2
726 1232.2
725.8 1232
725 1231.2
724 1231.2
723 1231.2
722.8 1231
722 1230.2
721 1230.2
720 1230.2
719 1230.2
718.8 1230
718 1229.2
717 1229.2
716 1229.2
715 1229.2
714.8 1229
714 1228.2
713 1228.2
712 1228.2
711.8 1228
711 1227.2
710 1227.2
709 1227.2
708.8 1227
708 1226.2
707 1226.2
706 1226.2
705.8 1226
705 1225.2
704 1225.2
703 1225.2
702.8 1225
702 1224.2
701 1224.2
700 1224.2
699.8 1224
699 1223.2
698 1223.2
697 1223.2
696.8 1223
696 1222.2
695 1222.2
694 1222.2
693 1222.2
692.8 1222
692.8 1221
692 1220.2
691.8 1220
691.8 1219
691 1218.2
690.8 1218
690.8 1217
690 1216.2
689.8 1216
689.8 1215
689 1214.2
688 1214.2
687.2 1215
687 1215.2
686.2 1216
686 1216.2
685 1216.2
684.2 1217
684 1217.2
683 1217.2
682.2 1218
682 1218.2
681 1218.2
680 1218.2
679 1218.2
678 1218.2
677 1218.2
676 1218.2
675 1218.2
674 1218.2
673 1218.2
672 1218.2
671 1218.2
670 1218.2
669.8 1218
669 1217.2
668 1217.2
667 1217.2
666 1217.2
665 1217.2
664.8 1217
664 1216.2
663 1216.2
662 1216.2
661.8 1216
661 1215.2
660 1215.2
659 1215.2
658.8 1215
658 1214.2
657 1214.2
656 1214.2
655 1214.2
654.8 1214
654 1213.2
653 1213.2
652 1213.2
651.8 1213
651 1212.2
650 1212.2
649 1212.2
648.8 1212
648 1211.2
647 1211.2
646.8 1211
646 1210.2
645 1210.2
644 1210.2
643.8 1210
643 1209.2
642 1209.2
641 1209.2
640.8 1209
640 1208.2
639 1208.2
638.8 1208
638 1207.2
637 1207.2
636.8 1207
636 1206.2
635 1206.2
634 1206.2
633.8 1206
633 1205.2
632.8 1205
632.8 1204
632 1203.2
631.8 1203
631 1202.2
630.8 1202
630 1201.2
629.8 1201
629 1200.2
628.8 1200
628 1199.2
627.8 1199
627 1198.2
626.8 1198
626.8 1197
626.8 1196
626 1195.2
625.8 1195
625.8 1194
625.8 1193
625 1192.2
624.8 1192
624.8 1191
624.8 1190
624.8 1189
624 1188.2
623.8 1188
624 1187.8
624.8 1187
624.8 1186
624.8 1185
624.8 1184
625 1183.8
625.8 1183
625.8 1182
625.8 1181
626 1180.8
626.8 1180
626.8 1179
627 1178.8
627.8 1178
628 1177.8
628.8 1177
629 1176.8
629.8 1176
630 1175.8
631 1175.8
631.8 1175
632 1174.8
632.8 1174
633 1173.8
634 1173.8
634.8 1173
635 1172.8
636 1172.8
637 1172.8
637.8 1172
638 1171.8
639 1171.8
640 1171.8
641 1171.8
642 1171.8
643 1171.8
644 1171.8
645 1171.8
646 1171.8
647 1171.8
648 1171.8
649 1171.8
650 1171.8
651 1171.8
651.2 1172
652 1172.8
653 1172.8
654 1172.8
655 1172.8
655.2 1173
656 1173.8
657 1173.8
658 1173.8
659 1173.8
659.2 1174
660 1174.8
661 1174.8
662 1174.8
663 1174.8
663.2 1175
664 1175.8
665 1175.8
666 1175.8
667 1175.8
667.2 1176
668 1176.8
669 1176.8
670 1176.8
671 1176.8
671.2 1177
672 1177.8
673 1177.8
674 1177.8
674.2 1178
675 1178.8
676 1178.8
676.2 1179
677 1179.8
678 1179.8
679 1179.8
679.2 1180
680 1180.8
681 1180.8
682 1180.8
682.2 1181
683 1181.8
684 1181.8
685 1181.8
685.2 1182
686 1182.8
687 1182.8
687.2 1183
688 1183.8
689 1183.8
689.2 1184
690 1184.8
690.2 1185
691 1185.8
691.2 1186
692 1186.8
692.2 1187
693 1187.8
693.2 1188
693.2 1189
694 1189.8
694.2 1190
694.2 1191
694.2 1192
695 1192.8
695.2 1193
695.2 1194
695.2 1195
695.2 1196
695.2 1197
696 1197.8
696.2 1198
696 1198.2
695.2 1199
695.2 1200
695.2 1201
696 1201.8
696.8 1201
697 1200.8
698 1200.8
698.8 1200
699 1199.8
700 1199.8
701 1199.8
702 1199.8
703 1199.8
704 1199.8
705 1199.8
706 1199.8
707 1199.8
708 1199.8
709 1199.8
710 1199.8
711 1199.8
712 1199.8
713 1199.8
714 1199.8
715 1199.8
716 1199.8
717 1199.8
718 1199.8
719 1199.8
720 1199.8
720.2 1200
721 1200.8
722 1200.8
723 1200.8
724 1200.8
725 1200.8
726 1200.8
726.2 1201
727 1201.8
728 1201.8
729 1201.8
730 1201.8
731 1201.8
731.2 1202
732 1202.8
733 1202.8
734 1202.8
735 1202.8
736 1202.8
736.2 1203
737 1203.8
738 1203.8
739 1203.8
740 1203.8
741 1203.8
742 1203.8
742.2 1204
743 1204.8
744 1204.8
745 1204.8
746 1204.8
747 1204.8
748 1204.8
749 1204.8
750 1204.8
751 1204.8
752 1204.8
752.2 1205
753 1205.8
754 1205.8
755 1205.8
756 1205.8
757 1205.8
758 1205.8
759 1205.8
760 1205.8
761 1205.8
761.2 1206
762 1206.8
763 1206.8
764 1206.8
764.2 1207
765 1207.8
766 1207.8
767 1207.8
768 1207.8
768.2 1208
769 1208.8
770 1208.8
771 1208.8
771.2 1209
772 1209.8
773 1209.8
774 1209.8
775 1209.8
775.2 1210
776 1210.8
777 1210.8
778 1210.8
779 1210.8
779.2 1211
780 1211.8
781 1211.8
782 1211.8
783 1211.8
783.2 1212
784 1212.8
785 1212.8
786 1212.8
787 1212.8
787.2 1213
788 1213.8
789 1213.8
790 1213.8
791 1213.8
792 1213.8
792.2 1214
793 1214.8
794 1214.8
794.2 1215
795 1215.8
796 1215.8
797 1215.8
797.2 1216
798 1216.8
799 1216.8
799.2 1217
800 1217.8
801 1217.8
802 1217.8
802.2 1218
803 1218.8
804 1218.8
805 1218.8
805.2 1219
806 1219.8
807 1219.8
807.2 1220
808 1220.8
809 1220.8
810 1220.8
810.2 1221
811 1221.8
812 1221.8
813 1221.8
813.2 1222
813.2 1223
814 1223.8
814.2 1224
815 1224.8
815.2 1225
816 1225.8
816.2 1226
816.2 1227
817 1227.8
817.2 1228
818 1228.8
818.2 1229
819 1229.8
820 1229.8
821 1229.8
822 1229.8
823 1229.8
824 1229.8
825 1229.8
826 1229.8
827 1229.8
828 1229.8
829 1229.8
830 1229.8
831 1229.8
832 1229.8
833 1229.8
834 1229.8
834.2 1230
835 1230.8
836 1230.8
837 1230.8
838 1230.8
839 1230.8
840 1230.8
841 1230.8
842 1230.8
842.2 1231
843 1231.8
844 1231.8
845 1231.8
846 1231.8
846.2 1232
847 1232.8
848 1232.8
849 1232.8
849.2 1233
850 1233.8
851 1233.8
852 1233.8
852.2 1234
853 1234.8
854 1234.8
855 1234.8
855.2 1235
856 1235.8
857 1235.8
858 1235.8
858.2 1236
859 1236.8
860 1236.8
860.2 1237
861 1237.8
862 1237.8
863 1237.8
863.2 1238
864 1238.8
865 1238.8
865.2 1239
866 1239.8
867 1239.8
868 1239.8
869 1239.8
870 1239.8
870.2 1240
871 1240.8
872 1240.8
872.2 1241
873 1241.8
874 1241.8
874.2 1242
875 1242.8
876 1242.8
876.2 1243
877 1243.8
877.2 1244
878 1244.8
878.2 1245
878.2 1246
878.2 1247
879 1247.8
879.2 1248
879.2 1249
879.2 1250
879.2 1251
879.2 1252
879.2 1253
879 1253.2
878.2 1254
878.2 1255
878 1255.2
877.2 1256
877 1256.2
876.2 1257
876 1257.2
};
\addplot [red]
table {%
2083 1224.2
2082 1224.2
2081 1224.2
2080 1224.2
2079 1224.2
2078 1224.2
2077 1224.2
2076 1224.2
2075 1224.2
2074 1224.2
2073 1224.2
2072.8 1224
2072 1223.2
2071 1223.2
2070 1223.2
2069 1223.2
2068 1223.2
2067.8 1223
2067 1222.2
2066 1222.2
2065 1222.2
2064 1222.2
2063 1222.2
2062.8 1222
2062 1221.2
2061 1221.2
2060 1221.2
2059 1221.2
2058 1221.2
2057 1221.2
2056 1221.2
2055 1221.2
2054 1221.2
2053 1221.2
2052 1221.2
2051.8 1221
2051 1220.2
2050 1220.2
2049 1220.2
2048 1220.2
2047.8 1220
2047 1219.2
2046 1219.2
2045 1219.2
2044 1219.2
2043 1219.2
2042.8 1219
2042 1218.2
2041 1218.2
2040 1218.2
2039.8 1218
2039 1217.2
2038.8 1217
2038 1216.2
2037 1216.2
2036.8 1216
2036 1215.2
2035.8 1215
2035 1214.2
2034.8 1214
2034 1213.2
2033 1213.2
2032.8 1213
2032 1212.2
2031.8 1212
2031 1211.2
2030.8 1211
2030.8 1210
2030.8 1209
2030 1208.2
2029.8 1208
2029.8 1207
2029 1206.2
2028.8 1206
2028.8 1205
2028.8 1204
2028 1203.2
2027.8 1203
2027.8 1202
2027.8 1201
2027.8 1200
2027.8 1199
2027.8 1198
2027.8 1197
2027.8 1196
2027.8 1195
2027.8 1194
2027.8 1193
2027.8 1192
2027.8 1191
2028 1190.8
2028.8 1190
2029 1189.8
2029.8 1189
2030 1188.8
2030.8 1188
2030.8 1187
2031 1186.8
2031.8 1186
2032 1185.8
2032.8 1185
2033 1184.8
2033.8 1184
2034 1183.8
2034.8 1183
2035 1182.8
2036 1182.8
2037 1182.8
2038 1182.8
2038.8 1182
2039 1181.8
2040 1181.8
2041 1181.8
2042 1181.8
2042.8 1181
2043 1180.8
2044 1180.8
2045 1180.8
2046 1180.8
2046.8 1180
2047 1179.8
2048 1179.8
2049 1179.8
2050 1179.8
2051 1179.8
2052 1179.8
2053 1179.8
2054 1179.8
2055 1179.8
2056 1179.8
2056.8 1179
2057 1178.8
2058 1178.8
2059 1178.8
2060 1178.8
2061 1178.8
2062 1178.8
2063 1178.8
2064 1178.8
2065 1178.8
2066 1178.8
2067 1178.8
2068 1178.8
2069 1178.8
2070 1178.8
2071 1178.8
2072 1178.8
2073 1178.8
2074 1178.8
2075 1178.8
2076 1178.8
2076.2 1179
2077 1179.8
2078 1179.8
2079 1179.8
2080 1179.8
2081 1179.8
2082 1179.8
2083 1179.8
2083.2 1180
2084 1180.8
2085 1180.8
2086 1180.8
2087 1180.8
2088 1180.8
2089 1180.8
2090 1180.8
2090.2 1181
2091 1181.8
2092 1181.8
2093 1181.8
2094 1181.8
2094.2 1182
2095 1182.8
2096 1182.8
2097 1182.8
2097.2 1183
2098 1183.8
2099 1183.8
2099.2 1184
2100 1184.8
2101 1184.8
2101.2 1185
2102 1185.8
2102.2 1186
2103 1186.8
2104 1186.8
2104.2 1187
2105 1187.8
2106 1187.8
2106.2 1188
2107 1188.8
2107.2 1189
2108 1189.8
2109 1189.8
2109.2 1190
2110 1190.8
2111 1190.8
2112 1190.8
2112.2 1191
2113 1191.8
2114 1191.8
2115 1191.8
2116 1191.8
2116.2 1192
2117 1192.8
2118 1192.8
2119 1192.8
2119.2 1193
2120 1193.8
2121 1193.8
2121.2 1194
2122 1194.8
2122.2 1195
2123 1195.8
2123.2 1196
2124 1196.8
2124.2 1197
2125 1197.8
2125.2 1198
2126 1198.8
2126.2 1199
2126.2 1200
2127 1200.8
2127.2 1201
2128 1201.8
2128.2 1202
2128 1202.2
2127.2 1203
2127 1203.2
2126.2 1204
2126 1204.2
2125 1204.2
2124.2 1205
2124 1205.2
2123.2 1206
2123 1206.2
2122.2 1207
2122 1207.2
2121.2 1208
2121 1208.2
2120 1208.2
2119.2 1209
2119 1209.2
2118.2 1210
2118 1210.2
2117 1210.2
2116 1210.2
2115 1210.2
2114.2 1211
2114 1211.2
2113 1211.2
2112 1211.2
2111 1211.2
2110.2 1212
2110 1212.2
2109 1212.2
2108 1212.2
2107 1212.2
2106.2 1213
2106 1213.2
2105.2 1214
2105 1214.2
2104 1214.2
2103.2 1215
2103 1215.2
2102.2 1216
2102 1216.2
2101 1216.2
2100.2 1217
2100 1217.2
2099 1217.2
2098.2 1218
2098 1218.2
2097.2 1219
2097 1219.2
2096 1219.2
2095 1219.2
2094.2 1220
2094 1220.2
2093 1220.2
2092.2 1221
2092 1221.2
2091 1221.2
2090 1221.2
2089.2 1222
2089 1222.2
2088 1222.2
2087 1222.2
2086.2 1223
2086 1223.2
2085 1223.2
2084 1223.2
2083.2 1224
2083 1224.2
};
\addplot [red]
table {%
925 1212.2
924 1212.2
923 1212.2
922 1212.2
921 1212.2
920 1212.2
919 1212.2
918.8 1212
918 1211.2
917 1211.2
916 1211.2
915.8 1211
915 1210.2
914 1210.2
913 1210.2
912.8 1210
912 1209.2
911 1209.2
910.8 1209
910 1208.2
909.8 1208
909 1207.2
908.8 1207
908 1206.2
907.8 1206
907 1205.2
906.8 1205
906.8 1204
906.8 1203
906.8 1202
906.8 1201
906.8 1200
906.8 1199
907 1198.8
907.8 1198
907.8 1197
907.8 1196
907.8 1195
908 1194.8
908.8 1194
908.8 1193
909 1192.8
909.8 1192
909.8 1191
909.8 1190
909.8 1189
910 1188.8
910.8 1188
911 1187.8
911.8 1187
912 1186.8
912.8 1186
913 1185.8
913.8 1185
914 1184.8
914.8 1184
915 1183.8
916 1183.8
916.8 1183
917 1182.8
918 1182.8
919 1182.8
919.8 1182
920 1181.8
921 1181.8
922 1181.8
923 1181.8
924 1181.8
925 1181.8
926 1181.8
926.2 1182
927 1182.8
928 1182.8
928.2 1183
929 1183.8
929.2 1184
930 1184.8
930.2 1185
931 1185.8
931.2 1186
931.2 1187
931.2 1188
932 1188.8
932.2 1189
932.2 1190
932.2 1191
932.2 1192
932.2 1193
932.2 1194
932.2 1195
932.2 1196
932.2 1197
932.2 1198
932.2 1199
932.2 1200
932.2 1201
932.2 1202
932 1202.2
931.2 1203
931.2 1204
931.2 1205
931.2 1206
931 1206.2
930.2 1207
930.2 1208
930 1208.2
929.2 1209
929.2 1210
929 1210.2
928 1210.2
927.2 1211
927 1211.2
926 1211.2
925.2 1212
925 1212.2
};
\addplot [red]
table {%
2217 1251.2
2216 1251.2
2215 1251.2
2214 1251.2
2213 1251.2
2212 1251.2
2211 1251.2
2210 1251.2
2209 1251.2
2208 1251.2
2207 1251.2
2206 1251.2
2205 1251.2
2204 1251.2
2203 1251.2
2202 1251.2
2201.8 1251
2201 1250.2
2200 1250.2
2199 1250.2
2198 1250.2
2197 1250.2
2196.8 1250
2196 1249.2
2195 1249.2
2194.8 1249
2194 1248.2
2193 1248.2
2192.8 1248
2192 1247.2
2191.8 1247
2191 1246.2
2190 1246.2
2189.8 1246
2189 1245.2
2188 1245.2
2187.8 1245
2187 1244.2
2186.8 1244
2186 1243.2
2185 1243.2
2184.8 1243
2184 1242.2
2183 1242.2
2182.8 1242
2182 1241.2
2181 1241.2
2180 1241.2
2179.8 1241
2179 1240.2
2178 1240.2
2177.8 1240
2177 1239.2
2176 1239.2
2175 1239.2
2174 1239.2
2173 1239.2
2172 1239.2
2171.8 1239
2171 1238.2
2170 1238.2
2169 1238.2
2168 1238.2
2167 1238.2
2166.8 1238
2166 1237.2
2165 1237.2
2164 1237.2
2163 1237.2
2162 1237.2
2161.8 1237
2161 1236.2
2160 1236.2
2159 1236.2
2158.8 1236
2158 1235.2
2157 1235.2
2156 1235.2
2155.8 1235
2155 1234.2
2154 1234.2
2153 1234.2
2152.8 1234
2152 1233.2
2151 1233.2
2150 1233.2
2149.8 1233
2149 1232.2
2148 1232.2
2147 1232.2
2146.8 1232
2146 1231.2
2145 1231.2
2144 1231.2
2143.8 1231
2143 1230.2
2142 1230.2
2141 1230.2
2140.8 1230
2140 1229.2
2139 1229.2
2138 1229.2
2137 1229.2
2136.8 1229
2136 1228.2
2135 1228.2
2134 1228.2
2133.8 1228
2133.8 1227
2133 1226.2
2132.8 1226
2132.8 1225
2132 1224.2
2131.8 1224
2131.8 1223
2131 1222.2
2130.8 1222
2130 1221.2
2129.8 1221
2129.8 1220
2129 1219.2
2128.8 1219
2128.8 1218
2128 1217.2
2127.8 1217
2127.8 1216
2127 1215.2
2126.8 1215
2126.8 1214
2126 1213.2
2125.8 1213
2125 1212.2
2124.8 1212
2125 1211.8
2126 1211.8
2126.8 1211
2127 1210.8
2127.8 1210
2128 1209.8
2128.8 1209
2129 1208.8
2129.8 1208
2130 1207.8
2130.8 1207
2131 1206.8
2132 1206.8
2132.8 1206
2133 1205.8
2133.8 1205
2134 1204.8
2134.8 1204
2135 1203.8
2135.8 1203
2136 1202.8
2137 1202.8
2137.8 1202
2138 1201.8
2138.8 1201
2139 1200.8
2140 1200.8
2141 1200.8
2142 1200.8
2143 1200.8
2144 1200.8
2145 1200.8
2146 1200.8
2147 1200.8
2148 1200.8
2149 1200.8
2150 1200.8
2151 1200.8
2152 1200.8
2153 1200.8
2154 1200.8
2155 1200.8
2156 1200.8
2157 1200.8
2158 1200.8
2159 1200.8
2160 1200.8
2161 1200.8
2162 1200.8
2163 1200.8
2164 1200.8
2165 1200.8
2166 1200.8
2167 1200.8
2168 1200.8
2168.2 1201
2169 1201.8
2170 1201.8
2171 1201.8
2172 1201.8
2172.2 1202
2173 1202.8
2174 1202.8
2175 1202.8
2176 1202.8
2177 1202.8
2177.2 1203
2178 1203.8
2179 1203.8
2180 1203.8
2181 1203.8
2181.2 1204
2182 1204.8
2183 1204.8
2184 1204.8
2185 1204.8
2186 1204.8
2186.2 1205
2187 1205.8
2188 1205.8
2189 1205.8
2190 1205.8
2191 1205.8
2191.2 1206
2192 1206.8
2193 1206.8
2194 1206.8
2195 1206.8
2196 1206.8
2196.2 1207
2197 1207.8
2198 1207.8
2199 1207.8
2199.2 1208
2200 1208.8
2201 1208.8
2202 1208.8
2203 1208.8
2203.2 1209
2204 1209.8
2205 1209.8
2206 1209.8
2207 1209.8
2207.2 1210
2208 1210.8
2209 1210.8
2209.2 1211
2210 1211.8
2211 1211.8
2212 1211.8
2213 1211.8
2213.2 1212
2214 1212.8
2215 1212.8
2216 1212.8
2216.2 1213
2217 1213.8
2218 1213.8
2219 1213.8
2219.2 1214
2220 1214.8
2221 1214.8
2221.2 1215
2222 1215.8
2223 1215.8
2224 1215.8
2224.2 1216
2225 1216.8
2225.2 1217
2226 1217.8
2227 1217.8
2227.2 1218
2228 1218.8
2228.2 1219
2229 1219.8
2229.2 1220
2230 1220.8
2230.2 1221
2231 1221.8
2231.2 1222
2231.2 1223
2232 1223.8
2232.2 1224
2232.2 1225
2233 1225.8
2233.2 1226
2233.2 1227
2234 1227.8
2234.2 1228
2234.2 1229
2234.2 1230
2234.2 1231
2234.2 1232
2235 1232.8
2235.2 1233
2235.2 1234
2235.2 1235
2235.2 1236
2235 1236.2
2234.2 1237
2234.2 1238
2234.2 1239
2234.2 1240
2234.2 1241
2234 1241.2
2233.2 1242
2233 1242.2
2232.2 1243
2232.2 1244
2232 1244.2
2231.2 1245
2231 1245.2
2230 1245.2
2229.2 1246
2229 1246.2
2228 1246.2
2227.2 1247
2227 1247.2
2226.2 1248
2226 1248.2
2225 1248.2
2224 1248.2
2223.2 1249
2223 1249.2
2222 1249.2
2221.2 1250
2221 1250.2
2220 1250.2
2219 1250.2
2218 1250.2
2217.2 1251
2217 1251.2
};
\addplot [red]
table {%
1957 1239.2
1956 1239.2
1955.8 1239
1955 1238.2
1954 1238.2
1953 1238.2
1952.8 1238
1952 1237.2
1951.8 1237
1951.8 1236
1951 1235.2
1950.8 1235
1950.8 1234
1950.8 1233
1950.8 1232
1950.8 1231
1950.8 1230
1951 1229.8
1951.8 1229
1951.8 1228
1951.8 1227
1952 1226.8
1952.8 1226
1953 1225.8
1953.8 1225
1954 1224.8
1954.8 1224
1955 1223.8
1955.8 1223
1956 1222.8
1956.8 1222
1957 1221.8
1958 1221.8
1959 1221.8
1960 1221.8
1961 1221.8
1962 1221.8
1963 1221.8
1964 1221.8
1964.2 1222
1965 1222.8
1965.2 1223
1966 1223.8
1966.2 1224
1967 1224.8
1967.2 1225
1968 1225.8
1968.2 1226
1968.2 1227
1969 1227.8
1969.2 1228
1969.2 1229
1969.2 1230
1969 1230.2
1968.2 1231
1968.2 1232
1968 1232.2
1967.2 1233
1967 1233.2
1966.2 1234
1966 1234.2
1965.2 1235
1965 1235.2
1964 1235.2
1963.2 1236
1963 1236.2
1962.2 1237
1962 1237.2
1961 1237.2
1960.2 1238
1960 1238.2
1959 1238.2
1958 1238.2
1957.2 1239
1957 1239.2
};
\addplot [red]
table {%
1040 1288.2
1039 1288.2
1038 1288.2
1037 1288.2
1036 1288.2
1035 1288.2
1034 1288.2
1033 1288.2
1032 1288.2
1031.8 1288
1031 1287.2
1030 1287.2
1029 1287.2
1028 1287.2
1027 1287.2
1026.8 1287
1026 1286.2
1025 1286.2
1024 1286.2
1023 1286.2
1022 1286.2
1021 1286.2
1020.8 1286
1020 1285.2
1019 1285.2
1018 1285.2
1017 1285.2
1016 1285.2
1015 1285.2
1014 1285.2
1013.8 1285
1013 1284.2
1012 1284.2
1011 1284.2
1010 1284.2
1009 1284.2
1008 1284.2
1007.8 1284
1007 1283.2
1006 1283.2
1005 1283.2
1004 1283.2
1003.8 1283
1003 1282.2
1002 1282.2
1001 1282.2
1000 1282.2
999.8 1282
999 1281.2
998 1281.2
997 1281.2
996 1281.2
995.8 1281
995 1280.2
994 1280.2
993 1280.2
992 1280.2
991 1280.2
990.8 1280
990 1279.2
989 1279.2
988.8 1279
988 1278.2
987 1278.2
986.8 1278
986 1277.2
985 1277.2
984 1277.2
983.8 1277
983 1276.2
982 1276.2
981.8 1276
981 1275.2
980 1275.2
979.8 1275
979 1274.2
978 1274.2
977.8 1274
977 1273.2
976 1273.2
975.8 1273
975 1272.2
974.8 1272
974 1271.2
973 1271.2
972.8 1271
972 1270.2
971.8 1270
971 1269.2
970 1269.2
969 1269.2
968.8 1269
968 1268.2
967 1268.2
966 1268.2
965.8 1268
965 1267.2
964 1267.2
963 1267.2
962 1267.2
961.8 1267
961 1266.2
960 1266.2
959 1266.2
958.8 1266
958 1265.2
957 1265.2
956 1265.2
955 1265.2
954.8 1265
954 1264.2
953 1264.2
952 1264.2
951 1264.2
950.8 1264
950 1263.2
949.8 1263
949 1262.2
948.8 1262
948 1261.2
947.8 1261
947 1260.2
946 1260.2
945.8 1260
945 1259.2
944.8 1259
944 1258.2
943.8 1258
943 1257.2
942 1257.2
941 1257.2
940.2 1258
940 1258.2
939 1258.2
938 1258.2
937 1258.2
936 1258.2
935 1258.2
934.2 1259
934 1259.2
933 1259.2
932 1259.2
931 1259.2
930 1259.2
929.8 1259
929 1258.2
928 1258.2
927 1258.2
926 1258.2
925 1258.2
924.8 1258
924 1257.2
923.8 1257
923 1256.2
922.8 1256
922 1255.2
921 1255.2
920 1255.2
919.8 1255
919 1254.2
918 1254.2
917 1254.2
916.8 1254
916 1253.2
915.8 1253
915 1252.2
914 1252.2
913.8 1252
913 1251.2
912 1251.2
911 1251.2
910.8 1251
910 1250.2
909 1250.2
908 1250.2
907.8 1250
907 1249.2
906.8 1249
906 1248.2
905 1248.2
904.8 1248
904 1247.2
903 1247.2
902.8 1247
902 1246.2
901 1246.2
900.8 1246
900 1245.2
899 1245.2
898.8 1245
898 1244.2
897 1244.2
896.8 1244
896.8 1243
896 1242.2
895.8 1242
895.8 1241
895 1240.2
894.8 1240
894.8 1239
894 1238.2
893.8 1238
893.8 1237
893 1236.2
892.8 1236
893 1235.8
894 1235.8
894.8 1235
895 1234.8
896 1234.8
897 1234.8
898 1234.8
899 1234.8
900 1234.8
901 1234.8
901.8 1234
902 1233.8
903 1233.8
904 1233.8
904.2 1234
905 1234.8
906 1234.8
907 1234.8
907.2 1235
908 1235.8
909 1235.8
910 1235.8
910.2 1236
911 1236.8
912 1236.8
913 1236.8
914 1236.8
914.2 1237
915 1237.8
916 1237.8
916.2 1238
917 1238.8
918 1238.8
918.2 1239
919 1239.8
920 1239.8
921 1239.8
921.2 1240
922 1240.8
923 1240.8
924 1240.8
924.2 1241
925 1241.8
926 1241.8
927 1241.8
928 1241.8
928.2 1242
929 1242.8
930 1242.8
931 1242.8
932 1242.8
933 1242.8
934 1242.8
934.2 1243
935 1243.8
935.2 1244
936 1244.8
937 1244.8
937.8 1244
937.8 1243
937.8 1242
937.8 1241
938 1240.8
938.8 1240
939 1239.8
939.8 1239
939 1238.2
938 1238.2
937 1238.2
936 1238.2
935 1238.2
934 1238.2
933 1238.2
932 1238.2
931.8 1238
931 1237.2
930 1237.2
929 1237.2
928 1237.2
927.8 1237
927 1236.2
926 1236.2
925.8 1236
925 1235.2
924.8 1235
924 1234.2
923.8 1234
923 1233.2
922.8 1233
922.8 1232
922.8 1231
922.8 1230
922.8 1229
922.8 1228
923 1227.8
923.8 1227
924 1226.8
925 1226.8
925.8 1226
926 1225.8
927 1225.8
928 1225.8
929 1225.8
929.8 1225
930 1224.8
931 1224.8
932 1224.8
933 1224.8
934 1224.8
935 1224.8
936 1224.8
936.2 1225
937 1225.8
938 1225.8
939 1225.8
939.2 1226
940 1226.8
941 1226.8
941.2 1227
942 1227.8
943 1227.8
944 1227.8
944.2 1228
945 1228.8
945.2 1229
946 1229.8
946.2 1230
947 1230.8
947.2 1231
948 1231.8
949 1231.8
949.2 1232
949.2 1233
949.2 1234
949.2 1235
949 1235.2
948.2 1236
948 1236.2
947.2 1237
948 1237.8
949 1237.8
950 1237.8
950.8 1237
951 1236.8
952 1236.8
953 1236.8
954 1236.8
955 1236.8
956 1236.8
957 1236.8
958 1236.8
959 1236.8
960 1236.8
961 1236.8
962 1236.8
962.2 1237
963 1237.8
964 1237.8
965 1237.8
966 1237.8
967 1237.8
968 1237.8
968.2 1238
969 1238.8
970 1238.8
971 1238.8
972 1238.8
973 1238.8
973.2 1239
974 1239.8
975 1239.8
976 1239.8
977 1239.8
978 1239.8
978.2 1240
979 1240.8
980 1240.8
981 1240.8
982 1240.8
982.2 1241
983 1241.8
984 1241.8
985 1241.8
986 1241.8
987 1241.8
988 1241.8
989 1241.8
989.2 1242
990 1242.8
991 1242.8
991.2 1243
992 1243.8
993 1243.8
993.2 1244
994 1244.8
995 1244.8
996 1244.8
997 1244.8
997.2 1245
998 1245.8
999 1245.8
999.2 1246
1000 1246.8
1001 1246.8
1001.2 1247
1002 1247.8
1003 1247.8
1004 1247.8
1005 1247.8
1005.2 1248
1006 1248.8
1007 1248.8
1007.2 1249
1008 1249.8
1008.2 1250
1009 1250.8
1009.2 1251
1010 1251.8
1010.2 1252
1011 1252.8
1011.2 1253
1012 1253.8
1012.2 1254
1013 1254.8
1014 1254.8
1014.2 1255
1015 1255.8
1016 1255.8
1016.2 1256
1017 1256.8
1018 1256.8
1019 1256.8
1019.2 1257
1020 1257.8
1021 1257.8
1022 1257.8
1022.2 1258
1023 1258.8
1024 1258.8
1025 1258.8
1025.2 1259
1026 1259.8
1027 1259.8
1028 1259.8
1028.2 1260
1029 1260.8
1030 1260.8
1030.2 1261
1031 1261.8
1032 1261.8
1033 1261.8
1033.2 1262
1034 1262.8
1035 1262.8
1036 1262.8
1036.2 1263
1037 1263.8
1038 1263.8
1039 1263.8
1039.2 1264
1040 1264.8
1040.2 1265
1041 1265.8
1041.2 1266
1042 1266.8
1043 1266.8
1043.2 1267
1044 1267.8
1044.2 1268
1045 1268.8
1045.2 1269
1046 1269.8
1047 1269.8
1047.2 1270
1048 1270.8
1048.2 1271
1049 1271.8
1049.2 1272
1050 1272.8
1050.2 1273
1051 1273.8
1052 1273.8
1052.2 1274
1053 1274.8
1053.2 1275
1053.2 1276
1053.2 1277
1053.2 1278
1053.2 1279
1053.2 1280
1053.2 1281
1053.2 1282
1053.2 1283
1053.2 1284
1053.2 1285
1053.2 1286
1053 1286.2
1052 1286.2
1051.2 1287
1051 1287.2
1050 1287.2
1049 1287.2
1048 1287.2
1047 1287.2
1046 1287.2
1045 1287.2
1044 1287.2
1043 1287.2
1042 1287.2
1041 1287.2
1040.2 1288
1040 1288.2
};
\addplot [red]
table {%
3428 1270.2
3427 1270.2
3426 1270.2
3425 1270.2
3424 1270.2
3423.8 1270
3423 1269.2
3422 1269.2
3421 1269.2
3420 1269.2
3419.8 1269
3419 1268.2
3418 1268.2
3417 1268.2
3416 1268.2
3415.8 1268
3415 1267.2
3414 1267.2
3413 1267.2
3412 1267.2
3411.8 1267
3411 1266.2
3410 1266.2
3409 1266.2
3408 1266.2
3407.8 1266
3407 1265.2
3406 1265.2
3405 1265.2
3404 1265.2
3403.8 1265
3403 1264.2
3402 1264.2
3401 1264.2
3400 1264.2
3399.8 1264
3399 1263.2
3398 1263.2
3397 1263.2
3396 1263.2
3395.8 1263
3395 1262.2
3394.8 1262
3394.8 1261
3394 1260.2
3393.8 1260
3393 1259.2
3392.8 1259
3392.8 1258
3392 1257.2
3391.8 1257
3391 1256.2
3390.8 1256
3390.8 1255
3390 1254.2
3389.8 1254
3389 1253.2
3388.8 1253
3388 1252.2
3387.8 1252
3387.8 1251
3387 1250.2
3386.8 1250
3386 1249.2
3385.8 1249
3385.8 1248
3385 1247.2
3384.8 1247
3384 1246.2
3383.8 1246
3384 1245.8
3385 1245.8
3385.8 1245
3386 1244.8
3386.8 1244
3387 1243.8
3387.8 1243
3388 1242.8
3389 1242.8
3389.8 1242
3390 1241.8
3390.8 1241
3391 1240.8
3391.8 1240
3392 1239.8
3393 1239.8
3393.8 1239
3394 1238.8
3394.8 1238
3395 1237.8
3396 1237.8
3396.8 1237
3397 1236.8
3397.8 1236
3398 1235.8
3398.8 1235
3399 1234.8
3400 1234.8
3400.8 1234
3401 1233.8
3401.8 1233
3402 1232.8
3403 1232.8
3403.8 1232
3404 1231.8
3404.8 1231
3405 1230.8
3406 1230.8
3407 1230.8
3408 1230.8
3409 1230.8
3409.8 1230
3410 1229.8
3411 1229.8
3412 1229.8
3413 1229.8
3414 1229.8
3415 1229.8
3416 1229.8
3416.8 1229
3417 1228.8
3418 1228.8
3419 1228.8
3420 1228.8
3421 1228.8
3422 1228.8
3423 1228.8
3423.8 1228
3424 1227.8
3425 1227.8
3426 1227.8
3427 1227.8
3428 1227.8
3429 1227.8
3430 1227.8
3430.8 1227
3431 1226.8
3432 1226.8
3433 1226.8
3434 1226.8
3435 1226.8
3436 1226.8
3437 1226.8
3438 1226.8
3439 1226.8
3440 1226.8
3441 1226.8
3442 1226.8
3443 1226.8
3444 1226.8
3445 1226.8
3446 1226.8
3447 1226.8
3448 1226.8
3449 1226.8
3450 1226.8
3451 1226.8
3452 1226.8
3453 1226.8
3454 1226.8
3455 1226.8
3456 1226.8
3457 1226.8
3458 1226.8
3458.2 1227
3459 1227.8
3460 1227.8
3461 1227.8
3462 1227.8
3463 1227.8
3464 1227.8
3465 1227.8
3466 1227.8
3466.8 1227
3467 1226.8
3468 1226.8
3469 1226.8
3470 1226.8
3470.2 1227
3471 1227.8
3472 1227.8
3473 1227.8
3474 1227.8
3475 1227.8
3476 1227.8
3477 1227.8
3478 1227.8
3479 1227.8
3480 1227.8
3481 1227.8
3482 1227.8
3483 1227.8
3484 1227.8
3485 1227.8
3486 1227.8
3487 1227.8
3488 1227.8
3489 1227.8
3489.8 1227
3490 1226.8
3490.2 1227
3491 1227.8
3492 1227.8
3493 1227.8
3494 1227.8
3495 1227.8
3496 1227.8
3496.2 1228
3497 1228.8
3498 1228.8
3499 1228.8
3500 1228.8
3500.2 1229
3501 1229.8
3502 1229.8
3503 1229.8
3503.2 1230
3504 1230.8
3505 1230.8
3506 1230.8
3506.2 1231
3507 1231.8
3508 1231.8
3508.2 1232
3509 1232.8
3510 1232.8
3510.2 1233
3511 1233.8
3512 1233.8
3512.2 1234
3512.2 1235
3513 1235.8
3513.2 1236
3513.2 1237
3514 1237.8
3514.2 1238
3514.2 1239
3514.2 1240
3514.2 1241
3514.2 1242
3514.2 1243
3514.2 1244
3514.2 1245
3514.2 1246
3514 1246.2
3513.2 1247
3513.2 1248
3513.2 1249
3513 1249.2
3512.2 1250
3512.2 1251
3512 1251.2
3511.2 1252
3511.2 1253
3511 1253.2
3510.2 1254
3510.2 1255
3510 1255.2
3509.2 1256
3509.2 1257
3509 1257.2
3508.2 1258
3508 1258.2
3507.2 1259
3507 1259.2
3506.2 1260
3506.2 1261
3506 1261.2
3505.2 1262
3505 1262.2
3504 1262.2
3503.2 1263
3503 1263.2
3502.2 1264
3502 1264.2
3501 1264.2
3500.2 1265
3500 1265.2
3499 1265.2
3498.2 1266
3498 1266.2
3497 1266.2
3496.2 1267
3496 1267.2
3495 1267.2
3494 1267.2
3493 1267.2
3492 1267.2
3491 1267.2
3490 1267.2
3489 1267.2
3488 1267.2
3487 1267.2
3486 1267.2
3485 1267.2
3484.2 1268
3484 1268.2
3483 1268.2
3482 1268.2
3481 1268.2
3480 1268.2
3479 1268.2
3478 1268.2
3477 1268.2
3476 1268.2
3475 1268.2
3474 1268.2
3473 1268.2
3472 1268.2
3471 1268.2
3470 1268.2
3469 1268.2
3468 1268.2
3467 1268.2
3466 1268.2
3465 1268.2
3464 1268.2
3463.8 1268
3463 1267.2
3462 1267.2
3461 1267.2
3460 1267.2
3459 1267.2
3458.2 1268
3458 1268.2
3457 1268.2
3456 1268.2
3455 1268.2
3454 1268.2
3453 1268.2
3452 1268.2
3451 1268.2
3450.2 1269
3450 1269.2
3449 1269.2
3448 1269.2
3447 1269.2
3446 1269.2
3445 1269.2
3444 1269.2
3443 1269.2
3442 1269.2
3441 1269.2
3440 1269.2
3439 1269.2
3438 1269.2
3437 1269.2
3436 1269.2
3435 1269.2
3434 1269.2
3433 1269.2
3432 1269.2
3431 1269.2
3430 1269.2
3429 1269.2
3428.2 1270
3428 1270.2
};
\addplot [red]
table {%
945.8 1238
945 1237.2
944 1237.2
943.2 1238
944 1238.8
945 1238.8
945.8 1238
};
\addplot [red]
table {%
3361 1277.2
3360 1277.2
3359 1277.2
3358 1277.2
3357 1277.2
3356 1277.2
3355 1277.2
3354 1277.2
3353 1277.2
3352 1277.2
3351 1277.2
3350 1277.2
3349 1277.2
3348.8 1277
3348 1276.2
3347 1276.2
3346.8 1276
3346 1275.2
3345.8 1275
3345 1274.2
3344.8 1274
3344 1273.2
3343.8 1273
3343.8 1272
3343 1271.2
3342.8 1271
3342.8 1270
3342.8 1269
3342.8 1268
3342.8 1267
3342.8 1266
3342.8 1265
3342.8 1264
3342.8 1263
3343 1262.8
3343.8 1262
3344 1261.8
3344.8 1261
3345 1260.8
3345.8 1260
3346 1259.8
3346.8 1259
3347 1258.8
3347.8 1258
3348 1257.8
3349 1257.8
3350 1257.8
3351 1257.8
3352 1257.8
3353 1257.8
3353.2 1258
3354 1258.8
3354.2 1259
3355 1259.8
3356 1259.8
3357 1259.8
3357.2 1260
3358 1260.8
3359 1260.8
3360 1260.8
3360.2 1261
3361 1261.8
3362 1261.8
3363 1261.8
3364 1261.8
3365 1261.8
3365.2 1262
3366 1262.8
3367 1262.8
3368 1262.8
3369 1262.8
3370 1262.8
3371 1262.8
3372 1262.8
3373 1262.8
3374 1262.8
3375 1262.8
3375.2 1263
3376 1263.8
3377 1263.8
3377.2 1264
3378 1264.8
3378.2 1265
3379 1265.8
3380 1265.8
3380.2 1266
3381 1266.8
3381.2 1267
3382 1267.8
3382.2 1268
3382 1268.2
3381 1268.2
3380.2 1269
3380 1269.2
3379.2 1270
3379 1270.2
3378 1270.2
3377.2 1271
3377 1271.2
3376.2 1272
3376 1272.2
3375 1272.2
3374.2 1273
3374 1273.2
3373 1273.2
3372 1273.2
3371 1273.2
3370.2 1274
3370 1274.2
3369 1274.2
3368 1274.2
3367.2 1275
3367 1275.2
3366 1275.2
3365 1275.2
3364.2 1276
3364 1276.2
3363 1276.2
3362 1276.2
3361.2 1277
3361 1277.2
};
\addplot [red]
table {%
4874 1352.2
4873 1352.2
4872 1352.2
4871 1352.2
4870 1352.2
4869 1352.2
4868 1352.2
4867 1352.2
4866 1352.2
4865 1352.2
4864 1352.2
4863 1352.2
4862 1352.2
4861 1352.2
4860 1352.2
4859 1352.2
4858 1352.2
4857 1352.2
4856 1352.2
4855 1352.2
4854 1352.2
4853 1352.2
4852 1352.2
4851 1352.2
4850 1352.2
4849 1352.2
4848 1352.2
4847.8 1352
4847 1351.2
4846 1351.2
4845 1351.2
4844 1351.2
4843 1351.2
4842 1351.2
4841 1351.2
4840.8 1351
4840 1350.2
4839 1350.2
4838.2 1351
4838 1351.2
4837 1351.2
4836.8 1351
4836 1350.2
4835 1350.2
4834 1350.2
4833 1350.2
4832 1350.2
4831 1350.2
4830 1350.2
4829 1350.2
4828 1350.2
4827.8 1350
4827 1349.2
4826 1349.2
4825 1349.2
4824 1349.2
4823.8 1349
4823 1348.2
4822 1348.2
4821.8 1348
4821 1347.2
4820.8 1347
4820 1346.2
4819 1346.2
4818 1346.2
4817 1346.2
4816 1346.2
4815 1346.2
4814 1346.2
4813 1346.2
4812 1346.2
4811.2 1347
4811 1347.2
4810 1347.2
4809 1347.2
4808 1347.2
4807 1347.2
4806 1347.2
4805 1347.2
4804 1347.2
4803 1347.2
4802 1347.2
4801 1347.2
4800 1347.2
4799 1347.2
4798.8 1347
4798 1346.2
4797 1346.2
4796 1346.2
4795.8 1346
4795 1345.2
4794 1345.2
4793.8 1345
4793 1344.2
4792 1344.2
4791 1344.2
4790.8 1344
4790 1343.2
4789.8 1343
4789 1342.2
4788.8 1342
4788 1341.2
4787.2 1342
4787 1342.2
4786.2 1343
4786 1343.2
4785.2 1344
4785 1344.2
4784.2 1345
4784 1345.2
4783.2 1346
4783 1346.2
4782 1346.2
4781.2 1347
4781 1347.2
4780.2 1348
4780 1348.2
4779 1348.2
4778.2 1349
4778 1349.2
4777 1349.2
4776 1349.2
4775 1349.2
4774.2 1350
4774 1350.2
4773 1350.2
4772 1350.2
4771 1350.2
4770 1350.2
4769 1350.2
4768 1350.2
4767 1350.2
4766 1350.2
4765.8 1350
4765 1349.2
4764 1349.2
4763 1349.2
4762 1349.2
4761 1349.2
4760 1349.2
4759 1349.2
4758.8 1349
4758 1348.2
4757 1348.2
4756 1348.2
4755 1348.2
4754.8 1348
4754 1347.2
4753 1347.2
4752 1347.2
4751 1347.2
4750.8 1347
4750 1346.2
4749 1346.2
4748 1346.2
4747 1346.2
4746.8 1346
4746 1345.2
4745 1345.2
4744 1345.2
4743.8 1345
4743 1344.2
4742 1344.2
4741.8 1344
4741 1343.2
4740.8 1343
4740.8 1342
4740 1341.2
4739.8 1341
4739 1340.2
4738.8 1340
4738.8 1339
4738.8 1338
4738.8 1337
4738.8 1336
4738.8 1335
4738.8 1334
4739 1333.8
4739.8 1333
4740 1332.8
4740.8 1332
4740.8 1331
4741 1330.8
4741.8 1330
4742 1329.8
4742.8 1329
4743 1328.8
4743.8 1328
4744 1327.8
4745 1327.8
4745.8 1327
4746 1326.8
4746.8 1326
4747 1325.8
4748 1325.8
4748.8 1325
4749 1324.8
4749.8 1324
4750 1323.8
4751 1323.8
4751.8 1323
4752 1322.8
4753 1322.8
4753.8 1322
4754 1321.8
4755 1321.8
4755.8 1321
4756 1320.8
4757 1320.8
4758 1320.8
4758.8 1320
4759 1319.8
4760 1319.8
4761 1319.8
4762 1319.8
4763 1319.8
4764 1319.8
4764.8 1319
4765 1318.8
4766 1318.8
4767 1318.8
4768 1318.8
4769 1318.8
4770 1318.8
4771 1318.8
4772 1318.8
4773 1318.8
4774 1318.8
4775 1318.8
4776 1318.8
4776.2 1319
4777 1319.8
4778 1319.8
4779 1319.8
4780 1319.8
4781 1319.8
4781.8 1319
4782 1318.8
4783 1318.8
4783.8 1318
4784 1317.8
4785 1317.8
4785.8 1317
4786 1316.8
4787 1316.8
4788 1316.8
4789 1316.8
4790 1316.8
4791 1316.8
4792 1316.8
4793 1316.8
4794 1316.8
4795 1316.8
4796 1316.8
4797 1316.8
4798 1316.8
4799 1316.8
4800 1316.8
4801 1316.8
4802 1316.8
4803 1316.8
4804 1316.8
4805 1316.8
4806 1316.8
4807 1316.8
4808 1316.8
4809 1316.8
4809.2 1317
4810 1317.8
4811 1317.8
4812 1317.8
4813 1317.8
4814 1317.8
4814.2 1318
4815 1318.8
4816 1318.8
4817 1318.8
4818 1318.8
4818.2 1319
4819 1319.8
4820 1319.8
4821 1319.8
4821.2 1320
4822 1320.8
4823 1320.8
4824 1320.8
4824.2 1321
4825 1321.8
4826 1321.8
4827 1321.8
4828 1321.8
4828.2 1322
4829 1322.8
4830 1322.8
4831 1322.8
4832 1322.8
4832.2 1323
4833 1323.8
4834 1323.8
4835 1323.8
4835.2 1324
4836 1324.8
4837 1324.8
4838 1324.8
4839 1324.8
4839.2 1325
4840 1325.8
4841 1325.8
4841.2 1326
4841.2 1327
4841.2 1328
4841.2 1329
4841.2 1330
4842 1330.8
4842.2 1331
4842.2 1332
4842 1332.2
4841.2 1333
4841 1333.2
4840.2 1334
4840.2 1335
4840 1335.2
4839.2 1336
4839 1336.2
4838.2 1337
4838.2 1338
4838 1338.2
4837 1338.2
4836.2 1339
4836 1339.2
4835 1339.2
4834.2 1340
4834 1340.2
4833 1340.2
4832.2 1341
4833 1341.8
4834 1341.8
4835 1341.8
4835.8 1341
4836 1340.8
4837 1340.8
4838 1340.8
4839 1340.8
4840 1340.8
4840.8 1340
4841 1339.8
4841.2 1340
4842 1340.8
4843 1340.8
4844 1340.8
4845 1340.8
4845.8 1340
4846 1339.8
4847 1339.8
4848 1339.8
4849 1339.8
4850 1339.8
4851 1339.8
4852 1339.8
4853 1339.8
4854 1339.8
4855 1339.8
4856 1339.8
4857 1339.8
4858 1339.8
4859 1339.8
4860 1339.8
4861 1339.8
4862 1339.8
4863 1339.8
4864 1339.8
4865 1339.8
4866 1339.8
4867 1339.8
4868 1339.8
4869 1339.8
4870 1339.8
4871 1339.8
4872 1339.8
4873 1339.8
4874 1339.8
4875 1339.8
4875.2 1340
4876 1340.8
4877 1340.8
4877.2 1341
4878 1341.8
4879 1341.8
4879.2 1342
4880 1342.8
4881 1342.8
4881.2 1343
4882 1343.8
4883 1343.8
4883.2 1344
4884 1344.8
4885 1344.8
4885.2 1345
4886 1345.8
4887 1345.8
4887.2 1346
4887 1346.2
4886 1346.2
4885 1346.2
4884.2 1347
4884 1347.2
4883 1347.2
4882.2 1348
4882 1348.2
4881 1348.2
4880.2 1349
4880 1349.2
4879 1349.2
4878.2 1350
4878 1350.2
4877 1350.2
4876.2 1351
4876 1351.2
4875 1351.2
4874.2 1352
4874 1352.2
};
\addplot [red]
table {%
1818 1460.2
1817 1460.2
1816 1460.2
1815 1460.2
1814 1460.2
1813 1460.2
1812 1460.2
1811 1460.2
1810 1460.2
1809 1460.2
1808 1460.2
1807 1460.2
1806.8 1460
1806 1459.2
1805 1459.2
1804 1459.2
1803 1459.2
1802 1459.2
1801 1459.2
1800 1459.2
1799 1459.2
1798 1459.2
1797.8 1459
1797 1458.2
1796 1458.2
1795 1458.2
1794 1458.2
1793.8 1458
1793 1457.2
1792 1457.2
1791 1457.2
1790.8 1457
1790 1456.2
1789 1456.2
1788 1456.2
1787.8 1456
1787 1455.2
1786 1455.2
1785 1455.2
1784 1455.2
1783.8 1455
1783 1454.2
1782 1454.2
1781 1454.2
1780 1454.2
1779.8 1454
1779 1453.2
1778 1453.2
1777 1453.2
1776.8 1453
1776 1452.2
1775 1452.2
1774 1452.2
1773.8 1452
1773 1451.2
1772 1451.2
1771 1451.2
1770.8 1451
1770 1450.2
1769 1450.2
1768 1450.2
1767.8 1450
1767 1449.2
1766 1449.2
1765 1449.2
1764.8 1449
1764 1448.2
1763 1448.2
1762 1448.2
1761.8 1448
1761 1447.2
1760 1447.2
1759 1447.2
1758 1447.2
1757.8 1447
1757 1446.2
1756 1446.2
1755 1446.2
1754 1446.2
1753.8 1446
1753 1445.2
1752 1445.2
1751 1445.2
1750 1445.2
1749.8 1445
1749 1444.2
1748 1444.2
1747 1444.2
1746 1444.2
1745.8 1444
1745 1443.2
1744 1443.2
1743 1443.2
1742 1443.2
1741.8 1443
1741 1442.2
1740 1442.2
1739 1442.2
1738 1442.2
1737.8 1442
1737 1441.2
1736 1441.2
1735 1441.2
1734 1441.2
1733.8 1441
1733 1440.2
1732 1440.2
1731 1440.2
1730 1440.2
1729.8 1440
1729 1439.2
1728.8 1439
1728 1438.2
1727.8 1438
1727.8 1437
1727 1436.2
1726.8 1436
1726 1435.2
1725.8 1435
1725 1434.2
1724.8 1434
1724 1433.2
1723.8 1433
1723 1432.2
1722.8 1432
1722 1431.2
1721.8 1431
1721 1430.2
1720.8 1430
1720 1429.2
1719.8 1429
1719 1428.2
1718.8 1428
1718 1427.2
1717.8 1427
1717 1426.2
1716.8 1426
1716 1425.2
1715.8 1425
1715 1424.2
1714.8 1424
1714 1423.2
1713.8 1423
1713.8 1422
1713.8 1421
1713.8 1420
1714 1419.8
1714.8 1419
1714.8 1418
1714.8 1417
1715 1416.8
1715.8 1416
1715.8 1415
1716 1414.8
1716.8 1414
1716 1413.2
1715 1413.2
1714 1413.2
1713 1413.2
1712 1413.2
1711.8 1413
1711 1412.2
1710 1412.2
1709 1412.2
1708 1412.2
1707.8 1412
1707 1411.2
1706 1411.2
1705 1411.2
1704 1411.2
1703.8 1411
1703 1410.2
1702 1410.2
1701 1410.2
1700 1410.2
1699.8 1410
1699 1409.2
1698 1409.2
1697 1409.2
1696 1409.2
1695.8 1409
1695 1408.2
1694 1408.2
1693 1408.2
1692 1408.2
1691 1408.2
1690 1408.2
1689 1408.2
1688.8 1408
1688 1407.2
1687 1407.2
1686 1407.2
1685 1407.2
1684.8 1407
1684 1406.2
1683 1406.2
1682 1406.2
1681 1406.2
1680 1406.2
1679.8 1406
1679 1405.2
1678 1405.2
1677.8 1405
1677 1404.2
1676 1404.2
1675.8 1404
1675 1403.2
1674.8 1403
1674 1402.2
1673 1402.2
1672.8 1402
1672 1401.2
1671 1401.2
1670.8 1401
1670.8 1400
1670 1399.2
1669.8 1399
1669.8 1398
1669 1397.2
1668.8 1397
1668.8 1396
1668 1395.2
1667.8 1395
1667.8 1394
1667 1393.2
1666.8 1393
1667 1392.8
1667.8 1392
1667.8 1391
1667.8 1390
1667.8 1389
1667.8 1388
1667.8 1387
1667.8 1386
1667.8 1385
1667.8 1384
1668 1383.8
1668.8 1383
1669 1382.8
1669.8 1382
1669.8 1381
1670 1380.8
1670.8 1380
1670.8 1379
1671 1378.8
1671.8 1378
1671.8 1377
1672 1376.8
1672.8 1376
1673 1375.8
1674 1375.8
1674.8 1375
1675 1374.8
1675.8 1374
1676 1373.8
1677 1373.8
1677.8 1373
1678 1372.8
1678.8 1372
1679 1371.8
1680 1371.8
1680.8 1371
1681 1370.8
1682 1370.8
1682.8 1370
1683 1369.8
1683.8 1369
1684 1368.8
1685 1368.8
1685.8 1368
1686 1367.8
1687 1367.8
1687.8 1367
1688 1366.8
1689 1366.8
1689.8 1366
1690 1365.8
1691 1365.8
1691.8 1365
1692 1364.8
1693 1364.8
1693.8 1364
1694 1363.8
1695 1363.8
1696 1363.8
1696.8 1363
1697 1362.8
1698 1362.8
1699 1362.8
1700 1362.8
1700.8 1362
1701 1361.8
1702 1361.8
1703 1361.8
1704 1361.8
1705 1361.8
1706 1361.8
1707 1361.8
1708 1361.8
1709 1361.8
1709.8 1361
1710 1360.8
1711 1360.8
1712 1360.8
1713 1360.8
1714 1360.8
1715 1360.8
1716 1360.8
1717 1360.8
1718 1360.8
1719 1360.8
1720 1360.8
1721 1360.8
1722 1360.8
1722.2 1361
1723 1361.8
1724 1361.8
1725 1361.8
1726 1361.8
1727 1361.8
1727.2 1362
1728 1362.8
1729 1362.8
1730 1362.8
1730.2 1363
1731 1363.8
1732 1363.8
1732.2 1364
1733 1364.8
1734 1364.8
1734.2 1365
1735 1365.8
1736 1365.8
1736.2 1366
1737 1366.8
1738 1366.8
1738.2 1367
1739 1367.8
1740 1367.8
1741 1367.8
1741.2 1368
1742 1368.8
1743 1368.8
1743.2 1369
1744 1369.8
1745 1369.8
1745.2 1370
1746 1370.8
1747 1370.8
1748 1370.8
1748.2 1371
1749 1371.8
1750 1371.8
1751 1371.8
1752 1371.8
1752.2 1372
1753 1372.8
1754 1372.8
1755 1372.8
1756 1372.8
1756.2 1373
1757 1373.8
1758 1373.8
1759 1373.8
1760 1373.8
1760.2 1374
1761 1374.8
1762 1374.8
1762.2 1375
1763 1375.8
1764 1375.8
1764.2 1376
1765 1376.8
1766 1376.8
1766.2 1377
1767 1377.8
1768 1377.8
1768.2 1378
1769 1378.8
1770 1378.8
1770.2 1379
1771 1379.8
1772 1379.8
1773 1379.8
1774 1379.8
1775 1379.8
1776 1379.8
1777 1379.8
1778 1379.8
1779 1379.8
1780 1379.8
1781 1379.8
1782 1379.8
1783 1379.8
1784 1379.8
1785 1379.8
1786 1379.8
1786.2 1380
1787 1380.8
1788 1380.8
1789 1380.8
1790 1380.8
1791 1380.8
1792 1380.8
1793 1380.8
1794 1380.8
1794.2 1381
1795 1381.8
1796 1381.8
1797 1381.8
1797.2 1382
1798 1382.8
1799 1382.8
1800 1382.8
1801 1382.8
1801.2 1383
1802 1383.8
1803 1383.8
1804 1383.8
1804.2 1384
1805 1384.8
1806 1384.8
1806.2 1385
1807 1385.8
1808 1385.8
1808.2 1386
1809 1386.8
1810 1386.8
1810.2 1387
1811 1387.8
1811.2 1388
1812 1388.8
1813 1388.8
1813.2 1389
1814 1389.8
1815 1389.8
1815.2 1390
1816 1390.8
1817 1390.8
1817.2 1391
1818 1391.8
1819 1391.8
1819.2 1392
1820 1392.8
1821 1392.8
1821.2 1393
1822 1393.8
1823 1393.8
1823.2 1394
1824 1394.8
1825 1394.8
1825.2 1395
1826 1395.8
1827 1395.8
1827.2 1396
1828 1396.8
1829 1396.8
1829.2 1397
1830 1397.8
1831 1397.8
1831.2 1398
1832 1398.8
1833 1398.8
1833.2 1399
1834 1399.8
1834.2 1400
1835 1400.8
1835.2 1401
1836 1401.8
1836.2 1402
1837 1402.8
1837.2 1403
1838 1403.8
1838.2 1404
1839 1404.8
1839.2 1405
1840 1405.8
1840.2 1406
1841 1406.8
1841.2 1407
1841.2 1408
1841.2 1409
1842 1409.8
1842.2 1410
1842.2 1411
1843 1411.8
1843.2 1412
1843.2 1413
1844 1413.8
1844.2 1414
1844.2 1415
1844.2 1416
1844.2 1417
1844.2 1418
1844.2 1419
1844 1419.2
1843.2 1420
1843.2 1421
1843.2 1422
1843 1422.2
1842.2 1423
1842.2 1424
1842.2 1425
1842 1425.2
1841.2 1426
1841 1426.2
1840.2 1427
1840 1427.2
1839.2 1428
1839 1428.2
1838 1428.2
1837.2 1429
1837 1429.2
1836.2 1430
1836 1430.2
1835.2 1431
1835 1431.2
1834 1431.2
1833.2 1432
1833 1432.2
1832 1432.2
1831 1432.2
1830 1432.2
1829 1432.2
1828 1432.2
1827 1432.2
1826 1432.2
1825.2 1433
1825 1433.2
1824 1433.2
1823 1433.2
1822 1433.2
1821.2 1434
1822 1434.8
1823 1434.8
1824 1434.8
1825 1434.8
1825.2 1435
1826 1435.8
1827 1435.8
1827.2 1436
1828 1436.8
1828.2 1437
1829 1437.8
1830 1437.8
1830.2 1438
1831 1438.8
1831.2 1439
1832 1439.8
1832.2 1440
1833 1440.8
1833.2 1441
1834 1441.8
1834.2 1442
1834.2 1443
1835 1443.8
1835.2 1444
1835.2 1445
1836 1445.8
1836.2 1446
1837 1446.8
1837.2 1447
1837 1447.2
1836.2 1448
1836.2 1449
1836.2 1450
1836 1450.2
1835.2 1451
1835.2 1452
1835.2 1453
1835 1453.2
1834.2 1454
1834 1454.2
1833 1454.2
1832.2 1455
1832 1455.2
1831.2 1456
1831 1456.2
1830 1456.2
1829.2 1457
1829 1457.2
1828 1457.2
1827.2 1458
1827 1458.2
1826 1458.2
1825 1458.2
1824.2 1459
1824 1459.2
1823 1459.2
1822 1459.2
1821 1459.2
1820 1459.2
1819 1459.2
1818.2 1460
1818 1460.2
};
\addplot [red]
table {%
4442 1444.2
4441 1444.2
4440 1444.2
4439 1444.2
4438 1444.2
4437.8 1444
4437 1443.2
4436 1443.2
4435 1443.2
4434 1443.2
4433 1443.2
4432 1443.2
4431 1443.2
4430.8 1443
4430 1442.2
4429 1442.2
4428 1442.2
4427 1442.2
4426 1442.2
4425 1442.2
4424 1442.2
4423.8 1442
4423 1441.2
4422 1441.2
4421 1441.2
4420 1441.2
4419.8 1441
4419 1440.2
4418 1440.2
4417 1440.2
4416.8 1440
4416 1439.2
4415 1439.2
4414.8 1439
4414 1438.2
4413 1438.2
4412 1438.2
4411.8 1438
4411 1437.2
4410 1437.2
4409 1437.2
4408.8 1437
4408 1436.2
4407 1436.2
4406 1436.2
4405.8 1436
4405 1435.2
4404 1435.2
4403.8 1435
4403 1434.2
4402 1434.2
4401 1434.2
4400.8 1434
4400 1433.2
4399 1433.2
4398 1433.2
4397.8 1433
4397 1432.2
4396 1432.2
4395.8 1432
4395 1431.2
4394.8 1431
4394 1430.2
4393 1430.2
4392.8 1430
4392 1429.2
4391.8 1429
4391 1428.2
4390 1428.2
4389.8 1428
4389 1427.2
4388.8 1427
4388 1426.2
4387 1426.2
4386.8 1426
4386.8 1425
4387 1424.8
4387.8 1424
4387.8 1423
4387.8 1422
4388 1421.8
4388.8 1421
4388.8 1420
4388.8 1419
4389 1418.8
4389.8 1418
4390 1417.8
4390.8 1417
4391 1416.8
4392 1416.8
4392.8 1416
4393 1415.8
4394 1415.8
4395 1415.8
4395.8 1415
4396 1414.8
4397 1414.8
4398 1414.8
4398.8 1414
4399 1413.8
4400 1413.8
4400.8 1413
4401 1412.8
4402 1412.8
4403 1412.8
4404 1412.8
4404.8 1412
4405 1411.8
4406 1411.8
4407 1411.8
4408 1411.8
4408.8 1411
4409 1410.8
4410 1410.8
4411 1410.8
4412 1410.8
4412.8 1410
4413 1409.8
4414 1409.8
4415 1409.8
4415.8 1409
4416 1408.8
4417 1408.8
4418 1408.8
4419 1408.8
4420 1408.8
4420.8 1408
4421 1407.8
4422 1407.8
4423 1407.8
4424 1407.8
4425 1407.8
4426 1407.8
4427 1407.8
4428 1407.8
4429 1407.8
4430 1407.8
4431 1407.8
4432 1407.8
4433 1407.8
4434 1407.8
4435 1407.8
4436 1407.8
4437 1407.8
4438 1407.8
4439 1407.8
4440 1407.8
4441 1407.8
4442 1407.8
4443 1407.8
4444 1407.8
4445 1407.8
4446 1407.8
4447 1407.8
4448 1407.8
4449 1407.8
4450 1407.8
4451 1407.8
4451.2 1408
4452 1408.8
4453 1408.8
4454 1408.8
4455 1408.8
4456 1408.8
4457 1408.8
4458 1408.8
4459 1408.8
4460 1408.8
4461 1408.8
4462 1408.8
4462.2 1409
4463 1409.8
4464 1409.8
4465 1409.8
4466 1409.8
4467 1409.8
4468 1409.8
4469 1409.8
4470 1409.8
4471 1409.8
4472 1409.8
4473 1409.8
4474 1409.8
4475 1409.8
4476 1409.8
4476.2 1410
4477 1410.8
4478 1410.8
4479 1410.8
4480 1410.8
4481 1410.8
4482 1410.8
4483 1410.8
4484 1410.8
4484.2 1411
4485 1411.8
4486 1411.8
4487 1411.8
4488 1411.8
4489 1411.8
4490 1411.8
4491 1411.8
4492 1411.8
4493 1411.8
4493.2 1412
4494 1412.8
4495 1412.8
4496 1412.8
4497 1412.8
4498 1412.8
4499 1412.8
4499.2 1413
4500 1413.8
4500.2 1414
4501 1414.8
4501.2 1415
4502 1415.8
4502.2 1416
4503 1416.8
4503.2 1417
4504 1417.8
4504.2 1418
4505 1418.8
4505.2 1419
4506 1419.8
4507 1419.8
4507.2 1420
4508 1420.8
4508.2 1421
4509 1421.8
4509.2 1422
4510 1422.8
4510.2 1423
4511 1423.8
4511.2 1424
4512 1424.8
4512.2 1425
4513 1425.8
4513.2 1426
4513 1426.2
4512 1426.2
4511.2 1427
4511 1427.2
4510.2 1428
4510 1428.2
4509 1428.2
4508.2 1429
4508 1429.2
4507 1429.2
4506.2 1430
4506 1430.2
4505 1430.2
4504.2 1431
4504 1431.2
4503 1431.2
4502.2 1432
4502 1432.2
4501 1432.2
4500.2 1433
4500 1433.2
4499.2 1434
4499 1434.2
4498 1434.2
4497.2 1435
4497 1435.2
4496 1435.2
4495.2 1436
4495 1436.2
4494 1436.2
4493.2 1437
4493 1437.2
4492 1437.2
4491.2 1438
4491 1438.2
4490 1438.2
4489 1438.2
4488 1438.2
4487 1438.2
4486 1438.2
4485 1438.2
4484 1438.2
4483 1438.2
4482.2 1439
4482 1439.2
4481 1439.2
4480 1439.2
4479 1439.2
4478 1439.2
4477 1439.2
4476 1439.2
4475 1439.2
4474.2 1440
4474 1440.2
4473 1440.2
4472 1440.2
4471 1440.2
4470 1440.2
4469 1440.2
4468 1440.2
4467 1440.2
4466 1440.2
4465.2 1441
4465 1441.2
4464 1441.2
4463 1441.2
4462 1441.2
4461 1441.2
4460 1441.2
4459 1441.2
4458.2 1442
4458 1442.2
4457 1442.2
4456 1442.2
4455 1442.2
4454 1442.2
4453 1442.2
4452 1442.2
4451.2 1443
4451 1443.2
4450 1443.2
4449 1443.2
4448 1443.2
4447 1443.2
4446 1443.2
4445 1443.2
4444 1443.2
4443 1443.2
4442.2 1444
4442 1444.2
};
\addplot [red]
table {%
6406 1489.2
6405 1489.2
6404 1489.2
6403 1489.2
6402 1489.2
6401.8 1489
6401 1488.2
6400 1488.2
6399 1488.2
6398.8 1488
6398 1487.2
6397 1487.2
6396 1487.2
6395.8 1487
6395 1486.2
6394 1486.2
6393 1486.2
6392.8 1486
6392 1485.2
6391 1485.2
6390 1485.2
6389 1485.2
6388.8 1485
6388 1484.2
6387.8 1484
6387 1483.2
6386.8 1483
6386.8 1482
6386 1481.2
6385.8 1481
6385.8 1480
6386 1479.8
6386.8 1479
6386.8 1478
6386.8 1477
6387 1476.8
6387.8 1476
6387 1475.2
6386 1475.2
6385.2 1476
6385 1476.2
6384 1476.2
6383 1476.2
6382 1476.2
6381 1476.2
6380 1476.2
6379 1476.2
6378 1476.2
6377 1476.2
6376 1476.2
6375 1476.2
6374 1476.2
6373 1476.2
6372.2 1477
6372 1477.2
6371 1477.2
6370 1477.2
6369 1477.2
6368 1477.2
6367 1477.2
6366 1477.2
6365 1477.2
6364.2 1478
6364 1478.2
6363 1478.2
6362 1478.2
6361 1478.2
6360 1478.2
6359 1478.2
6358 1478.2
6357 1478.2
6356 1478.2
6355 1478.2
6354.2 1479
6354 1479.2
6353 1479.2
6352 1479.2
6351 1479.2
6350 1479.2
6349 1479.2
6348 1479.2
6347 1479.2
6346 1479.2
6345 1479.2
6344 1479.2
6343 1479.2
6342 1479.2
6341 1479.2
6340 1479.2
6339 1479.2
6338 1479.2
6337 1479.2
6336 1479.2
6335 1479.2
6334 1479.2
6333.8 1479
6333 1478.2
6332 1478.2
6331 1478.2
6330 1478.2
6329 1478.2
6328 1478.2
6327.8 1478
6327 1477.2
6326 1477.2
6325 1477.2
6324 1477.2
6323.8 1477
6323 1476.2
6322 1476.2
6321 1476.2
6320 1476.2
6319.8 1476
6319 1475.2
6318 1475.2
6317 1475.2
6316.8 1475
6316 1474.2
6315 1474.2
6314.8 1474
6314 1473.2
6313.8 1473
6313 1472.2
6312 1472.2
6311.8 1472
6311 1471.2
6310 1471.2
6309.8 1471
6309 1470.2
6308.8 1470
6308 1469.2
6307.8 1469
6307.8 1468
6307 1467.2
6306.8 1467
6306 1466.2
6305.8 1466
6305 1465.2
6304.8 1465
6304 1464.2
6303.8 1464
6303.8 1463
6303.8 1462
6303 1461.2
6302.8 1461
6302.8 1460
6302 1459.2
6301.8 1459
6301.8 1458
6301.8 1457
6301.8 1456
6301.8 1455
6301.8 1454
6301.8 1453
6301.8 1452
6301.8 1451
6301.8 1450
6301.8 1449
6301.8 1448
6301.8 1447
6301.8 1446
6301.8 1445
6302 1444.8
6302.8 1444
6302.8 1443
6302.8 1442
6303 1441.8
6303.8 1441
6303.8 1440
6304 1439.8
6304.8 1439
6305 1438.8
6305.8 1438
6305.8 1437
6306 1436.8
6306.8 1436
6307 1435.8
6307.8 1435
6307.8 1434
6308 1433.8
6308.8 1433
6309 1432.8
6309.8 1432
6310 1431.8
6310.8 1431
6311 1430.8
6312 1430.8
6312.8 1430
6313 1429.8
6313.8 1429
6314 1428.8
6314.8 1428
6315 1427.8
6315.8 1427
6316 1426.8
6317 1426.8
6317.8 1426
6318 1425.8
6318.8 1425
6319 1424.8
6319.8 1424
6320 1423.8
6321 1423.8
6321.8 1423
6322 1422.8
6322.8 1422
6323 1421.8
6323.8 1421
6324 1420.8
6324.8 1420
6325 1419.8
6326 1419.8
6326.8 1419
6327 1418.8
6327.8 1418
6328 1417.8
6329 1417.8
6329.8 1417
6330 1416.8
6330.8 1416
6331 1415.8
6332 1415.8
6333 1415.8
6333.8 1415
6334 1414.8
6335 1414.8
6335.8 1414
6336 1413.8
6337 1413.8
6338 1413.8
6338.8 1413
6339 1412.8
6340 1412.8
6341 1412.8
6342 1412.8
6343 1412.8
6343.8 1412
6344 1411.8
6345 1411.8
6346 1411.8
6347 1411.8
6348 1411.8
6349 1411.8
6350 1411.8
6351 1411.8
6352 1411.8
6352.2 1412
6353 1412.8
6354 1412.8
6355 1412.8
6356 1412.8
6357 1412.8
6358 1412.8
6359 1412.8
6360 1412.8
6361 1412.8
6362 1412.8
6363 1412.8
6364 1412.8
6365 1412.8
6366 1412.8
6367 1412.8
6368 1412.8
6369 1412.8
6370 1412.8
6371 1412.8
6372 1412.8
6372.2 1413
6373 1413.8
6374 1413.8
6375 1413.8
6376 1413.8
6377 1413.8
6378 1413.8
6379 1413.8
6379.2 1414
6380 1414.8
6381 1414.8
6382 1414.8
6383 1414.8
6383.2 1415
6384 1415.8
6385 1415.8
6386 1415.8
6387 1415.8
6388 1415.8
6388.2 1416
6389 1416.8
6390 1416.8
6391 1416.8
6391.2 1417
6392 1417.8
6393 1417.8
6393.2 1418
6394 1418.8
6394.2 1419
6395 1419.8
6396 1419.8
6396.2 1420
6397 1420.8
6398 1420.8
6398.2 1421
6399 1421.8
6399.2 1422
6400 1422.8
6401 1422.8
6401.2 1423
6402 1423.8
6402.2 1424
6403 1424.8
6403.2 1425
6403.2 1426
6404 1426.8
6404.2 1427
6404.2 1428
6405 1428.8
6405.2 1429
6405.2 1430
6406 1430.8
6406.2 1431
6406.2 1432
6407 1432.8
6407.2 1433
6407.2 1434
6408 1434.8
6408.2 1435
6408.2 1436
6408.2 1437
6408.2 1438
6408.2 1439
6408.2 1440
6408.2 1441
6408.2 1442
6408.2 1443
6408.2 1444
6409 1444.8
6409.2 1445
6409.2 1446
6409.2 1447
6409.2 1448
6409.2 1449
6409.2 1450
6409.2 1451
6409.2 1452
6409.2 1453
6409.2 1454
6409.2 1455
6409.2 1456
6409.2 1457
6409.2 1458
6410 1458.8
6411 1458.8
6411.2 1459
6412 1459.8
6413 1459.8
6413.2 1460
6414 1460.8
6415 1460.8
6415.2 1461
6416 1461.8
6417 1461.8
6417.2 1462
6418 1462.8
6418.2 1463
6418.2 1464
6419 1464.8
6419.2 1465
6420 1465.8
6420.2 1466
6420.2 1467
6421 1467.8
6421.2 1468
6421.2 1469
6421.2 1470
6421.2 1471
6421.2 1472
6421.2 1473
6421.2 1474
6421.2 1475
6421.2 1476
6421.2 1477
6421.2 1478
6421 1478.2
6420.2 1479
6420.2 1480
6420.2 1481
6420 1481.2
6419.2 1482
6419 1482.2
6418.2 1483
6418 1483.2
6417.2 1484
6417 1484.2
6416.2 1485
6416 1485.2
6415 1485.2
6414.2 1486
6414 1486.2
6413 1486.2
6412.2 1487
6412 1487.2
6411 1487.2
6410 1487.2
6409.2 1488
6409 1488.2
6408 1488.2
6407 1488.2
6406.2 1489
6406 1489.2
};
\addplot [red]
table {%
6472 1488.2
6471.8 1488
6471 1487.2
6470 1487.2
6469 1487.2
6468 1487.2
6467 1487.2
6466 1487.2
6465 1487.2
6464 1487.2
6463 1487.2
6462.8 1487
6462 1486.2
6461 1486.2
6460 1486.2
6459 1486.2
6458 1486.2
6457 1486.2
6456.8 1486
6456 1485.2
6455 1485.2
6454 1485.2
6453 1485.2
6452 1485.2
6451.8 1485
6451 1484.2
6450 1484.2
6449 1484.2
6448.8 1484
6448 1483.2
6447 1483.2
6446.8 1483
6446 1482.2
6445 1482.2
6444 1482.2
6443.8 1482
6443 1481.2
6442 1481.2
6441.8 1481
6441 1480.2
6440 1480.2
6439.8 1480
6439 1479.2
6438.8 1479
6438 1478.2
6437 1478.2
6436.8 1478
6436 1477.2
6435 1477.2
6434.8 1477
6434 1476.2
6433.8 1476
6433 1475.2
6432 1475.2
6431.8 1475
6431 1474.2
6430.8 1474
6430 1473.2
6429 1473.2
6428.8 1473
6428 1472.2
6427.8 1472
6427 1471.2
6426.8 1471
6426 1470.2
6425.8 1470
6425 1469.2
6424.8 1469
6424.8 1468
6424 1467.2
6423.8 1467
6423.8 1466
6423 1465.2
6422.8 1465
6422.8 1464
6422 1463.2
6421.8 1463
6421.8 1462
6421.8 1461
6421.8 1460
6421 1459.2
6420.8 1459
6420.8 1458
6420.8 1457
6420.8 1456
6420.8 1455
6420.8 1454
6420.8 1453
6420.8 1452
6421 1451.8
6421.8 1451
6421.8 1450
6422 1449.8
6422.8 1449
6423 1448.8
6423.8 1448
6423.8 1447
6424 1446.8
6424.8 1446
6424.8 1445
6425 1444.8
6425.8 1444
6426 1443.8
6426.8 1443
6427 1442.8
6427.8 1442
6428 1441.8
6429 1441.8
6429.8 1441
6430 1440.8
6430.8 1440
6431 1439.8
6432 1439.8
6432.8 1439
6433 1438.8
6433.8 1438
6434 1437.8
6435 1437.8
6435.8 1437
6436 1436.8
6436.8 1436
6437 1435.8
6438 1435.8
6438.8 1435
6439 1434.8
6439.8 1434
6440 1433.8
6441 1433.8
6441.8 1433
6442 1432.8
6442.8 1432
6443 1431.8
6444 1431.8
6444.8 1431
6445 1430.8
6445.8 1430
6446 1429.8
6447 1429.8
6447.8 1429
6448 1428.8
6448.8 1428
6448 1427.2
6447.2 1428
6447 1428.2
6446 1428.2
6445 1428.2
6444 1428.2
6443 1428.2
6442 1428.2
6441 1428.2
6440 1428.2
6439 1428.2
6438.8 1428
6438 1427.2
6437 1427.2
6436 1427.2
6435.8 1427
6435 1426.2
6434 1426.2
6433.8 1426
6433 1425.2
6432 1425.2
6431 1425.2
6430 1425.2
6429.8 1425
6429 1424.2
6428 1424.2
6427 1424.2
6426 1424.2
6425 1424.2
6424.8 1424
6424 1423.2
6423 1423.2
6422 1423.2
6421.8 1423
6421 1422.2
6420.8 1422
6420 1421.2
6419.8 1421
6419 1420.2
6418.8 1420
6419 1419.8
6419.8 1419
6420 1418.8
6420.8 1418
6421 1417.8
6421.8 1417
6422 1416.8
6423 1416.8
6424 1416.8
6425 1416.8
6425.8 1416
6426 1415.8
6427 1415.8
6428 1415.8
6429 1415.8
6429.8 1415
6430 1414.8
6431 1414.8
6432 1414.8
6433 1414.8
6433.8 1414
6434 1413.8
6435 1413.8
6436 1413.8
6437 1413.8
6438 1413.8
6439 1413.8
6440 1413.8
6441 1413.8
6442 1413.8
6443 1413.8
6444 1413.8
6445 1413.8
6445.2 1414
6446 1414.8
6447 1414.8
6448 1414.8
6448.2 1415
6449 1415.8
6450 1415.8
6450.2 1416
6451 1416.8
6451.2 1417
6451.2 1418
6452 1418.8
6452.2 1419
6453 1419.8
6453.2 1420
6453.2 1421
6453.2 1422
6453 1422.2
6452.2 1423
6452.2 1424
6453 1424.8
6453.8 1424
6454 1423.8
6455 1423.8
6455.8 1423
6456 1422.8
6456.8 1422
6457 1421.8
6457.8 1421
6458 1420.8
6459 1420.8
6459.8 1420
6460 1419.8
6461 1419.8
6461.8 1419
6462 1418.8
6463 1418.8
6463.8 1418
6464 1417.8
6465 1417.8
6466 1417.8
6467 1417.8
6467.8 1417
6468 1416.8
6469 1416.8
6470 1416.8
6471 1416.8
6472 1416.8
6473 1416.8
6474 1416.8
6474.2 1417
6475 1417.8
6476 1417.8
6477 1417.8
6478 1417.8
6479 1417.8
6479.2 1418
6480 1418.8
6481 1418.8
6482 1418.8
6482.2 1419
6483 1419.8
6484 1419.8
6484.2 1420
6485 1420.8
6486 1420.8
6487 1420.8
6488 1420.8
6488.2 1421
6489 1421.8
6490 1421.8
6491 1421.8
6491.2 1422
6492 1422.8
6493 1422.8
6493.2 1423
6494 1423.8
6495 1423.8
6495.2 1424
6496 1424.8
6497 1424.8
6497.2 1425
6498 1425.8
6499 1425.8
6499.2 1426
6500 1426.8
6501 1426.8
6501.2 1427
6502 1427.8
6503 1427.8
6503.2 1428
6504 1428.8
6505 1428.8
6505.2 1429
6506 1429.8
6507 1429.8
6507.2 1430
6508 1430.8
6508.2 1431
6509 1431.8
6509.2 1432
6510 1432.8
6510.2 1433
6511 1433.8
6511.2 1434
6512 1434.8
6513 1434.8
6513.2 1435
6513.2 1436
6514 1436.8
6514.2 1437
6514.2 1438
6515 1438.8
6515.2 1439
6515.2 1440
6516 1440.8
6516.2 1441
6516.2 1442
6516.2 1443
6517 1443.8
6517.2 1444
6517.2 1445
6517.2 1446
6517.2 1447
6517.2 1448
6517.2 1449
6517.2 1450
6517.2 1451
6517.2 1452
6517.2 1453
6517.2 1454
6517.2 1455
6517.2 1456
6517.2 1457
6517.2 1458
6517.2 1459
6517.2 1460
6517 1460.2
6516.2 1461
6516.2 1462
6516.2 1463
6516 1463.2
6515.2 1464
6515.2 1465
6515 1465.2
6514.2 1466
6514.2 1467
6514 1467.2
6513.2 1468
6513.2 1469
6513 1469.2
6512.2 1470
6512 1470.2
6511.2 1471
6511.2 1472
6511 1472.2
6510.2 1473
6510 1473.2
6509.2 1474
6509 1474.2
6508.2 1475
6508 1475.2
6507.2 1476
6507 1476.2
6506.2 1477
6506 1477.2
6505 1477.2
6504.2 1478
6504 1478.2
6503 1478.2
6502.2 1479
6502 1479.2
6501.2 1480
6501 1480.2
6500 1480.2
6499.2 1481
6499 1481.2
6498 1481.2
6497.2 1482
6497 1482.2
6496 1482.2
6495.2 1483
6495 1483.2
6494 1483.2
6493 1483.2
6492.2 1484
6492 1484.2
6491 1484.2
6490 1484.2
6489.2 1485
6489 1485.2
6488 1485.2
6487 1485.2
6486.2 1486
6486 1486.2
6485 1486.2
6484 1486.2
6483 1486.2
6482 1486.2
6481.2 1487
6481 1487.2
6480 1487.2
6479 1487.2
6478 1487.2
6477 1487.2
6476 1487.2
6475 1487.2
6474 1487.2
6473 1487.2
6472.2 1488
6472 1488.2
};
\addplot [red]
table {%
1819.8 1433
1819 1432.2
1818.2 1433
1819 1433.8
1819.8 1433
};
\addplot [red]
table {%
4399 1476.2
4398 1476.2
4397 1476.2
4396 1476.2
4395 1476.2
4394 1476.2
4393 1476.2
4392 1476.2
4391 1476.2
4390 1476.2
4389 1476.2
4388 1476.2
4387 1476.2
4386 1476.2
4385 1476.2
4384 1476.2
4383 1476.2
4382 1476.2
4381 1476.2
4380 1476.2
4379 1476.2
4378 1476.2
4377 1476.2
4376 1476.2
4375 1476.2
4374 1476.2
4373 1476.2
4372 1476.2
4371 1476.2
4370 1476.2
4369 1476.2
4368 1476.2
4367 1476.2
4366 1476.2
4365 1476.2
4364.8 1476
4364 1475.2
4363 1475.2
4362 1475.2
4361 1475.2
4360 1475.2
4359 1475.2
4358 1475.2
4357 1475.2
4356 1475.2
4355 1475.2
4354 1475.2
4353 1475.2
4352 1475.2
4351.8 1475
4351 1474.2
4350 1474.2
4349 1474.2
4348 1474.2
4347 1474.2
4346.8 1474
4346 1473.2
4345 1473.2
4344 1473.2
4343 1473.2
4342 1473.2
4341.8 1473
4341 1472.2
4340 1472.2
4339 1472.2
4338 1472.2
4337.8 1472
4337 1471.2
4336 1471.2
4335 1471.2
4334 1471.2
4333.8 1471
4333 1470.2
4332 1470.2
4331.8 1470
4331 1469.2
4330 1469.2
4329.8 1469
4329 1468.2
4328.8 1468
4328 1467.2
4327.8 1467
4327 1466.2
4326.8 1466
4326 1465.2
4325 1465.2
4324.8 1465
4324 1464.2
4323.8 1464
4323 1463.2
4322.8 1463
4322 1462.2
4321.8 1462
4321 1461.2
4320.8 1461
4320 1460.2
4319 1460.2
4318.8 1460
4319 1459.8
4319.8 1459
4320 1458.8
4320.8 1458
4321 1457.8
4321.8 1457
4321.8 1456
4322 1455.8
4322.8 1455
4323 1454.8
4323.8 1454
4324 1453.8
4324.8 1453
4324.8 1452
4325 1451.8
4325.8 1451
4326 1450.8
4326.8 1450
4326.8 1449
4327 1448.8
4328 1448.8
4329 1448.8
4329.8 1448
4330 1447.8
4331 1447.8
4332 1447.8
4333 1447.8
4334 1447.8
4334.8 1447
4335 1446.8
4336 1446.8
4337 1446.8
4338 1446.8
4339 1446.8
4339.8 1446
4340 1445.8
4341 1445.8
4342 1445.8
4343 1445.8
4344 1445.8
4344.8 1445
4345 1444.8
4346 1444.8
4347 1444.8
4348 1444.8
4349 1444.8
4350 1444.8
4351 1444.8
4352 1444.8
4353 1444.8
4354 1444.8
4355 1444.8
4356 1444.8
4357 1444.8
4358 1444.8
4359 1444.8
4360 1444.8
4360.2 1445
4361 1445.8
4362 1445.8
4363 1445.8
4364 1445.8
4365 1445.8
4366 1445.8
4367 1445.8
4368 1445.8
4369 1445.8
4370 1445.8
4371 1445.8
4372 1445.8
4373 1445.8
4374 1445.8
4375 1445.8
4376 1445.8
4377 1445.8
4378 1445.8
4379 1445.8
4380 1445.8
4381 1445.8
4382 1445.8
4383 1445.8
4384 1445.8
4385 1445.8
4386 1445.8
4387 1445.8
4388 1445.8
4389 1445.8
4390 1445.8
4391 1445.8
4392 1445.8
4393 1445.8
4394 1445.8
4395 1445.8
4396 1445.8
4397 1445.8
4398 1445.8
4399 1445.8
4399.2 1446
4400 1446.8
4401 1446.8
4402 1446.8
4403 1446.8
4404 1446.8
4405 1446.8
4406 1446.8
4407 1446.8
4408 1446.8
4409 1446.8
4410 1446.8
4411 1446.8
4412 1446.8
4413 1446.8
4414 1446.8
4415 1446.8
4416 1446.8
4417 1446.8
4418 1446.8
4418.2 1447
4419 1447.8
4420 1447.8
4421 1447.8
4422 1447.8
4422.2 1448
4423 1448.8
4424 1448.8
4425 1448.8
4426 1448.8
4426.2 1449
4427 1449.8
4428 1449.8
4429 1449.8
4430 1449.8
4430.2 1450
4431 1450.8
4432 1450.8
4432.2 1451
4433 1451.8
4433.2 1452
4433.2 1453
4434 1453.8
4434.2 1454
4434.2 1455
4435 1455.8
4435.2 1456
4435.2 1457
4436 1457.8
4436.2 1458
4436.2 1459
4437 1459.8
4437.2 1460
4437 1460.2
4436.2 1461
4436 1461.2
4435.2 1462
4435 1462.2
4434.2 1463
4434 1463.2
4433.2 1464
4433 1464.2
4432.2 1465
4432 1465.2
4431.2 1466
4431.2 1467
4431 1467.2
4430.2 1468
4430 1468.2
4429 1468.2
4428.2 1469
4428 1469.2
4427 1469.2
4426.2 1470
4426 1470.2
4425 1470.2
4424 1470.2
4423.2 1471
4423 1471.2
4422 1471.2
4421.2 1472
4421 1472.2
4420 1472.2
4419 1472.2
4418.2 1473
4418 1473.2
4417 1473.2
4416 1473.2
4415 1473.2
4414 1473.2
4413.2 1474
4413 1474.2
4412 1474.2
4411 1474.2
4410 1474.2
4409 1474.2
4408 1474.2
4407.2 1475
4407 1475.2
4406 1475.2
4405 1475.2
4404 1475.2
4403 1475.2
4402 1475.2
4401 1475.2
4400 1475.2
4399.2 1476
4399 1476.2
};
\addplot [red]
table {%
6360 1623.2
6359 1623.2
6358 1623.2
6357 1623.2
6356 1623.2
6355 1623.2
6354 1623.2
6353 1623.2
6352 1623.2
6351 1623.2
6350 1623.2
6349 1623.2
6348 1623.2
6347 1623.2
6346 1623.2
6345.8 1623
6345 1622.2
6344 1622.2
6343 1622.2
6342 1622.2
6341 1622.2
6340 1622.2
6339.8 1622
6339 1621.2
6338 1621.2
6337 1621.2
6336 1621.2
6335 1621.2
6334.8 1621
6334 1620.2
6333 1620.2
6332 1620.2
6331 1620.2
6330 1620.2
6329 1620.2
6328.8 1620
6328 1619.2
6327 1619.2
6326 1619.2
6325 1619.2
6324 1619.2
6323.8 1619
6323 1618.2
6322 1618.2
6321 1618.2
6320 1618.2
6319 1618.2
6318.8 1618
6318 1617.2
6317 1617.2
6316.8 1617
6316 1616.2
6315.8 1616
6315 1615.2
6314.8 1615
6314 1614.2
6313 1614.2
6312.8 1614
6312 1613.2
6311.8 1613
6311 1612.2
6310.8 1612
6310 1611.2
6309.8 1611
6309 1610.2
6308 1610.2
6307.8 1610
6307 1609.2
6306.8 1609
6306 1608.2
6305.8 1608
6305 1607.2
6304 1607.2
6303.8 1607
6303 1606.2
6302.8 1606
6302 1605.2
6301.8 1605
6301 1604.2
6300 1604.2
6299.8 1604
6299 1603.2
6298.8 1603
6298 1602.2
6297.8 1602
6297 1601.2
6296.8 1601
6296 1600.2
6295.8 1600
6296 1599.8
6296.8 1599
6296.8 1598
6296.8 1597
6296.8 1596
6297 1595.8
6297.8 1595
6297.8 1594
6297.8 1593
6298 1592.8
6298.8 1592
6298.8 1591
6298.8 1590
6298.8 1589
6299 1588.8
6299.8 1588
6299.8 1587
6299.8 1586
6300 1585.8
6300.8 1585
6300.8 1584
6300.8 1583
6300.8 1582
6301 1581.8
6301.8 1581
6301.8 1580
6301.8 1579
6301.8 1578
6302 1577.8
6302.8 1577
6302.8 1576
6303 1575.8
6303.8 1575
6304 1574.8
6304.8 1574
6305 1573.8
6305.8 1573
6306 1572.8
6306.8 1572
6307 1571.8
6307.8 1571
6308 1570.8
6308.8 1570
6309 1569.8
6309.8 1569
6310 1568.8
6310.8 1568
6311 1567.8
6311.8 1567
6312 1566.8
6312.8 1566
6313 1565.8
6313.8 1565
6314 1564.8
6314.8 1564
6315 1563.8
6315.8 1563
6316 1562.8
6316.8 1562
6317 1561.8
6317.8 1561
6318 1560.8
6318.8 1560
6319 1559.8
6320 1559.8
6320.8 1559
6321 1558.8
6321.8 1558
6322 1557.8
6322.8 1557
6323 1556.8
6323.8 1556
6324 1555.8
6324.8 1555
6325 1554.8
6325.8 1554
6326 1553.8
6326.8 1553
6327 1552.8
6328 1552.8
6328.8 1552
6329 1551.8
6329.8 1551
6330 1550.8
6330.8 1550
6331 1549.8
6331.8 1549
6332 1548.8
6332.8 1548
6333 1547.8
6334 1547.8
6334.8 1547
6335 1546.8
6335.8 1546
6336 1545.8
6336.8 1545
6337 1544.8
6337.8 1544
6338 1543.8
6339 1543.8
6339.8 1543
6340 1542.8
6340.8 1542
6341 1541.8
6342 1541.8
6343 1541.8
6343.8 1541
6344 1540.8
6345 1540.8
6345.8 1540
6346 1539.8
6347 1539.8
6347.8 1539
6348 1538.8
6349 1538.8
6349.8 1538
6350 1537.8
6351 1537.8
6351.8 1537
6352 1536.8
6353 1536.8
6353.8 1536
6354 1535.8
6355 1535.8
6355.8 1535
6356 1534.8
6357 1534.8
6358 1534.8
6359 1534.8
6360 1534.8
6360.8 1534
6361 1533.8
6362 1533.8
6363 1533.8
6364 1533.8
6365 1533.8
6366 1533.8
6367 1533.8
6368 1533.8
6369 1533.8
6369.8 1533
6370 1532.8
6371 1532.8
6372 1532.8
6373 1532.8
6374 1532.8
6375 1532.8
6376 1532.8
6377 1532.8
6378 1532.8
6379 1532.8
6380 1532.8
6381 1532.8
6382 1532.8
6383 1532.8
6384 1532.8
6385 1532.8
6386 1532.8
6386.2 1533
6387 1533.8
6388 1533.8
6389 1533.8
6390 1533.8
6391 1533.8
6392 1533.8
6393 1533.8
6394 1533.8
6395 1533.8
6396 1533.8
6397 1533.8
6398 1533.8
6398.2 1534
6399 1534.8
6400 1534.8
6401 1534.8
6402 1534.8
6403 1534.8
6404 1534.8
6405 1534.8
6406 1534.8
6407 1534.8
6407.2 1535
6408 1535.8
6409 1535.8
6410 1535.8
6411 1535.8
6412 1535.8
6413 1535.8
6414 1535.8
6415 1535.8
6416 1535.8
6417 1535.8
6418 1535.8
6419 1535.8
6420 1535.8
6421 1535.8
6422 1535.8
6423 1535.8
6423.2 1536
6424 1536.8
6425 1536.8
6426 1536.8
6427 1536.8
6428 1536.8
6429 1536.8
6430 1536.8
6431 1536.8
6432 1536.8
6433 1536.8
6434 1536.8
6435 1536.8
6436 1536.8
6437 1536.8
6438 1536.8
6439 1536.8
6440 1536.8
6441 1536.8
6442 1536.8
6443 1536.8
6444 1536.8
6445 1536.8
6446 1536.8
6447 1536.8
6448 1536.8
6449 1536.8
6450 1536.8
6450.2 1537
6451 1537.8
6452 1537.8
6453 1537.8
6454 1537.8
6455 1537.8
6456 1537.8
6457 1537.8
6458 1537.8
6459 1537.8
6460 1537.8
6460.2 1538
6461 1538.8
6462 1538.8
6463 1538.8
6464 1538.8
6465 1538.8
6466 1538.8
6466.2 1539
6467 1539.8
6468 1539.8
6469 1539.8
6470 1539.8
6471 1539.8
6472 1539.8
6472.2 1540
6473 1540.8
6474 1540.8
6474.2 1541
6475 1541.8
6476 1541.8
6476.2 1542
6477 1542.8
6478 1542.8
6478.2 1543
6479 1543.8
6479.2 1544
6480 1544.8
6481 1544.8
6481.2 1545
6482 1545.8
6483 1545.8
6483.2 1546
6484 1546.8
6484.2 1547
6485 1547.8
6486 1547.8
6486.2 1548
6487 1548.8
6487.2 1549
6487.2 1550
6487.2 1551
6488 1551.8
6488.2 1552
6488.2 1553
6488.2 1554
6489 1554.8
6489.2 1555
6489.2 1556
6489.2 1557
6490 1557.8
6490.2 1558
6490.2 1559
6490.2 1560
6491 1560.8
6491.2 1561
6491.2 1562
6491.2 1563
6491.2 1564
6491 1564.2
6490.2 1565
6490.2 1566
6490.2 1567
6490.2 1568
6490.2 1569
6490 1569.2
6489.2 1570
6489.2 1571
6489.2 1572
6489.2 1573
6489.2 1574
6489.2 1575
6489 1575.2
6488.2 1576
6488.2 1577
6488 1577.2
6487.2 1578
6487.2 1579
6487 1579.2
6486.2 1580
6486.2 1581
6486 1581.2
6485.2 1582
6485.2 1583
6485 1583.2
6484.2 1584
6484.2 1585
6484 1585.2
6483.2 1586
6483.2 1587
6483 1587.2
6482.2 1588
6482 1588.2
6481.2 1589
6481 1589.2
6480.2 1590
6480 1590.2
6479.2 1591
6479 1591.2
6478 1591.2
6477.2 1592
6477 1592.2
6476.2 1593
6476 1593.2
6475.2 1594
6475 1594.2
6474 1594.2
6473.2 1595
6473 1595.2
6472.2 1596
6472 1596.2
6471.2 1597
6471 1597.2
6470 1597.2
6469.2 1598
6469 1598.2
6468 1598.2
6467.2 1599
6467 1599.2
6466 1599.2
6465.2 1600
6465 1600.2
6464 1600.2
6463.2 1601
6463 1601.2
6462 1601.2
6461.2 1602
6461 1602.2
6460 1602.2
6459.2 1603
6459 1603.2
6458 1603.2
6457 1603.2
6456.2 1604
6456 1604.2
6455 1604.2
6454.2 1605
6454 1605.2
6453 1605.2
6452 1605.2
6451.2 1606
6451 1606.2
6450 1606.2
6449 1606.2
6448.2 1607
6448 1607.2
6447 1607.2
6446.2 1608
6446 1608.2
6445 1608.2
6444.2 1609
6444 1609.2
6443 1609.2
6442.2 1610
6442 1610.2
6441 1610.2
6440.2 1611
6440 1611.2
6439 1611.2
6438 1611.2
6437.2 1612
6437 1612.2
6436 1612.2
6435.2 1613
6435 1613.2
6434 1613.2
6433 1613.2
6432.2 1614
6432 1614.2
6431 1614.2
6430 1614.2
6429.2 1615
6429 1615.2
6428 1615.2
6427 1615.2
6426.2 1616
6426 1616.2
6425 1616.2
6424 1616.2
6423 1616.2
6422.2 1617
6422 1617.2
6421 1617.2
6420 1617.2
6419 1617.2
6418.2 1618
6418 1618.2
6417 1618.2
6416 1618.2
6415 1618.2
6414.2 1619
6414 1619.2
6413 1619.2
6412 1619.2
6411 1619.2
6410 1619.2
6409.2 1620
6409 1620.2
6408 1620.2
6407 1620.2
6406 1620.2
6405 1620.2
6404.2 1621
6404 1621.2
6403 1621.2
6402 1621.2
6401 1621.2
6400 1621.2
6399 1621.2
6398 1621.2
6397 1621.2
6396 1621.2
6395 1621.2
6394 1621.2
6393 1621.2
6392 1621.2
6391 1621.2
6390 1621.2
6389.2 1622
6389 1622.2
6388 1622.2
6387 1622.2
6386 1622.2
6385 1622.2
6384 1622.2
6383 1622.2
6382 1622.2
6381 1622.2
6380 1622.2
6379 1622.2
6378 1622.2
6377 1622.2
6376 1622.2
6375 1622.2
6374 1622.2
6373 1622.2
6372 1622.2
6371 1622.2
6370 1622.2
6369 1622.2
6368 1622.2
6367 1622.2
6366 1622.2
6365 1622.2
6364 1622.2
6363 1622.2
6362 1622.2
6361 1622.2
6360.2 1623
6360 1623.2
};
\addplot [red]
table {%
5890 1572.2
5889.8 1572
5889 1571.2
5888 1571.2
5887 1571.2
5886 1571.2
5885 1571.2
5884 1571.2
5883 1571.2
5882 1571.2
5881.8 1571
5881 1570.2
5880 1570.2
5879 1570.2
5878 1570.2
5877.8 1570
5877 1569.2
5876 1569.2
5875 1569.2
5874 1569.2
5873.8 1569
5873 1568.2
5872 1568.2
5871 1568.2
5870 1568.2
5869.8 1568
5869 1567.2
5868 1567.2
5867 1567.2
5866 1567.2
5865 1567.2
5864 1567.2
5863.8 1567
5863 1566.2
5862 1566.2
5861 1566.2
5860 1566.2
5859 1566.2
5858 1566.2
5857.8 1566
5857 1565.2
5856 1565.2
5855 1565.2
5854 1565.2
5853 1565.2
5852.8 1565
5852 1564.2
5851 1564.2
5850 1564.2
5849 1564.2
5848 1564.2
5847 1564.2
5846.8 1564
5846 1563.2
5845 1563.2
5844 1563.2
5843 1563.2
5842.8 1563
5842 1562.2
5841 1562.2
5840.8 1562
5840 1561.2
5839.8 1561
5839 1560.2
5838 1560.2
5837.8 1560
5837 1559.2
5836.8 1559
5836 1558.2
5835 1558.2
5834.8 1558
5834 1557.2
5833.8 1557
5833 1556.2
5832 1556.2
5831.8 1556
5831 1555.2
5830.8 1555
5830 1554.2
5829 1554.2
5828.8 1554
5829 1553.8
5829.8 1553
5829.8 1552
5830 1551.8
5830.8 1551
5830.8 1550
5831 1549.8
5831.8 1549
5831.8 1548
5832 1547.8
5832.8 1547
5832.8 1546
5833 1545.8
5833.8 1545
5833.8 1544
5834 1543.8
5834.8 1543
5835 1542.8
5836 1542.8
5837 1542.8
5837.8 1542
5838 1541.8
5839 1541.8
5840 1541.8
5841 1541.8
5841.8 1541
5842 1540.8
5843 1540.8
5844 1540.8
5845 1540.8
5845.8 1540
5846 1539.8
5847 1539.8
5848 1539.8
5848.8 1539
5849 1538.8
5850 1538.8
5851 1538.8
5852 1538.8
5853 1538.8
5854 1538.8
5855 1538.8
5856 1538.8
5857 1538.8
5858 1538.8
5859 1538.8
5860 1538.8
5861 1538.8
5862 1538.8
5863 1538.8
5864 1538.8
5865 1538.8
5866 1538.8
5867 1538.8
5868 1538.8
5869 1538.8
5870 1538.8
5871 1538.8
5872 1538.8
5873 1538.8
5874 1538.8
5875 1538.8
5876 1538.8
5877 1538.8
5878 1538.8
5879 1538.8
5880 1538.8
5881 1538.8
5882 1538.8
5883 1538.8
5884 1538.8
5885 1538.8
5886 1538.8
5887 1538.8
5888 1538.8
5889 1538.8
5890 1538.8
5890.2 1539
5891 1539.8
5892 1539.8
5893 1539.8
5894 1539.8
5895 1539.8
5895.2 1540
5896 1540.8
5897 1540.8
5898 1540.8
5899 1540.8
5899.2 1541
5900 1541.8
5901 1541.8
5902 1541.8
5902.2 1542
5903 1542.8
5904 1542.8
5904.2 1543
5905 1543.8
5906 1543.8
5906.2 1544
5907 1544.8
5907.2 1545
5908 1545.8
5909 1545.8
5909.2 1546
5910 1546.8
5910.2 1547
5911 1547.8
5911.2 1548
5912 1548.8
5912.2 1549
5912.2 1550
5913 1550.8
5913.2 1551
5913.2 1552
5914 1552.8
5914.2 1553
5914.2 1554
5914.2 1555
5914.2 1556
5914.2 1557
5914.2 1558
5914.2 1559
5914 1559.2
5913.2 1560
5913.2 1561
5913 1561.2
5912.2 1562
5912.2 1563
5912 1563.2
5911.2 1564
5911 1564.2
5910.2 1565
5910 1565.2
5909.2 1566
5909 1566.2
5908.2 1567
5908 1567.2
5907.2 1568
5907 1568.2
5906.2 1569
5906 1569.2
5905 1569.2
5904.2 1570
5904 1570.2
5903 1570.2
5902 1570.2
5901.2 1571
5901 1571.2
5900 1571.2
5899 1571.2
5898 1571.2
5897 1571.2
5896 1571.2
5895 1571.2
5894 1571.2
5893 1571.2
5892 1571.2
5891 1571.2
5890.2 1572
5890 1572.2
};
\addplot [red]
table {%
6012 1599.2
6011 1599.2
6010 1599.2
6009 1599.2
6008 1599.2
6007 1599.2
6006 1599.2
6005 1599.2
6004 1599.2
6003 1599.2
6002 1599.2
6001 1599.2
6000 1599.2
5999 1599.2
5998 1599.2
5997.8 1599
5997 1598.2
5996.2 1599
5996 1599.2
5995 1599.2
5994 1599.2
5993 1599.2
5992 1599.2
5991 1599.2
5990 1599.2
5989 1599.2
5988 1599.2
5987 1599.2
5986 1599.2
5985 1599.2
5984 1599.2
5983 1599.2
5982 1599.2
5981 1599.2
5980 1599.2
5979 1599.2
5978 1599.2
5977 1599.2
5976 1599.2
5975 1599.2
5974 1599.2
5973 1599.2
5972 1599.2
5971 1599.2
5970 1599.2
5969 1599.2
5968 1599.2
5967 1599.2
5966 1599.2
5965 1599.2
5964 1599.2
5963 1599.2
5962 1599.2
5961.8 1599
5961 1598.2
5960 1598.2
5959 1598.2
5958 1598.2
5957.8 1598
5957 1597.2
5956 1597.2
5955 1597.2
5954 1597.2
5953.8 1597
5953 1596.2
5952 1596.2
5951 1596.2
5950.8 1596
5950 1595.2
5949 1595.2
5948 1595.2
5947 1595.2
5946.8 1595
5946 1594.2
5945 1594.2
5944 1594.2
5943.8 1594
5943 1593.2
5942 1593.2
5941 1593.2
5940 1593.2
5939.8 1593
5939 1592.2
5938 1592.2
5937.8 1592
5937.8 1591
5937.8 1590
5937.8 1589
5937.8 1588
5937.8 1587
5937.8 1586
5937.8 1585
5937.8 1584
5937.8 1583
5937.8 1582
5937.8 1581
5937 1580.2
5936.8 1580
5936.8 1579
5936.8 1578
5936.8 1577
5936.8 1576
5937 1575.8
5938 1575.8
5938.8 1575
5939 1574.8
5939.8 1574
5940 1573.8
5941 1573.8
5941.8 1573
5942 1572.8
5942.8 1572
5943 1571.8
5943.8 1571
5944 1570.8
5945 1570.8
5945.8 1570
5946 1569.8
5946.8 1569
5947 1568.8
5948 1568.8
5948.8 1568
5949 1567.8
5949.8 1567
5950 1566.8
5950.8 1566
5951 1565.8
5952 1565.8
5952.8 1565
5953 1564.8
5953.8 1564
5954 1563.8
5955 1563.8
5955.8 1563
5956 1562.8
5957 1562.8
5957.8 1562
5958 1561.8
5959 1561.8
5960 1561.8
5960.8 1561
5961 1560.8
5962 1560.8
5962.8 1560
5963 1559.8
5964 1559.8
5964.8 1559
5965 1558.8
5966 1558.8
5967 1558.8
5967.8 1558
5968 1557.8
5969 1557.8
5969.8 1557
5970 1556.8
5971 1556.8
5971.8 1556
5972 1555.8
5973 1555.8
5974 1555.8
5975 1555.8
5976 1555.8
5976.8 1555
5977 1554.8
5978 1554.8
5979 1554.8
5980 1554.8
5980.8 1554
5981 1553.8
5982 1553.8
5983 1553.8
5984 1553.8
5985 1553.8
5985.8 1553
5986 1552.8
5987 1552.8
5988 1552.8
5989 1552.8
5990 1552.8
5991 1552.8
5992 1552.8
5993 1552.8
5994 1552.8
5995 1552.8
5996 1552.8
5997 1552.8
5998 1552.8
5999 1552.8
6000 1552.8
6001 1552.8
6002 1552.8
6002.8 1552
6003 1551.8
6004 1551.8
6005 1551.8
6006 1551.8
6007 1551.8
6008 1551.8
6009 1551.8
6010 1551.8
6011 1551.8
6012 1551.8
6013 1551.8
6014 1551.8
6015 1551.8
6016 1551.8
6017 1551.8
6018 1551.8
6019 1551.8
6020 1551.8
6021 1551.8
6021.2 1552
6022 1552.8
6023 1552.8
6024 1552.8
6025 1552.8
6026 1552.8
6027 1552.8
6028 1552.8
6029 1552.8
6029.2 1553
6030 1553.8
6031 1553.8
6032 1553.8
6033 1553.8
6034 1553.8
6035 1553.8
6035.2 1554
6036 1554.8
6037 1554.8
6038 1554.8
6038.2 1555
6039 1555.8
6040 1555.8
6040.2 1556
6041 1556.8
6042 1556.8
6042.2 1557
6043 1557.8
6043.2 1558
6044 1558.8
6045 1558.8
6045.2 1559
6046 1559.8
6046.2 1560
6047 1560.8
6047.2 1561
6048 1561.8
6048.2 1562
6049 1562.8
6049.2 1563
6050 1563.8
6050.2 1564
6051 1564.8
6051.2 1565
6052 1565.8
6052.2 1566
6053 1566.8
6053.2 1567
6053.2 1568
6054 1568.8
6054.2 1569
6055 1569.8
6055.2 1570
6055.2 1571
6055.2 1572
6056 1572.8
6056.2 1573
6056.2 1574
6056.2 1575
6057 1575.8
6057.2 1576
6057.2 1577
6057.2 1578
6057.2 1579
6057.2 1580
6057.2 1581
6057.2 1582
6057.2 1583
6057.2 1584
6057 1584.2
6056.2 1585
6056 1585.2
6055.2 1586
6055.2 1587
6055 1587.2
6054.2 1588
6054 1588.2
6053.2 1589
6053.2 1590
6053 1590.2
6052 1590.2
6051.2 1591
6051 1591.2
6050.2 1592
6050 1592.2
6049 1592.2
6048.2 1593
6048 1593.2
6047.2 1594
6047 1594.2
6046 1594.2
6045 1594.2
6044.2 1595
6044 1595.2
6043 1595.2
6042 1595.2
6041.2 1596
6041 1596.2
6040 1596.2
6039 1596.2
6038 1596.2
6037 1596.2
6036.2 1597
6036 1597.2
6035 1597.2
6034 1597.2
6033 1597.2
6032 1597.2
6031 1597.2
6030 1597.2
6029 1597.2
6028 1597.2
6027 1597.2
6026 1597.2
6025 1597.2
6024 1597.2
6023 1597.2
6022 1597.2
6021 1597.2
6020.2 1598
6020 1598.2
6019 1598.2
6018 1598.2
6017 1598.2
6016 1598.2
6015 1598.2
6014 1598.2
6013 1598.2
6012.2 1599
6012 1599.2
};
\addplot [red]
table {%
5437 1925.2
5436 1925.2
5435 1925.2
5434 1925.2
5433 1925.2
5432 1925.2
5431 1925.2
5430 1925.2
5429 1925.2
5428 1925.2
5427 1925.2
5426 1925.2
5425 1925.2
5424.8 1925
5424 1924.2
5423 1924.2
5422 1924.2
5421 1924.2
5420 1924.2
5419 1924.2
5418.8 1924
5418 1923.2
5417 1923.2
5416.8 1923
5416 1922.2
5415.8 1922
5415 1921.2
5414.8 1921
5414 1920.2
5413.8 1920
5413.8 1919
5413 1918.2
5412.8 1918
5412.8 1917
5412.8 1916
5412.8 1915
5412.8 1914
5413 1913.8
5413.8 1913
5413.8 1912
5414 1911.8
5414.8 1911
5414.8 1910
5414.8 1909
5415 1908.8
5415.8 1908
5416 1907.8
5416.8 1907
5417 1906.8
5417.8 1906
5417.8 1905
5418 1904.8
5418.8 1904
5419 1903.8
5419.8 1903
5420 1902.8
5421 1902.8
5421.8 1902
5421 1901.2
5420 1901.2
5419 1901.2
5418 1901.2
5417.8 1901
5417 1900.2
5416 1900.2
5415 1900.2
5414.8 1900
5414 1899.2
5413.2 1900
5414 1900.8
5414.2 1901
5414.2 1902
5414.2 1903
5414.2 1904
5414.2 1905
5414.2 1906
5414.2 1907
5414 1907.2
5413.2 1908
5413.2 1909
5413 1909.2
5412.2 1910
5412 1910.2
5411.2 1911
5411 1911.2
5410.2 1912
5410 1912.2
5409.2 1913
5409 1913.2
5408.2 1914
5408 1914.2
5407 1914.2
5406 1914.2
5405 1914.2
5404 1914.2
5403 1914.2
5402 1914.2
5401.8 1914
5401 1913.2
5400 1913.2
5399.8 1913
5399 1912.2
5398 1912.2
5397.8 1912
5397 1911.2
5396 1911.2
5395.8 1911
5395 1910.2
5394.8 1910
5394 1909.2
5393.8 1909
5393 1908.2
5392.8 1908
5392 1907.2
5391.8 1907
5391 1906.2
5390.8 1906
5390.8 1905
5390 1904.2
5389.8 1904
5389 1903.2
5388.8 1903
5388.8 1902
5388 1901.2
5387.8 1901
5387.8 1900
5387.8 1899
5387.8 1898
5387.8 1897
5387.8 1896
5388 1895.8
5388.8 1895
5388.8 1894
5389 1893.8
5390 1893.8
5390.8 1893
5391 1892.8
5392 1892.8
5392.8 1892
5393 1891.8
5394 1891.8
5395 1891.8
5396 1891.8
5396.8 1891
5397 1890.8
5398 1890.8
5399 1890.8
5400 1890.8
5401 1890.8
5402 1890.8
5402.2 1891
5403 1891.8
5404 1891.8
5405 1891.8
5405.8 1891
5405 1890.2
5404.8 1890
5404.8 1889
5404.8 1888
5404 1887.2
5403.8 1887
5403.8 1886
5403.8 1885
5403.8 1884
5403.8 1883
5403.8 1882
5403.8 1881
5403.8 1880
5403.8 1879
5403.8 1878
5404 1877.8
5404.8 1877
5404.8 1876
5405 1875.8
5405.8 1875
5406 1874.8
5406.8 1874
5407 1873.8
5408 1873.8
5408.8 1873
5409 1872.8
5410 1872.8
5410.8 1872
5411 1871.8
5412 1871.8
5413 1871.8
5414 1871.8
5415 1871.8
5416 1871.8
5417 1871.8
5418 1871.8
5419 1871.8
5420 1871.8
5421 1871.8
5422 1871.8
5423 1871.8
5424 1871.8
5425 1871.8
5426 1871.8
5426.2 1872
5427 1872.8
5428 1872.8
5429 1872.8
5429.2 1873
5430 1873.8
5431 1873.8
5432 1873.8
5432.2 1874
5433 1874.8
5433.2 1875
5434 1875.8
5434.2 1876
5435 1876.8
5435.2 1877
5435.2 1878
5436 1878.8
5436.2 1879
5436.2 1880
5436.2 1881
5436.2 1882
5436.2 1883
5436.2 1884
5437 1884.8
5437.2 1885
5437.2 1886
5437 1886.2
5436.2 1887
5436.2 1888
5436.2 1889
5436 1889.2
5435.2 1890
5435.2 1891
5435.2 1892
5435 1892.2
5434.2 1893
5434 1893.2
5433.2 1894
5433.2 1895
5433 1895.2
5432.2 1896
5432 1896.2
5431.2 1897
5431 1897.2
5430.2 1898
5430 1898.2
5429 1898.2
5428.2 1899
5429 1899.8
5430 1899.8
5430.8 1899
5431 1898.8
5432 1898.8
5433 1898.8
5434 1898.8
5434.8 1898
5435 1897.8
5436 1897.8
5437 1897.8
5438 1897.8
5439 1897.8
5440 1897.8
5441 1897.8
5442 1897.8
5443 1897.8
5444 1897.8
5444.2 1898
5445 1898.8
5446 1898.8
5446.2 1899
5447 1899.8
5448 1899.8
5448.2 1900
5449 1900.8
5449.2 1901
5450 1901.8
5450.2 1902
5451 1902.8
5451.2 1903
5451.2 1904
5452 1904.8
5452.2 1905
5452.2 1906
5453 1906.8
5453.2 1907
5453.2 1908
5453.2 1909
5453.2 1910
5453.2 1911
5453.2 1912
5453 1912.2
5452.2 1913
5452.2 1914
5452.2 1915
5452 1915.2
5451.2 1916
5451 1916.2
5450.2 1917
5450.2 1918
5450 1918.2
5449.2 1919
5449 1919.2
5448 1919.2
5447.2 1920
5447 1920.2
5446.2 1921
5446 1921.2
5445 1921.2
5444.2 1922
5444 1922.2
5443 1922.2
5442.2 1923
5442 1923.2
5441 1923.2
5440.2 1924
5440 1924.2
5439 1924.2
5438 1924.2
5437.2 1925
5437 1925.2
};
\addplot [red]
table {%
5427.8 1900
5427 1899.2
5426.2 1900
5427 1900.8
5427.8 1900
};
\end{axis}

\end{tikzpicture}

    \end{minipage}
    \item Create a binary mask for quadratic regions of interest (ROI)
    \begin{enumerate}
        \item Select the size of these ROIs and crop the dimension of the images to multiples of it.
        \item Pad the edges of the images.
        \item Create a binary mask for ROIs based on a threshold value \(t_{\text{ROI}}\) for the percentage of white pixels in the ROI \(p_{\text{white}}\):
        \(I \to 
        \begin{cases}
            0 & p_{\text{white}} < t_{\text{ROI}}\\
            1 & p_{\text{white}} \geq t_{\text{ROI}}
        \end{cases}
        \).
    \end{enumerate}
        \begin{minipage}{0.5\textwidth}
        % This file was created with tikzplotlib v0.10.1.
\begin{tikzpicture}

\definecolor{darkgray176}{RGB}{176,176,176}

\begin{axis}[
height = \textwidth*0.29357798165137616,
hide x axis,
hide y axis,
tick align=outside,
tick pos=left,
title={ROI Mask (size = 30, $t_{\text{ROI}}$ = 0.3)},
width=\textwidth,
x grid style={darkgray176},
xmin=-0.5, xmax=217.5,
xtick style={color=black},
y dir=reverse,
y grid style={darkgray176},
ymin=-0.5, ymax=63.5,
ytick style={color=black}
]
\addplot graphics [includegraphics cmd=\pgfimage,xmin=-0.5, xmax=217.5, ymin=63.5, ymax=-0.5] {figures/roi_mask-000.png};
\end{axis}

\end{tikzpicture}

    \end{minipage}
\end{enumerate}

\clearpage

\end{multicols}
\end{document}